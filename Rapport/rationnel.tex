\subsection{Fonctions de parking rationnelles}

Cette partie est illustrée par les exemples 17 à 19 de l'annexe C.

\begin{definition}[a, b - Fonction de Parking]
    Une \emph{a, b - fonction de parking} est une séquence d'entiers
    positifs $(a_1, a_2, \ldots, a_n)$ telle que :\\
    \begin{itemize*}
        \item $n = a$\\
        \item son tri croissant $(b_1, b_2, \ldots, b_n)$ respecte la
        condition suivante :$b_i \leqslant \frac{b}{a}(i-1) + 1$ 
        pour tout $i \leqslant n$.
    \end{itemize*}
\end{definition}

On note $\mathcal{PF}_{a,b}$ l'ensemble des a, b - fonctions de parking. 

\begin{theorem}[Armstrong, Loehr et Warrington, 2014]
    Soit $pf_{a,b}$ le cardinal de $\mathcal{PF}_{a,b}$.
    Nous avons $$pf_{a,b} = b^{a-1}$$
\end{theorem}

\begin{rem}
Le cas classique peut être vu comme le cas $a = n, b = n + 1$.
Autrement dit, $\mathcal{PF}_{n, n + 1} = \mathcal{PF}_n$.
\end{rem}

Similairement au cas classique, on définit une fonction de parking
\emph{rationnelle primitive} comme une fonction de parking primitive
triée en ordre croissant.
On note $\mathcal{PF'}_{a,b}$ l'ensemble des a, b - fonctions de parking
primitives.

\begin{theorem}
    Soit $pf'_{a,b}$ le cardinal de $\mathcal{PF'}_{a,b}$.
    Nous avons $$\displaystyle pf'_{a,b} = 
    \frac{1}{a + b} \binom{a + b}{b}$$
\end{theorem}

Ce nombre est appelé le \emph{nombre de Catalan rationnel}, et on le note
$Cat(a,b)$.
Là aussi, le cas classique correspond à $a = n, b = n + 1$.
Autrement dit, $\mathcal{PF'}_{n, n + 1} = \mathcal{PF'}_n$.

\subsection{Chemins de Dyck rationnels}

Cette partie est illustrée par les exemples 20 à XXX de l'annexe C.

\begin{definition}[a, b - mot de Dyck]
    Un \emph{a, b - mot de Dyck} est un mot $w \in \{0,1\}^*$ tel que :
    \begin{itemize}
        \item pour tout \emph{suffixe} $w'$ de $w$,
            $\displaystyle |w'|_1 \geqslant \frac{a}{b}|w'|_0$.
        \item $|w|_0 = b$.
        \item $|w|_1 = a$.
    \end{itemize}
\end{definition}

Un a, b - mot de Dyck peut être représenté par un chemin allant du point
$(0,0)$ au point $(b,a)$, et restant au dessus de l'axe $y = \frac{a}{b}x$,
appelé \emph{a, b - chemin de Dyck} :
\begin{itemize}
    \item Chaque $1$ correspond à un \emph{pas Nord} $\uparrow$. 
    \item Chaque $0$ correspond à un \emph{pas Est} $\rightarrow$.
\end{itemize}

On note $\mathcal{R}_{a, b}$ l'ensemble des a, b - mots de Dyck.

\begin{theorem}[Bizley, 1954]
    Soit $r_{a,b}$ le cardinal de $\mathcal{R}_{a,b}$.
    Nous avons $$r_{a,b} = \frac{1}{a+b} \binom {a+b}{a} =
    \frac{(a+b-1)!}{a!b!}$$
\end{theorem}

\subsection{Posets rationnels}