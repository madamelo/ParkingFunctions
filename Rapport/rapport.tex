\documentclass[11pt]{article}

\usepackage{amsmath}
\usepackage{amssymb}
\usepackage{amsthm}

\usepackage[francais]{babel}
\setlength{\parskip}{0.3\baselineskip}

\usepackage{graphicx}
\graphicspath{ {./} }

\usepackage[utf8]{inputenc}
\usepackage[T1]{fontenc}

\usepackage[inline]{enumitem}

\usepackage{a4wide}

\usepackage{hyperref}

\usepackage{tikz}
\usetikzlibrary{shapes.multipart}
\usetikzlibrary{arrows}
\usetikzlibrary{patterns}

\newtheorem{theorem}{Theorem}
\newtheorem{lemma}{Lemma}
\newtheorem*{prop}{Proposition}
\newtheorem{definition}{Definition}
\newtheorem*{example}{Example}
\newtheorem*{notation}{Notation}
\newtheorem*{cor}{Corollary}
\newtheorem*{rem}{Remark}
\newtheorem*{conj}{Conjecture}

\begin{document}

\title{Fonctions de Parking Classiques et Rationnelles}
\author{Tessa Lelièvre-Osswald\\
    {\small Encadrant: Matthieu Josuat-Vergès}\\
    {\small Equipe: IRIF - Pôle Combinatoire}}
\date{\today}

\maketitle

\subsection*{Contexte général}

La \emph{combinatoire} est un domaine des mathématiques et de 
l'informatique théorique étudiant les ensembles finis par leur énumération
et leur comptage.
Se divisant en plusieurs branches, nous abordons dans ce rapport deux
branches principales : la \emph{combinatoire énumérative}, domaine le plus
classique, basé sur le dénombrement; et \emph{la combinatoire bijective},
consistant à déduire une égalité entre les résultats de comptage de deux
classes combinatoires qui sont en bijection. 

On s'intéresse ici en particulier à des objets combinatoires appelés
\emph{fonctions de parking}. Introduites en 1966 par Konheim et Weiss
(\cite{ref1}), les fonctions de parking sont les séquences d'entiers
positifs dont le nombre d'éléments \emph{inférieurs ou égaux} à $i$ est
\emph{supérieur ou égal} à $i$ pour tout entier $i$ entre $1$ et $n$.\\
Leur appellation vient du problème de hachage suivant :
soient un parking contenant $n$ places numérotées de $1$ à $n$, et $n$
voitures à garer.
A chaque voiture $i$ est associé un entier $a_i$, indiquant que cette
voiture doit être garée à une place dont le numéro est supérieur ou égal
à $a_i$.
La séquence $(a_1, \ldots, a_n)$ est alors appelée fonction de parking si
et seulement si il existe une configuration de garage des $n$ voitures
respectant les contraintes données par les entiers $a_i$.

Longuement étudiées, les principaux résultats de comptage émanent des
travaux de Stanley (\cite{ref2}, \cite{ref3}), Kreweras (\cite{ref4}),
et Edelman (\cite{ref5}).
Plus récemment, la notion de fonction de parking \emph{rationnelle} à été
introduite et étudiée -- entre autres -- dans les travaux d'Armstrong, Loehr
et Warrington (\cite{ref7}), ainsi que de Bodnar (\cite{ref8}).
Bien que les travaux mentionnées ci-dessus concernent principalement le
comptage des fonctions de parking et des relations de leurs treillis, 
ces dernières ont des applications dans de nombreux domaines tels que
l'analyse et la géométrie.

\subsection*{Problèmes étudiés}

Dans cet article, nous abordons deux principaux problèmes.

Premièrement, à notre connaissance, les treillis proposés jusque là pour
les fonctions de parking dépendent tous de bijections entre les fonctions
de parking et une autre structure combinatoire, pour laquelle un treillis
était déjà défini. Cela rend la définition d'une relation de couverture
assez lourde, et rajoute des étapes au processus de comparaison.

Ensuite, l'extension au cas rationnel étant en plein essor dans les travaux
les plus récents, de nombreux concept restent à redéfinir en dehors du cas
classique.

\subsection*{Contribution proposée}
Nous présentons ici un nouveau treillis pour les fonctions de
parking, dans le cas classique ainsi que dans le cas rationnel.
Ainsi, nous introduisons une relation de couverture plus élégante,
définie sans structure intermédiaire.
De plus, celle-ci est en bijection avec une relation naturelle que nous
définissons sur les chemins de Dyck, qui sont une structure élémentaire
de la combinatoire.

Enfin, nous reprenons la notion d'\emph{arbres de parking} donnée par
Delcroix-Oger, Josuat-Vergès et Randazzo (\cite{ref9}), afin d'en donner
une version rationnelle.

Pour ces deux contributions, le cas rationnel est traité pour toute paire
d'entiers premiers entre eux -- sans se limiter au cas $a < b$ comme il a
pu être fait dans certains travaux.

Une version longue de ce rapport est disponible 
\href{https://github.com/tessalsifi/ParkingFunctions}{ici}
\footnote{github.com/tessalsifi/ParkingFunctions}, ainsi que l'encodage
en Sage des principales notions abordées dans les références 1 à 8 et des
constructions que nous présentons.

\subsection*{Arguments en faveur de sa validité}
Les treillis introduits dans le cas classique sont à l'origine de nos
deux résultats principaux.

Dans le cas des fonctions de parking \emph{primitives}, nous exhibons
une preuve du Théorème 8, qui donne le nombre d'intervalles dans le
treillis.
Quant à son équivalent pour le cas non-primitif, nous établissons une
Conjecture sur le nombre d'intervalles, vérifiée sur les cas
$n = 1, \ldots, 8$.

\subsection*{Bilan et perspectives}
En conclusion, nous avons maintenant des treillis définis de manière
directe pour les quatre types de fonctions de parking : classiques, 
classiques primitives, rationnelles, et rationnelles primitives.

Par la suite, il sera nécessaire de trouver une preuve de la Conjecture,
ou bien si elle doit être réfutée, d'exhiber une formule comptant les
intervalles dans notre treillis des foncions de parking classiques.
Quant au cas rationnel, il faudra également trouver des formules exprimant
le nombre de relations.

Enfin, l'on pourrait également vouloir étudier les relations de couvertures
sur les arbres de parking rationnels. 

\newpage

\tableofcontents

\bibliographystyle{unsrt}
\bibliography{ref}

\end{document}