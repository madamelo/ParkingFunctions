\documentclass[11pt]{article}

\usepackage{amsmath}
\usepackage{amssymb}
\usepackage{amsthm}

\usepackage[francais]{babel}
\setlength{\parskip}{0.3\baselineskip}

\usepackage{graphicx}
\graphicspath{ {./} }

\usepackage[utf8]{inputenc}
\usepackage[T1]{fontenc}

\usepackage[inline]{enumitem}

\usepackage{a4wide}

\usepackage{hyperref}

\usepackage{tikz}
\usetikzlibrary{shapes.multipart}
\usetikzlibrary{arrows}
\usetikzlibrary{patterns}

\newtheorem{theorem}{Théorème}
\newtheorem{lemma}{Lemme}
\newtheorem{prop}{Proposition}
\newtheorem{definition}{Définition}
\newtheorem{example}{Exemple}
\newtheorem*{expl}{Exemple} %exemples non numérotés
\newtheorem*{notation}{Notation}
\newtheorem*{cor}{Corollaire}
\newtheorem*{rem}{Remarque}
\newtheorem*{conj}{Conjecture}

\begin{document}

\title{Fonctions de Parking Classiques et Rationnelles}
\author{Tessa Lelièvre-Osswald\\
    {\small Encadrant: Matthieu Josuat-Vergès}\\
    {\small Equipe: IRIF - Pôle Combinatoire}}
\date{\today}

\maketitle

\subsection*{Contexte général}

La \emph{combinatoire} est un domaine des mathématiques et de 
l'informatique théorique étudiant les ensembles finis par leur énumération
et leur comptage.
Se divisant en plusieurs branches, nous abordons dans ce rapport deux
branches principales : la \emph{combinatoire énumérative}, domaine le plus
classique, basé sur le dénombrement; et \emph{la combinatoire bijective},
consistant à déduire une égalité entre les résultats de comptage de deux
classes combinatoires qui sont en bijection. 

On s'intéresse ici en particulier à des objets combinatoires appelés
\emph{fonctions de parking}. Introduites en 1966 par Konheim et Weiss
(\cite{ref1}), les fonctions de parking sont les séquences d'entiers
positifs dont le nombre d'éléments \emph{inférieurs ou égaux} à $i$ est
\emph{supérieur ou égal} à $i$ pour tout entier $i$ entre $1$ et $n$.\\
Leur appellation vient du problème de hachage suivant :
soient un parking contenant $n$ places numérotées de $1$ à $n$, et $n$
voitures à garer.
A chaque voiture $i$ est associé un entier $a_i$, indiquant que cette
voiture doit être garée à une place dont le numéro est supérieur ou égal
à $a_i$.
La séquence $(a_1, \ldots, a_n)$ est alors appelée fonction de parking si
et seulement si il existe une configuration de garage des $n$ voitures
respectant les contraintes données par les entiers $a_i$.

Longuement étudiées, les principaux résultats de comptage émanent des
travaux de Stanley (\cite{ref2}, \cite{ref3}), Kreweras (\cite{ref4}),
et Edelman (\cite{ref5}).
Plus récemment, la notion de fonction de parking \emph{rationnelle} à été
introduite et étudiée -- entre autres -- dans les travaux d'Armstrong, Loehr
et Warrington (\cite{ref7}), ainsi que de Bodnar (\cite{ref8}).
Bien que les travaux mentionnées ci-dessus concernent principalement le
comptage des fonctions de parking et des relations de leurs posets, 
ces dernières ont des applications dans de nombreux domaines tels que
l'analyse et la géométrie.

\subsection*{Problèmes étudiés}

Dans cet article, nous abordons deux principaux problèmes.

Premièrement, à notre connaissance, les posets proposés jusque là pour
les fonctions de parking dépendent tous de bijections entre les fonctions
de parking et une autre structure combinatoire, pour laquelle un poset
était déjà défini. Cela rend la définition d'une relation de couverture
assez lourde, et rajoute des étapes au processus de comparaison.

Ensuite, l'extension au cas rationnel étant en plein essor dans les travaux
les plus récents, de nombreux concept restent à redéfinir en dehors du cas
classique.

\subsection*{Contribution proposée}
Nous présentons ici un nouveau poset pour les fonctions de
parking, dans le cas classique ainsi que dans le cas rationnel.
Ainsi, nous introduisons une relation de couverture plus élégante,
définie sans structure intermédiaire.
De plus, celle-ci est en bijection avec une relation naturelle que nous
définissons sur les chemins de Dyck, qui sont une structure élémentaire
de la combinatoire.

Enfin, nous reprenons la notion d'\emph{arbres de parking} donnée par
Delcroix-Oger, Josuat-Vergès et Randazzo (\cite{ref9}), afin d'en donner
une version rationnelle.

Pour ces deux contributions, le cas rationnel est traité pour toute paire
d'entiers premiers entre eux -- sans se limiter au cas $a < b$ comme il a
pu être fait dans certains travaux.

Une version longue de ce rapport est disponible 
\href{https://github.com/tessalsifi/ParkingFunctions}{ici}
\footnote{github.com/tessalsifi/ParkingFunctions}, ainsi que l'encodage
en Sage des principales notions abordées dans les références 1 à 8 et des
constructions que nous présentons.

\subsection*{Arguments en faveur de sa validité}
Les posets introduits dans le cas classique sont à l'origine de nos
deux résultats principaux.

Dans le cas des fonctions de parking \emph{primitives}, nous exhibons
une preuve du Théorème 5, qui donne le nombre d'intervalles dans le
poset.
Quant à son équivalent pour le cas non-primitif, nous établissons une
Conjecture sur le nombre d'intervalles, vérifiée sur les cas
$n = 1, \ldots, 8$.

\subsection*{Bilan et perspectives}
En conclusion, nous avons maintenant des posets définis de manière
directe pour les quatre types de fonctions de parking : classiques, 
classiques primitives, rationnelles, et rationnelles primitives.

Par la suite, il sera nécessaire de trouver une preuve de la Conjecture,
ou bien si elle doit être réfutée, d'exhiber une formule comptant les
intervalles dans notre poset des foncions de parking classiques.
Quant au cas rationnel, il faudra également trouver des formules exprimant
le nombre de relations.

Enfin, l'on pourrait également vouloir étudier les relations de couvertures
sur les arbres de parking rationnels. 

\newpage

\tableofcontents

\section{Introduction}
Nous commençons par rappeler les définitions et résultats principaux sur
les structures utilisées dans cet article.

\subsection{Fonctions de Parking}

En annexe A, les exemples 1 à 4 illustrent les définitions et théorèmes
de cette section.

\begin{definition}[Fonction de Parking]
    Une \emph{fonction de parking} est une séquence d'entiers strictement
    positifs $(a_1, a_2, \ldots, a_n)$ dont le tri croissant
    $(b_1, b_2, \ldots, b_n)$ respecte la condition suivante :
    $b_i \leqslant i$ pour tout $i \leqslant n$.
\end{definition}

En d'autres termes, $\#\{i\ |\ a_i \leqslant k\} \geqslant k\ 
\forall k \leqslant n$.

On note $\mathcal{PF}_n$ l'ensemble des fonctions de parking de longueur $n$.

\begin{theorem}[Konheim et Weiss, 1966]
    Soit $pf_n$ le cardinal de $\mathcal{PF}_n$.
    Nous avons $$pf_n = (n + 1)^{n-1}.$$
\end{theorem}

\begin{definition}[Fonction de Parking Primitive]
    Une fonction de parking $(a_1, a_2, \ldots, a_n)$ est dite
    \emph{primitive} si elle est déjà triée en ordre croissant.    
\end{definition}

On note $\mathcal{PF'}_n$ l'ensemble des fonctions de parking primitives
de longueur $n$.

\begin{theorem}[Stanley, 1999]
    Soit $pf'_n$ le cardinal de $\mathcal{PF'}_n$.
    Nous avons $$pf'_n = \frac{1}{n + 1} \binom{2n}{n}.$$
\end{theorem}

Ce nombre est le $n^{e}$ nombre de Catalan $Cat(n)$.

\subsection{Chemins de Dyck}

En annexe A, les exemples 5 à 8 illustrent les définitions et théorèmes
de cette section.

\begin{notation}
    On note par $|w|_s$ le nombre d'occurences du symbole $s$ dans
    le mot $w$ .
\end{notation}

\begin{definition}[Mot de Dyck]
    Un \emph{mot de Dyck} est un mot $w \in \{0,1\}^*$ tel que :
    \begin{itemize}
        \item pour tout \emph{préfixe} $w'$ de $w$,
            $|w'|_1 \geqslant |w'|_0$.
        \item $|w|_0 = |w|_1$.
    \end{itemize}
\end{definition}

Un mot de Dyck de longueur $2n$ peut être représenté par un \emph{chemin}
allant du point $(0,0)$ au point $(n,n)$, et restant au dessus de l'axe
$y = x$, appelé \emph{chemin de Dyck} :
\begin{itemize}
    \item Chaque $1$ correspond à un \emph{pas Nord}
    $\uparrow$. 
    \item Chaque $0$ correspond à un \emph{pas Est}
    $\rightarrow$.
\end{itemize}

On note $\mathcal{D}_n$ l'ensemble des mots de Dyck de longeur $2n$.

\begin{theorem}[André, 1887]
    Soit $d_n$ le cardinal de $\mathcal{D}_n$.
    Nous avons $$d_n = \frac{1}{n + 1} \binom {2n}{n}.$$
\end{theorem}

\begin{definition}[Mot de Dyck Décoré]
    Un \emph{mot de Dyck décoré} est un mot $w \in 
    \{0, \ldots, n\}^*$ tel que :
    \begin{itemize}
        \item pour tout préfixe $w'$ de $w$,
            $|w'|_{\neq 0} \geqslant |w'|_0$.
        \item $|w|_0 = |w|_{\neq 0}$.
        \item pour tout $i \in \{1, \ldots, n\}$, $w$ contient
            exactement une occurence de $i$.
        \item si $w_i \neq 0$ et $w_{i+1} \neq 0$,
            alors $w_i < w_{i+1}$. Autrement dit, les labels de pas Nord
            consécutifs doivent être croissants.
    \end{itemize}
\end{definition}

Un mot de Dyck décoré de longueur $2n$ peut être représenté par un 
\emph{chemin} allant du point $(0,0)$ au point $(n,n)$, où chaque pas
North est associé à un label :
\begin{itemize}
    \item Chaque $i \neq 0$ correspond à un \emph{pas Nord} $\uparrow$
    de label $i$.
    \item Chaque $0$ corresponds à un \emph{pas Est} $\rightarrow$.
\end{itemize}

Ces chemins sont appelés \emph{chemins de Dyck décorés}.\\
On note $\mathcal{LD}_n$ l'ensemble des mots de Dyck décorés de longueur
$2n$.

\begin{theorem}
    Soit $ld_n$ le cardinal de $\mathcal{LD}_n$.
    Nous avons $$ld_n = (n + 1)^{n - 1}.$$
\end{theorem}

Ces notions et résultats vont nous permettre de définir deux bijections
et quatre posets pour les quatre ensembles présentés.

\newpage
\section{Un poset pour les fonctions de parking classiques}
\subsection{Le cas primitif}

Les exemples 9 à 13 donnés en annexe B illustrent les propositions,
définitions et théorèmes de cette section.

Puisque $\mathcal{PF'}_n$ et $\mathcal{D}_n$ ont le même cardinal,
nous pouvons créer une bijection entre les deux ensembles.

\begin{prop}
    Il existe une \emph{bijection explicite} entre les fonctions de parking
    classiques primitives de longueur $n$ et les mots de Dyck de
    longueur $2n$.
\end{prop}

\begin{proof}
    Nous créons ici deux bijections inverse l'une de l'autre.
\begin{itemize}
    \item $\mathcal{PF'}_n \to \mathcal{D}_n$ :
    Soit $f = (a_1, \ldots, a_n) \in \mathcal{PF'}_n$
    une fonction de parking classique primitive.
    Pour tout $i \in \{1, \ldots, n\}$, notons $l_i$ le nombre
    d'occurences de $i$ dans $f$.\\
    Le mot de Dyck correspondant sera alors
    $\underbrace{1 \cdots 1}_{l_1}0
     \underbrace{1 \cdots 1}_{l_2}0 \cdots
     \underbrace{1 \cdots 1}_{l_n}0$.
    
    \item $\mathcal{D}_n \to \mathcal{PF'}_n$ :
    Soit $w \in \mathcal{D}_n$ un mot de Dyck. Considérons sa
    représentation sous la forme d'un chemin de Dyck.
    Notons $s_i$ l'abscisse du $i^{e}$ pas Nord.
    On pose alors $a_i = s_i + 1$.\\
    La fonction de parking primitive correspondante sera ainsi
    $(a_1, \ldots, a_n)$.
\end{itemize}
\end{proof}

Nous proposons maintenant des relations de couverture pour ces deux
ensembles, telles que les posets ainsi créés soient isomorphes, et que 
l'un puisse être obtenu en appliquant la bijection ci-dessus à l'autre.

\begin{definition}[$\gtrdot_d$]
    Soient $w$ et $w'$ deux mots de Dyck de longueur $2n$.
    On dit que $w$ couvre $w'$, noté $w \gtrdot_d w'$, s'il existe deux
    mots $w_1$ et $w_2$ tels que :
    \begin{itemize}
        \item $w = w_101w_2$
        \item $w' = w_110w_2$
    \end{itemize}  
\end{definition}

\begin{rem}
    Si $w_1 \gtrdot_d w_2$, alors le chemin de Dyck correspondant à $w_2$
    est \emph{au dessus} de celui correspondant à $w_1$, et la
    \emph{différence} entre les deux chemins est un carré de côté 1.
\end{rem}

\begin{definition}[Chemins de Dyck Imbriqués]
    Deux chemins de Dyck $w_1$ et $w_2$ sont dits \emph{imbriqués}
    si $w_1$ est égal à $w_2$ ou au dessus de $w_2$. 
\end{definition}

On déduit donc la proposition suivante de la remarque précédente.

\begin{prop}
    S'il existe une séquence $w_1 \gtrdot_d w_2 \gtrdot_d
    w_3 \gtrdot_d \cdots \gtrdot_d w_k$ avec $k \geqslant 0$,
    alors $w_1$ et $w_k$ sont \emph{imbriqués}.
\end{prop}

Cette relation de couverture engendre notre \emph{poset} pour
$\mathcal{D}_n$.
Ainsi, le poset contient l'\emph{intervalle} $[w_1;w_2]$ si et seulement
si $w_1$ et $w_2$ sont imbriqués.

On définit maintenant la relation bijective sur les fonctions de parking.
Celle-ci sera la même pour les 4 types de fonctions de parking (classiques,
classiques primitives, rationnelles, et rationnelles primitives).

\begin{definition}[$\gtrdot$]
    Soient $f$ et $g$ deux fonctions de parking.
    On dit que $f$ couvre $g$, noté $f \gtrdot g$, s'il existe $i$ tel que :
    \begin{itemize}
        \item $f = (a_1, \ldots, a_{i-1}, a_i,\ \ \ \ 
            a_{i+1}, \ldots, a_n)$
        \item $g = (a_1, \ldots, a_{i-1}, a_i - 1, a_{i+1},
        \ldots, a_n)$
    \end{itemize}
\end{definition}

Cette relation de couverture engendre notre poset pour $\mathcal{PF'}_n$.

\begin{theorem}[Théorème principal]
    Le nombre d'intervalles dans ces posets est égal à
    $$\frac {6 (2n)! (2n+2)!}{n!(n+1)!(n+2)!(n+3)!}$$.
\end{theorem}

\begin{proof}
    Puisque le nombre d'intervalles dans le poset de $\mathcal{D}_n$
    peut être vu comme le nombre de paires $(w_1, w_k)$ telles que
    $w_1 \gtrdot_d w_2 \gtrdot_d \cdots \gtrdot_d w_k$, alors nous pouvons
    décrire le nombre d'intervalles comme étant le nombre de 
    \emph{paires de chemins de Dyck imbriqués}.\\
    On introduit alors une notion de chemin de Dyck \emph{étoilé},
    c'est-à-dire un chemin de Dyck dont certains pas sont annotés d'une
    étoile *. Ces chemins sont une représentation des paires de chemins de
    Dyck imbriqués : en omettant les étoiles, on obtient le chemin du dessous.
    Ensuite, pour déduire le chemin du dessus :
    \begin{itemize}
        \item Un pas non étoilé est conservé
        \item Un pas Nord étoilé est remplacé par un pas Est
        \item Un pas Est étoilé est remplacé par un pas Nord.
    \end{itemize}
    On créé maintenant une bijection entre les chemins de Dyck étoilés
    de longueur $2n$ correspondant à des paires de chemins de Dyck de
    longueur $2n$, et les chemins allant du point (0,0) au point (0,0)
    composées de $2n$ pas Nord, Est, Sud, Ouest qui restent dans le
    premier octant.
    Pour cela, on effectue la transformation suivante :
    \begin{itemize}
        \item Nord non-étoilé $\longleftrightarrow$ Nord
        \item Nord étoilé $\longleftrightarrow$ Ouest
        \item Est non-étoilé $\longleftrightarrow$ Sud
        \item Est étoilé $\longleftrightarrow$ Est
    \end{itemize}
    Ainsi, puisque $\mathcal{D}_n \longleftrightarrow \{$Chemins de Dyck
    étoilés de longueur $2n\} \longleftrightarrow \{$ Chemins NESO de
    longueur $2n\}$, on sait que le nombre de paires de chemins de Dyck
    imbriqués de longueur $2n$ est égal au nombre de chemins NESO de
    longueur $2n$.

    Or, par définition, ce nombre est égal au $n+1^{e}$ terme de la suite
    de l'OEIS \href{https://oeis.org/A005700}{A005700}
    \footnote{https://oeis.org/A005700} (Ce qui rejoint le commentaire de 
    Bruce Westbury).\\
    Alec Mihailovs a démontré que ce nombre est bien égal à 
    $\displaystyle\frac {6 (2n)! (2n+2)!}{n!(n+1)!(n+2)!(n+3)!}$.
\end{proof}

Les premiers termes de cette suite sont $1, 1, 3, 14, 84,
594, 4719, 40898, 379236, 3711916, ...$\\

\begin{expl}[Les posets de $\mathcal{D}_4$ et $\mathcal{PF'}_4$]
    ~\\
    \begin{center}
        \input{fig/fig7.tex}
        Ces posets contiennent chacun $\frac {1}{5} \binom{8}{4} =
        14$ éléments et $84$ relations.
    \end{center}
\end{expl}

On souhaite maintenant étendre cette construction au cas non-primitif.
Bien que celle sur les fonctions de parking soit la même, il
reste à expliciter la bijection, et à définir la relation de couverture
sur les mots de Dyck \emph{décorés}.

\subsection{Le cas général}

Les exemples 14 à 17 donnés en annexe B illustrent les propositions,
définitions et théorèmes de cette section.

Puisque $\mathcal{PF}_n$ et $\mathcal{LD}_n$ ont le même cardinal,
nous pouvons créer une bijection entre les deux ensembles.

\begin{prop}
    Il existe une \emph{bijection explicite} entre les fonctions de parking
    classiques de longueur $n$ et les mots de Dyck décorés de
    longueur $2n$.
\end{prop}

\begin{proof}
    Nous créons à nouveau deux bijections inverse l'une de l'autre.
    \begin{itemize}
        \item $\mathcal{PF}_n \to \mathcal{LD}_n$ :
        Soit $f = (a_1, \ldots, a_n) \in \mathcal{PF}_n$ une fonction de
        parking. Pour tout $i \in \{1, \ldots, n\}$, posons $im_i$ :
        $\{j\ |\ a_j = i\}$. \\
        Notons alors $im_{i,1}, \ldots, im_{i,k_i}$ les éléments de $im_i$
        triés par ordre croissant.\\
        Le mot de Dyck décoré correspondant sera 
        $\underbrace{im_{1,1} \cdots im_{1,k_1}}_{im_1}0
         \underbrace{im_{2,1} \cdots im_{2,k_2}}_{im_2}0
         \cdots
         \underbrace{im_{n,1} \cdots im_{n,k_n}}_{im_n}0$.

        \item $\mathcal{LD}_n \to \mathcal{PF}_n$ :
        Soit $w$ un mot de Dyck décoré. Considérons sa représentation sous
        la forme d'un chemin de Dyck. Notons $s_i$ l'abscisse du $i^{e}$ pas
        Nord.\\
        On note alors $label(i)$ le label du $i^{e}$ pas nord, et
        $dist_i = \{label(j) | s_j = i\}$ l'ensemble  des labels des pas
        Nord à distance $i$ de l'axe des ordonnées.\\
        Ainsi, si $j \in dist_i$, on pose $a_j = i + 1$.\\
        La fonction de parking correspondante sera donc $(a_1, \ldots, a_n)$.
    \end{itemize}
\end{proof}

La relation suivante est l'extension de $\gtrdot_d$ au cas décoré.

\begin{definition}[$\gtrdot_{ld}$]
    Soient $w$ et $w'$ deux mots de Dyck décorés. On dit que $w$ couvre
    $w'$, noté $w \gtrdot_{ld} w'$, s'il existe $l$, $r$, $x$, $x'$, $y$,
    $z$, et $z'$ tels que :
    \begin{itemize}
        \item $l$ est le mot vide, ou finit par un $0$
        \item $r$ est le mot vide, ou commence par un $0$
        \item $x = x_1x_2 \cdots$ avec $x_i > 0$ pour tout $i$
        \item $z = z_1z_2 \cdots$ avec $z_i > 0$ pour tout $i$
        \item $x' = x$ où $y$ est correctement inséré en ordre croissant
        \item $y$ apparait dans $z$, et $z' = z$ où $y$ à été supprimé
        \item $w = lx0zr$
        \item $w' = lx'0z'r$
    \end{itemize}
\end{definition}

Pour expliquer l'idée derrière cette relation de couverture, nous aurons
besoin de la définition suivante.

\begin{definition}[Montée]
    Une \emph{montée} d'un mot de Dyck (décoré ou non) est une sous-mot
    maximal ne contenant pas de 0, et suivi d'un 0.
\end{definition}

\begin{rem}
    Si $w_1 \gtrdot_{ld} w_2$, alors le chemin de Dyck décoré correspondant
    à $w_2$ est \emph{au dessus} de celui correspondant à $w_1$,
    et la \emph{différence} entre les deux est un carré de côté 1.\\
    De plus, la relation $\gtrdot_{ld}$ peut être vue ainsi :
    $w_1$ couvre $w_2$ si et seulement si l'on peut obtenir $w_2$ à partir
    de $w_1$ en enlevant un élément de la $i + 1^{e}$ montée, 
    et en réinsérant cet élément en ordre croissant dans la $i^{e}$
    montée.
\end{rem}

Cette relation de couverture engendre notre poset pour $\mathcal{LD}_n$.
En gardant la relation $\gtrdot$ sur l'ensemble des fonctions de parking
classiques, on obtient ainsi les posets bijectifs espérés.


\begin{conj}[Conjecture Principale]
    Le nombre d'intervalles des posets définis ici pour $\mathcal{LD}_n$
    et $\mathcal{PF}_n$ est le $n+1^{e}$ terme de la suite de l'OEIS
    \href{https://oeis.org/A196304}{A196304}
    \footnote{https://oeis.org/A196304}.
\end{conj}

Les premiers termes de cette suite sont $1, 1, 5, 64, 1587,
65421, 4071178, 357962760, 4237910716, ...$.

\begin{expl}[Les posets de $\mathcal{LD}_3$ et $\mathcal{PF}_3$]
    ~\\
    \begin{center}
        \begin{center}
\begin{tikzpicture}[scale = 0.18]
    \draw [ultra thick, color = cyan] (0,0) -- (0,1)
        -- (0,2) -- (0,3) -- (0,4) -- (0,5) -- (1,5)
        -- (2,5) -- (3,5);

    \draw [ultra thick, color = cyan] (0,9) -- (0,10)
        -- (0,11) -- (0,12) -- (0,13) -- (1,13) -- (1,14)
        -- (2,14) -- (3,14);
        
    \draw [ultra thick, color = cyan] (-4,18) -- (-4,19)
        -- (-4,20) -- (-4,21) -- (-3,21) -- (-3,22)
        -- (-3,23) -- (-2,23) -- (-1,23);

    \draw [ultra thick, color = cyan] (4,18) -- (4,19)
        -- (4,20) -- (4,21) -- (4,22) -- (5,22) -- (6,22)
        -- (6,23) -- (7,23);

    \draw [ultra thick, color = cyan] (-4,27) -- (-4,28)
        -- (-4,29) -- (-3,29) -- (-3,30) -- (-3,31)
        -- (-3,32) -- (-2,32) -- (-1,32);

    \draw [ultra thick, color = cyan] (4,27) -- (4,28)
        -- (4,29) -- (4,30) -- (5,30) -- (5,31) -- (6,31)
        -- (6,32) -- (7,32);

    \draw [ultra thick, color = cyan] (0,36) -- (0,37)
        -- (0,38) -- (1,38) -- (1,39) -- (1,40) -- (2,40)
        -- (2,41) -- (3,41);

    \draw [->][out=-90,in=90, ultra thick] 
        [color=magenta](1.2,35.5) to (-2.5,32.5);
    \draw [->][color=magenta, ultra thick]
        (-2.5,26.5) to (-2.5,23.5);
    \draw [->][color=magenta, ultra thick]
        (-2.5,17.5) to (-1,14.5);        

    \draw [->][out=-90,in=90, ultra thick] 
        [color=green](1.5,8.5) to (1.5,5.5);

    \draw [->][out=-90,in=90, ultra thick]
        [color=violet](1.8,35.5) to (5.5,32.5);
    \draw [->][color=violet, ultra thick]
        (5.5,26.5) to (5.5,23.5);
    \draw [->][color=violet, ultra thick]
        (5.5,26.5) to (-1.7,23.7);
    \draw [->][color=violet, ultra thick]
        (5.5,17.5) to (4,14.5);

\end{tikzpicture}
\end{center}
        Il y a $4^2 = 16$ éléments et $64$ intervalles dans chacun de ces
        posets.
    \end{center}
\end{expl}

Bien que, à notre connaissance, il n'existe pas de structure combinatoire
associée à cette suite, les tests effectués via Sagemath sur
$n = 1, 2, \cdots, 8$ suggèrent que le nombre d'intervalles de nos posets
puissent en être une.\\

Pour aller plus loin, la prochaine partie aborde une généralisation des
fonctions de parking : les fonctions de parking \emph{rationnelles}.

Cette extension peut être vue ainsi :
Soit $(a_1, \ldots, a_n)$ une séquence d'entiers positifs,
et $(b_1, \ldots, b_n)$ son tri par ordre croissant.
Dans le cas classique, les limites pour $(b_1, \ldots, b_n)$
étaient $(1, \ldots, n)$, et dépendaient donc d'un unique entier $n$.
Dans le cas rationnel, les limites dépendront de \emph{deux entier
premiers entre eux} $a$ et $b$.
Plus précisément, ces limites seront $(1, 1 + \frac{b}{a},
1 + \frac{2b}{a}, 1 + \frac{3b}{a}, \ldots)$, avec $a = n$. 

\newpage
\section{Un poset pour les fonctions de parking rationnelles}
\subsection{Fonctions de parking rationnelles}

Cette partie est illustrée par les exemples 17 à 19 de l'annexe C.

\begin{definition}[a, b - Fonction de Parking]
    Une \emph{a, b - fonction de parking} est une séquence d'entiers
    positifs $(a_1, a_2, \ldots, a_n)$ telle que :\\
    \begin{itemize*}
        \item $n = a$\\
        \item son tri croissant $(b_1, b_2, \ldots, b_n)$ respecte la
        condition suivante :$b_i \leqslant \frac{b}{a}(i-1) + 1$ 
        pour tout $i \leqslant n$.
    \end{itemize*}
\end{definition}

On note $\mathcal{PF}_{a,b}$ l'ensemble des a, b - fonctions de parking. 

\begin{theorem}[Armstrong, Loehr et Warrington, 2014]
    Soit $pf_{a,b}$ le cardinal de $\mathcal{PF}_{a,b}$.
    Nous avons $$pf_{a,b} = b^{a-1}$$
\end{theorem}

\begin{rem}
Le cas classique peut être vu comme le cas $a = n, b = n + 1$.
Autrement dit, $\mathcal{PF}_{n, n + 1} = \mathcal{PF}_n$.
\end{rem}

Similairement au cas classique, on définit une fonction de parking
\emph{rationnelle primitive} comme une fonction de parking primitive
triée en ordre croissant.
On note $\mathcal{PF'}_{a,b}$ l'ensemble des a, b - fonctions de parking
primitives.

\begin{theorem}
    Soit $pf'_{a,b}$ le cardinal de $\mathcal{PF'}_{a,b}$.
    Nous avons $$\displaystyle pf'_{a,b} = 
    \frac{1}{a + b} \binom{a + b}{b}$$
\end{theorem}

Ce nombre est appelé le \emph{nombre de Catalan rationnel}, et on le note
$Cat(a,b)$.
Là aussi, le cas classique correspond à $a = n, b = n + 1$.
Autrement dit, $\mathcal{PF'}_{n, n + 1} = \mathcal{PF'}_n$.

\subsection{Chemins de Dyck rationnels}

Cette partie est illustrée par les exemples 20 à XXX de l'annexe C.

\begin{definition}[a, b - mot de Dyck]
    Un \emph{a, b - mot de Dyck} est un mot $w \in \{0,1\}^*$ tel que :
    \begin{itemize}
        \item pour tout \emph{suffixe} $w'$ de $w$,
            $\displaystyle |w'|_1 \geqslant \frac{a}{b}|w'|_0$.
        \item $|w|_0 = b$.
        \item $|w|_1 = a$.
    \end{itemize}
\end{definition}

Un a, b - mot de Dyck peut être représenté par un chemin allant du point
$(0,0)$ au point $(b,a)$, et restant au dessus de l'axe $y = \frac{a}{b}x$,
appelé \emph{a, b - chemin de Dyck} :
\begin{itemize}
    \item Chaque $1$ correspond à un \emph{pas Nord} $\uparrow$. 
    \item Chaque $0$ correspond à un \emph{pas Est} $\rightarrow$.
\end{itemize}

On note $\mathcal{R}_{a, b}$ l'ensemble des a, b - mots de Dyck.

\begin{theorem}[Bizley, 1954]
    Soit $r_{a,b}$ le cardinal de $\mathcal{R}_{a,b}$.
    Nous avons $$r_{a,b} = \frac{1}{a+b} \binom {a+b}{a} =
    \frac{(a+b-1)!}{a!b!}$$
\end{theorem}

\subsection{Posets rationnels}

\newpage
\section{Arbres de parking rationnels}
Nous commençons par rappeler ce qu'est un arbre de parking.

\begin{definition}[Arbre de Parking]
    Un \emph{arbre de parking} est défini à partir d'une fonction de
    parking $f = (a_1, \ldots, a_n) \in \mathcal{PF}_n$ ainsi :
    \begin{itemize}
        \item Pour tout $1 \leqslant i \leqslant n+1$, on définit
            $s_i$ comme $\{j\ |\ a_j = i\}$
        \item $[s_1, \ldots, s_{n+1}]$ décrit le parcours préfixe de
            l'arbre.
        \item Chaque noeud étiquetté par un ensemble de taille $k$
            est d'arité $k$. 
    \end{itemize}
\end{definition}

\begin{rem}
    Les feuilles de l'arbre correspondent aux éléments $i$ tels que
    $1 \leqslant i \leqslant n + 1$, où $i$ n'est \emph{pas} dans $f$.\\
    De plus, comme l'arbre possèdera -- par définition -- $n$ branches,
    la présence d'un noeud correspondant à $n + 1$ est nécessaire, bien
    que son étiquette sera toujours l'ensemble vide.
\end{rem}

\begin{expl}[$n = 12$]
    ~\\
    \begin{itemize*}
        \item $f = (5, 7, 1, 3, 1, 8, 2, 7, 1, 3, 11, 11)$\\
        \item Labels : $[\{3, 5, 9\},\ \{7\},\ \{4, 10\},\ 
            \emptyset,\ \{1\},\ \emptyset,\ \{2,8\},\ 
            \{6\},\ \emptyset,\ \emptyset,\ \{11, 12\},\ 
            \emptyset, \emptyset]$
    \end{itemize*}
    \begin{center}
    \begin{tikzpicture}[scale=0.75]
        \node (1)  at (0,16)   {$\{3, 5, 9\}$};
        \node (2)  at (-6,12) {$\{7\}$};
        \node (3)  at (-6,8)  {$\{4, 10\}$};
        \node (5)  at (-5,4)   {$\{1\}$};
        \node (7)  at (0,12)   {$\{2, 8\}$};
        \node (8)  at (-2,8)   {$\{6\}$};
        \node (11) at (6,12)  {$\{11, 12\}$};

        \node (a) at (-1.5,16)    {$1$};
        \node (b) at (-7.5,12)    {$2$};
        \node (c) at (-7.5,8)     {$3$};
        \node (d) at (-7.5,6.15)  {$4$};
        \node (e) at (-6.5,4)     {$5$};
        \node (f) at (-6,2.15)    {$6$};
        \node (g) at (-1.5,12)    {$7$};
        \node (h) at (-3.5,8)     {$8$};
        \node (i) at (-3,6.15)    {$9$};
        \node (j) at (1.75,10.2)  {$10$};
        \node (k) at (7.75,12)    {$11$};
        \node (l) at (4.25,10.2)  {$12$};
        \node (m) at (7.75,10.2)  {$13$};

        \draw[ultra thick][color=brown!70!orange]
            (0,16)  circle (1);
        \draw[ultra thick][color=brown!70!orange]
            (-6,12) circle (1);
        \draw[ultra thick][color=brown!70!orange]
            (-6,8)  circle (1);
        \draw[ultra thick][color=brown!70!orange]
            (-5,4)  circle (1);
        \draw[ultra thick][color=brown!70!orange]
            (0,12)  circle (1);
        \draw[ultra thick][color=brown!70!orange]
            (-2,8)  circle (1);
        \draw[ultra thick][color=brown!70!orange]
            (6,12)  circle (1);

        \draw [->][ultra thick][color=brown!70!orange]
            (-0.5,15.15) to (-6,13.2);
        \draw [->][ultra thick][color=brown!70!orange]
            (0,15) to (0,13.2);
        \draw [->][ultra thick][color=brown!70!orange]
            (0.5,15.15) to (6,13.2);
        \draw [->][ultra thick][color=brown!70!orange]
            (-6,11) to (-6,9.2);
        \draw [->][ultra thick][color=brown!70!orange]
            (-0.25,11) to (-2,9.2);
        \draw [->][ultra thick][color=brown!70!orange]
            (-5.75,7) to (-5,5.2);

        \draw [-*][ultra thick][color=green!60!gray]
            (-6.25,7) to (-6.75,6);
        \draw [-*][ultra thick][color=green!60!gray]
            (-5,3) to (-5, 2);
        \draw [-*][ultra thick][color=green!60!gray]
            (-2,7) to (-2,6);
        \draw [-*][ultra thick][color=green!60!gray]
            (0.25,11) to (0.75,10);
        \draw [-*][ultra thick][color=green!60!gray]
            (5.75,11) to (5.25,10);
        \draw [-*][ultra thick][color=green!60!gray]
            (6.25,11) to (6.75,10);
    \end{tikzpicture}
\end{center}    
\end{expl}

Inversement, en lisant les étiquettes d'un arbre de parking en ordre
préfixe, on obtient la liste des positions de chaque nombre dans la fonction
de parking correspondante, ce qui créé ainsi une \emph{bijection}.

\begin{expl}[De l'arbre à la fonction]
    ~
\begin{center}
    \begin{tikzpicture}[scale=0.75]
        \node (1)  at (0,16)   {$\{5\}$};
        \node (2)  at (0,12)   {$\{2\}$};
        \node (3)  at (0,8)   {$\{1,3,4,7\}$};
        \node (6)  at (1,4)    {$\{8\}$};
        \node (7)  at (1,0)   {$\{6\}$};

        \node (a) at (-1.5,16)    {$1$};
        \node (b) at (-1.5,12)    {$2$};
        \node (c) at (-1.5,8)     {$3$};
        \node (d) at (-1.5,5.5)   {$4$};
        \node (e) at (-0.5,5.5)   {$5$};
        \node (f) at (-0.5,4)     {$6$};
        \node (g) at (-0.5,0)     {$7$};
        \node (h) at (0,-1.5)     {$8$};
        \node (i) at (1.5,5.5)    {$9$};

        \draw[ultra thick][color=brown!70!orange]
            (0,16)  circle (1);
        \draw[ultra thick][color=brown!70!orange]
            (0,12) circle (1);
        \draw[ultra thick][color=brown!70!orange]
            (0,8)  circle (1);
        \draw[ultra thick][color=brown!70!orange]
            (1,4)  circle (1);
        \draw[ultra thick][color=brown!70!orange]
            (1,0)  circle (1);

        \draw [->][ultra thick][color=brown!70!orange]
            (0,15) to (0,13.2);
        \draw [->][ultra thick][color=brown!70!orange]
            (0,11) to (0,9.2);
        \draw [->][ultra thick][color=brown!70!orange]
            (0.2,7) to (1,5.2);
        \draw [->][ultra thick][color=brown!70!orange]
            (1,3) to (1,1.2);

        \draw [-*][ultra thick][color=green!60!gray]
            (-0.5,7.15) to (-1.5,6);
        \draw [-*][ultra thick][color=green!60!gray]
            (-0.25,7.05) to (-0.5, 6);
        \draw [-*][ultra thick][color=green!60!gray]
            (0.5,7.15) to (1.5,6);
        \draw [-*][ultra thick][color=green!60!gray]
            (1,-1) to (1,-2);
    \end{tikzpicture}
\end{center}  
    \begin{itemize}
        \item Les étiquettes sont $[\{5\},\ \{2\},\ \{1,3,4,7\},\ 
        \emptyset,\ \emptyset,\ \{8\},\ \{6\},\ \emptyset,\ 
        \emptyset]$.
        \item La fonction correspondante est donc
            $(3,2,3,3,1,7,3,6) \in \mathcal{PF}_8$.
    \end{itemize}
\end{expl}

On cherche maintenant à étendre cette construction au cas rationnel.


\begin{definition}[Arbre de Parking Rationnel]
    Un \emph{arbre de parking rationnel} est défini à partir d'une fonction
    de parking rationnelle $f = (a_1, \ldots, a_a) \in \mathcal{PF}_{a,b}$
    ainsi :
    \begin{itemize}
        \item Pour tout $1 \leqslant i \leqslant n+1$, on définit la limite
            $l_i$ comme étant la \emph{partie entière} de
            $\displaystyle \frac{b}{a}(i-1) + 1$.
            \subitem Posons $l_0 = 0$.
        \item De ces limites, nous déduisons les intervalles
            $itv_i =\ ]l_{i-1}, l_i]$ for $1 \leqslant i
            \leqslant a+1$.
        \item Pour tout $1 \leqslant i \leqslant b + 1$, posons
        $s_i = \{j\ |\ a_j = i\}$.
        \item $[s_1, \ldots, s_{b+1}]$ décrit alors le parcours préfixe
            de notre arbre.
        \item Chaque noeud étiquetté par un ensemble de taille $k$
            possède $k$ \emph{groupes} d'enfants, qui sont définis par
            les intervalles.
    \end{itemize}
\end{definition}

\begin{expl}[$a < b$]
    ~
    \begin{itemize}
        \item $a = 7$
        \item $b = 9$
        \item Limites : $[1,\ 2 \frac{2}{7},\ 
            3 \frac{4}{7},\ 4 \frac{6}{7},\  
            6 \frac{1}{7},\ 7 \frac{3}{7},\ 
            8 \frac{5}{7},\ 10]$
        \item Limites entières : $[0,1,2,3,4,6,7,8,10]$
        \item Intervalles :
            \subitem $]0, 1]$ \hspace{5mm} $]1, 2]$
            \hspace{5mm} $]2, 3]$ \hspace{5mm} $]3, 4]$
            \hspace{5mm} $]4, 6]$ \hspace{5mm} $]6, 7]$
            \hspace{5mm} $]7, 8]$ \hspace{5mm} $]8, 10]$
        \item Groupes d'enfants :
            \subitem $[1]$ \hspace{5mm} $[2]$ \hspace{5mm}
            $[3]$ \hspace{5mm} $[4]$ \hspace{5mm}
            $[5,6]$ \hspace{5mm} $[7]$ \hspace{5mm} $[8]$
        \item $f = (6,2,6,1,4,7,2)$
        \item Etiquettes : $\{\{4\},\ \{2,7\},\ \emptyset,\ 
            \{5\},\ \emptyset,\ \{1,3\},\ \{6\},\ 
            \emptyset,\ \emptyset,\ \emptyset\}$\\
    \end{itemize}
        \begin{center}
        \begin{tikzpicture}[scale=0.75]
            \node (1)  at (0,17)   {$\{4\}$};
            \node (2)  at (0,13)   {$\{2,7\}$};
            \node (4)  at (3,9)   {$\{5\}$};
            \node (6)  at (6,5)    {$\{1,3\}$};
            \node (7)  at (3,1)    {$\{6\}$};

            \node (a) at (-1.5,17)    {$1$};
            \node (b) at (-1.5,13)    {$2$};
            \node (c) at (-2,11)      {$3$};
            \node (d) at (1.5,9)      {$4$};
            \node (e) at (1,7)        {$5$};
            \node (f) at (4.5,5)      {$6$};
            \node (g) at (1.5,1)      {$7$};
            \node (h) at (2.5,-1)     {$8$};
            \node (i) at (5.5,3)      {$9$};
            \node (j) at (8.5,3)      {$10$};

            \node[right] (ac) at (1.5,17) {$1$ groupe d'enfants : $[2]$};
            \node[right] (bc) at (1.5,13) {$2$ groupes d'enfants : $[3]$
             et $[4]$};
            \node[right] (dc) at (4.5,9) {$1$ groupe d'enfants : $[5,6]$};
            \node[right] (fc) at (7.5,5) {$2$ groupes d'enfants : $[7]$
             et $[9,10]$};
            \node[right] (gc) at (4.5,1) {$1$ groupe d'enfants : $[8]$};

            \draw[ultra thick][color=brown!70!orange]
                (0,17)  circle (1);
            \draw[ultra thick][color=brown!70!orange]
                (0,13) circle (1);
            \draw[ultra thick][color=brown!70!orange]
                (3,9)  circle (1);
            \draw[ultra thick][color=brown!70!orange]
                (6,5)  circle (1);
            \draw[ultra thick][color=brown!70!orange]
                (3,1)  circle (1);

            \draw [->][ultra thick][color=brown!70!orange]
                (0,16) to (0,14.2);
            \draw [->][ultra thick][color=brown!70!orange]
                (0.6,12.25) to (2.7,10.2);
            \draw [->][ultra thick][color=brown!70!orange]
                (3.6,8.25) to (5.7,6.2);
            \draw [->][ultra thick][color=brown!70!orange]
                (5.4,4.25) to (3.3,2.2);

            \draw [-*][ultra thick][color=green!60!gray]
                (-0.6,12.2) to (-1.6,11);
            \draw [-*][ultra thick][color=green!60!gray]
                (2.4,8.2) to (1.4,7);
            \draw [-*][ultra thick][color=green!60!gray]
                (3,0) to (3,-1);
            \draw [-*][ultra thick][color=green!60!gray]
                (6,4) to (6,3);
            \draw [-*][ultra thick][color=green!60!gray]
                (6.6,4.2) to (8,3);
        \end{tikzpicture}
    \end{center}
\end{expl}

\begin{expl}[$a > b$]
    ~
    \begin{itemize}
        \item $a = 9$
        \item $b = 7$
        \item Limites : $[1,\ 1 \frac{7}{9},\ 
            2 \frac{5}{9},\ 3 \frac{3}{9},\  
            4 \frac{1}{9},\ 4 \frac{8}{9},\ 
            5 \frac{6}{9},\ 6 \frac{4}{9},\ 
            7 \frac{2}{9},\ 8]$
        \item Limites entières : $[0,1,1,2,3,4,4,5,6,7,8]$
        \item Intervalles :
            \subitem $]0, 1]$ \hspace{5mm} $]1, 1]$
            \hspace{5mm} $]1, 2]$ \hspace{5mm} $]2, 3]$
            \hspace{5mm} $]3, 4]$ 
            \subitem $]4, 4]$ \hspace{5mm} $[4, 5]$
            \hspace{5mm} $]5, 6]$ \hspace{5mm} $]6, 7]$
            \hspace{5mm} $]7, 8]$
        \item Groupes d'enfants :
            \subitem $[1]$ \hspace{5mm} $\emptyset$ 
            \hspace{5mm} $[2]$ \hspace{5mm} $[3]$
            \hspace{5mm} $[4]$ \hspace{5mm} $\emptyset$
            \hspace{5mm} $[5]$ \hspace{5mm} $[6]$
            \hspace{5mm} $[7]$ \hspace{5mm} $[8]$
        \item $f = (4,2,2,1,4,5,7,4,1)$
        \item Etiquettes : $\{\{4,9\},\ \{2,3\},\ \emptyset,\ 
            \{1,5,8\}, \{6\},\ \emptyset,\ \{7\},\ 
            \emptyset\}$\\
    \end{itemize}
    \begin{center}
    \begin{tikzpicture}[scale=0.8]
        \node (1)  at (0,17)   {$\{4,9\}$};
        \node (2)  at (0,13)   {$\{2,3\}$};
        \node (4)  at (3,9)   {$\{1,5,8\}$};
        \node (5)  at (0,5)    {$\{6\}$};
        \node (7)  at (6,5)    {$\{7\}$};

        \node (a) at (-1.5,17)    {$1$};
        \node (b) at (-1.5,13)    {$2$};
        \node (c) at (-2,11)      {$3$};
        \node (d) at (1.5,9)      {$4$};
        \node (e) at (1.5,5)      {$5$};
        \node (f) at (-1,3)       {$6$};
        \node (g) at (4.5,5)      {$7$};
        \node (h) at (5,3)        {$8$};

        \node[right] (ac) at (1.5,17) {$2$ groupes d'enfants : $\emptyset$
         et $[2]$};
        \node[right] (bc) at (1.5,13) {$2$ groupes d'enfants : $[3]$
         et $[4]$};
        \node[right] (dc) at (4.5,9) {$3$ groupes d'enfants : $\emptyset$,
         $[5]$ et $[7]$};
        \node[left] (fc) at (-1.5,5) {$1$ groupe d'enfants : $[6]$};
        \node[right] (gc) at (7.5,5) {$1$ groupe d'enfants : $[8]$};

        \draw[ultra thick][color=brown!70!orange]
            (0,17)  circle (1);
        \draw[ultra thick][color=brown!70!orange]
            (0,13) circle (1);
        \draw[ultra thick][color=brown!70!orange]
            (3,9)  circle (1);
        \draw[ultra thick][color=brown!70!orange]
            (0,5)  circle (1);
        \draw[ultra thick][color=brown!70!orange]
            (6,5)  circle (1);

        \draw [->][ultra thick][color=brown!70!orange]
            (0,16) to (0,14.2);
        \draw [->][ultra thick][color=brown!70!orange]
            (0.6,12.25) to (2.7,10.2);
        \draw [->][ultra thick][color=brown!70!orange]
            (2.4,8.25) to (0.3,6.2);
        \draw [->][ultra thick][color=brown!70!orange]
            (3.6,8.25) to (5.7,6.2);

        \draw [-*][ultra thick][color=green!60!gray]
            (-0.6,12.2) to (-1.6,11);
        \draw [-*][ultra thick][color=green!60!gray]
            (0,4) to (0,3);
        \draw [-*][ultra thick][color=green!60!gray]
            (6,4) to (6,3);
    \end{tikzpicture}
\end{center}
\end{expl}

Dans les deux cas, la direction inverse de la \emph{bijection} est obtenue
-- comme pour le cas classique -- par un parcours préfixe.

\newpage
\section{Conclusion}
Nous avons ainsi répondu aux deux problématiques abordées :\\

Premièrement, nous avons défini des relations de couvertures créant des
posets pour les quatre types de fonctions de parking, ainsi que pour les
quatre types de chemins de Dyck que nous avons mis en bijections avec ces
dernières.
Dans chaque cas, le poset de l'un peut être obtenu en appliquant la
bijection présentée au poset de l'autre.
De plus, dans le cas rationnel, nous traitons aussi bien la configuration
$a > b$ que $a < b$.
Dans plusieurs travaux sur les fonctions de parking rationnelles
et les structures en bijection avec celles-ci, seul le cas $a < b$ avait
été traité.

La particularité de notre approche réside autant dans cette généralisation
que dans l'utilisation des chemins de Dyck.
En effet, la plupart des travaux précédents se basaient sur les
\emph{partitions non-croisées} (\cite{ref4, ref5, ref8, ref9}).
Ici, nous avons choisi de ne pas utiliser cette structure, car les posets
obtenus en appliquant la bijection Partitions Non Croisées $\to$ Fonctions
de Parking ne laissaient pas paraître de relation de couverture évidente
pour les fonctions de parking.

De plus, dans le cas rationnel, la définition de partitions non-croisées
rationnelles est complexe.
A notre connaissance, la seule définition donnée pour tous les entiers $a$
et $b$ premiers entre eux (et non seulement lorsque $a < b$) est celle donnée
par Bodnar (\cite{ref8}), qui nécessite de nombreuses étapes et définitions
intermédiaires.\\

Ensuite, nous avons étendu la notion d'arbres de parking au cas rationnel.\\

Une question qui émerge naturellement de ce travail est le besoin d'une
preuve pour la Conjecture Principale.
De plus, s'il existe une preuve entièrement combinatoire, nous pourrions
ainsi obtenir une formule et une interprétation combinatoire à la suite
entière A196304, sans nécessiter de passer par les séries génératrices.

Quant au cas rationnel, de futurs travaux pourront inclure la recherche de
formules généralisées pour le nombre d'intervalles des posets, dans le cas
primitif comme dans le cas général.

Enfin, en suivant le cheminement de \cite{ref9} sur les arbres de parking,
il pourra être intéressant d'étudier les relations de couverture sur les
arbres de parking rationnels.

\bibliographystyle{unsrt}
\bibliography{ref}

\appendix

\newpage
\section{Exemples pour l'Introduction}
\begin{example}
On appelle \emph{permutation circulaire} une permutation dont la
représentation sous forme de composition de cycles contient un unique cycle
de taille $n$. Il est ainsi possible de créer une bijection explicite entre
$\mathfrak{S}_n$ et l'ensemble des permutations circulaires de
$\mathfrak{S}_{n+1}.$
On présente ici deux bijections inverse l'une de l'autre.
\begin{itemize}
    \item $\mathfrak{S}_n \to\ ^{n+1}\mathfrak{S}_{n+1}$ :
    Soit $\sigma = a_1 \ldots a_n$ notre permutation.
    $\sigma' = (n+1 \ a_1 \ldots a_n) \in ^{n+1}\mathfrak{S}_{n+1}.$
    \item $^{n+1}\mathfrak{S}_{n+1} \to\ \mathfrak{S}_n$ :
    Soit $\sigma' = (a_1 \ldots a_{n+1})$ notre permutation circulaire.
    Notons $i$ l'indice tel que $a_i = n + 1$.
    $\sigma = a_{i+1} \ldots a_{n+1} a_1 \ldots a_{i-1} \in \mathfrak{S}_n.$
\end{itemize}
L'existence d'une telle bijection implique que les deux ensembles sont de
même cardinal.
En effet, le cardinal de $\mathfrak{S}_n$ est $n!$. Quant à celui de
$^{n+1}\mathfrak{S}_{n+1}$, on définit tout d'abord un \emph{k-cycle}.
$\sigma$ est un k-cycle si son écriture sous la forme d'une composition de
cycles est un unique cycle de taille $k$. Ainsi, $^{n+1}\mathfrak{S}_{n+1}$
est l'ensemble des (n + 1)-cycles de $\mathfrak{S}_{n+1}$.
Or, dans le groupe symétrique $\mathfrak{S}_{n}$, le nombre de k-cycles est
$\displaystyle \frac{n!}{(n-k)!k}$.
Ainsi, le cardinal de $^{n+1}\mathfrak{S}_{n+1}$ est
$\displaystyle \frac{(n+1)!}{(n+1-n-1)!(n+1)} = \frac{(n+1)!}{n+1} = n!$.
On a bien l'égalité.\\
\end{example}

\begin{example}[Définition 1 : $n = 7$]
    ~
    \begin{itemize}
        \item $f_1 = (7, 3, 1, 4, 2, 5, 2) \in \mathcal{PF}_7$
        \item $f_2 = (7, 3, 1, 4, 2, 5, 4) \notin \mathcal{PF}_7$
    \end{itemize}
\end{example}

\begin{example}[Théorème 2 : $n = 3 : pf_3 = 16$]
    ~\\
    \begin{itemize*}             
        \item $(1, 1, 1)$
        \item $(1, 1, 2)$
        \item $(1, 1, 3)$
        \item $(1, 2, 1)$
        \item $(1, 2, 2)$
        \item $(1, 2, 3)$
        \item $(1, 3, 1)$
        \item $(1, 3, 2)$
        \item $(2, 1, 1)$
        \item $(2, 1, 2)$
        \item $(2, 1, 3)$
        \item $(2, 2, 1)$
        \item $(2, 3, 1)$
        \item $(3, 1, 1)$
        \item $(3, 1, 2)$
        \item $(3, 2, 1)$
    \end{itemize*}
\end{example}

\begin{example}[Définition 3 : $n = 4$]
    ~
    \begin{itemize}
        \item $f_1 = (1, 2, 2, 3) \in \mathcal{PF'}_4$
        \item $f_2 = (1, 2, 3, 2) \notin \mathcal{PF'}_4
         \text{, bien que } f_2 \in \mathcal{PF}_4$
    \end{itemize}
\end{example}

\begin{example}[Théorème 4 : $n = 3 : pf'_3 = 5$]
    ~\\
    \begin{itemize*}
        \item $(1, 1, 1)$
        \item $(1, 1, 2)$
        \item $(1, 1, 3)$
        \item $(1, 2, 2)$
        \item $(1, 2, 3)$
    \end{itemize*}
\end{example}

\begin{example}[Définition 5 : $n = 5$]
    ~
    \begin{itemize}
        \item $w_1 = 1011000110 \text{ n'est \emph{pas} un mot de Dyck,
        car } |1011000|_0 > |1011000|_1.$
        \item $w_2 = 1011010101 \text{ n'est \emph{pas} un mot de Dyck,
        car } |w_2|_0 \neq |w_2|_1.$
        \item $w_3 = 1011010100 \text{ \emph{est} un mot de Dyck : }$
    \end{itemize}
    \input{fig/fig1.tex}
\end{example}

\begin{example}[Théorème 6 : $n = 3$]
    $d_3 = 5$.
    \begin{center}
        \begin{center}
    \begin{tikzpicture}[scale=0.6]
        \node (a) at (0, 0) {};
        \node (b) at (0, 6) {};
        \node (c) at (6, 0) {};
        \node (d) at (5.5, 5.5) {};
        \node (e) at (4.5, 6) [color = magenta]
            {$x = y$}; 
        \draw [dashed, very thick, ->] (a) to (b);
        \draw [dashed, very thick, ->] (a) to (c);
        \draw [dashed, very thick, ->]
            [color = magenta] (a) to (d);

        \node (1)  at (0,0)   {};
        \node (2)  at (0,1)   {};
        \node (3)  at (1,1)   {};
        \node (4)  at (1,2)   {};
        \node (5)  at (1,3)   {};
        \node (6)  at (2,3)   {};
        \node (7)  at (3,3)   {};
        \node (8)  at (3,4)   {};
        \node (9)  at (4,4)   {};
        \node (10) at (4,5)   {};
        \node (11) at (5,5)   {};
        \draw [->, ultra thick, color = cyan]
            (1)  to (2);
        \draw [->, ultra thick, color = cyan] 
            (2)  to (3);
        \draw [->, ultra thick, color = cyan]
            (3)  to (4);
        \draw [->, ultra thick, color = cyan]
            (4)  to (5);
        \draw [->, ultra thick, color = cyan]
            (5)  to (6);
        \draw [->, ultra thick, color = cyan]
            (6)  to (7);
        \draw [->, ultra thick, color = cyan]
            (7)  to (8);
        \draw [->, ultra thick, color = cyan]
            (8)  to (9);
        \draw [->, ultra thick, color = cyan]
            (9)  to (10);
        \draw [->, ultra thick, color = cyan]
            (10) to (11);

        \node at (-0.2, -0.2) {$0$};
        \node at (-0.3, 1)    {$1$};
        \node at (1, -0.3)    {$1$};
        \node at (-0.3, 2)    {$2$};
        \node at (2, -0.3)    {$2$};
        \node at (-0.3, 3)    {$3$};
        \node at (3, -0.3)    {$3$};
        \node at (-0.3, 4)    {$4$};
        \node at (4, -0.3)    {$4$};
        \node at (-0.3, 5)    {$5$};
        \node at (5, -0.3)    {$5$};

        \node [color = cyan] at (-1, 0.5) {\textbf{4}};
        \node [color = cyan] at (-1, 1.5) {\textbf{1}};
        \node [color = cyan] at (-1, 2.5) {\textbf{5}};
        \node [color = cyan] at (-1, 3.5) {\textbf{2}};
        \node [color = cyan] at (-1, 4.5) {\textbf{3}};

    \end{tikzpicture}
\end{center}
    \end{center}
\end{example}

\begin{example}[Définition 7 : $n = 5$]
    ~
    \begin{itemize}
        \item $w_1 = 4051002030 \text{ n'est \emph{pas} un mot de Dyck
            décoré, car } 5 > 1.$
        \item $w_2 = 4015002030 \text{ \emph{est} un mot de Dyck décoré :}$
    \end{itemize}
    \begin{tikzpicture}[scale = 0.13]
    \draw [ultra thick, color = cyan] (0,0) -- (0,1)
        -- (0,2) -- (0,3) -- (0,4) -- (1,4) -- (2,4)
        -- (3,4) -- (4,4);

    \draw [ultra thick, color = cyan] (0,8) -- (0,9)
        -- (0,10) -- (0,11) -- (1,11) -- (1,12) -- (2,12)
        -- (3,12) -- (4,12);
        
    \draw [ultra thick, color = cyan] (-6,16) -- (-6,17)
        -- (-6,18) -- (-5,18) -- (-5,19) -- (-5,20)
        -- (-4,20) -- (-3,20) -- (-2,20);

    \draw [ultra thick, color = cyan] (6,16) -- (6,17)
        -- (6,18) -- (6,19) -- (7,19) -- (8,19) -- (8,20)
        -- (9,20) -- (10,20);

    \draw [ultra thick, color = cyan] (-8,24) -- (-8,25)
        -- (-7,25) -- (-7,26) -- (-7,27) -- (-7,28)
        -- (-6,28) -- (-5,28) -- (-4,28);

    \draw [ultra thick, color = cyan] (0,24) -- (0,25)
        -- (0,26) -- (1,26) -- (1,27) -- (2,27) -- (2,28)
        -- (3,28) -- (4,28);

    \draw [ultra thick, color = cyan] (8,24) -- (8,25)
        -- (8,26) -- (8,27) -- (9,27) -- (10,27) -- (11,27)
        -- (11,28) -- (12,28);

    \draw [ultra thick, color = cyan] (-8,32) -- (-8,33)
        -- (-7,33) -- (-7,34) -- (-7,35) -- (-6,35)
        -- (-6,36) -- (-5,36) -- (-4,36);

    \draw [ultra thick, color = cyan] (0,32) -- (0,33)
        -- (0,34) -- (1,34) -- (2,34) -- (2,35) -- (2,36)
        -- (3,36) -- (4,36);

    \draw [ultra thick, color = cyan] (8,32) -- (8,33)
        -- (8,34) -- (9,34) -- (9,35) -- (10,35) -- (11,35)
        -- (11,36) -- (12,36);

    \draw [ultra thick, color = cyan] (-8,40) -- (-8,41)
        -- (-7,41) -- (-7,42) -- (-6,42) -- (-6,43)
        -- (-6,44) -- (-5,44) -- (-4,44);

    \draw [ultra thick, color = cyan] (0,40) -- (0,41)
        -- (1,41) -- (1,42) -- (1,43) -- (2,43) -- (3,43)
        -- (3,44) -- (4,44);

    \draw [ultra thick, color = cyan] (8,40) -- (8,41)
        -- (8,42) -- (9,42) -- (10,42) -- (10,43) -- (11,43)
        -- (11,44) -- (12,44);

    \draw [ultra thick, color = cyan] (0,48) -- (0,49)
        -- (1,49) -- (1,50) -- (2,50) -- (2,51) -- (3,51)
        -- (3,52) -- (4,52);


    \draw [->][color=magenta, ultra thick](1,47.5) to (-5.5,45);
    \draw [->][color=magenta, ultra thick](-6,39.5) to (-6,36.5);
    \draw [->][color=magenta, ultra thick](-6,39.5) to (1,36.5); 
    \draw [->][color=magenta, ultra thick](-6,31.5) to (-6,28.5);
    \draw [->][color=magenta, ultra thick](-6,31.5) to (1,28.5);
    \draw [->][color=magenta, ultra thick](-6,23.5) to (-4,20.5);
    \draw [->][color=magenta, ultra thick](-4,15.5) to (1,13);


    \draw [->][color=green, ultra thick](2,47.5) to (2,44.5);
    \draw [->][color=green, ultra thick](2,39.5) to (-3.5,36.5);
    \draw [->][color=green, ultra thick](2,39.5) to (7.5,36.5);
    \draw [->][color=green, ultra thick](2,31.5) to (2,28.5);
    \draw [->][color=green, ultra thick](2,23.5) to (-1.5,20.5);
    \draw [->][color=green, ultra thick](2,23.5) to (5.5,20.5);
    \draw [->][color=green, ultra thick](2,7.5) to (2,5);

    \draw [->][color=violet, ultra thick](3,47.5) to (9.5,45);
    \draw [->][color=violet, ultra thick](10,39.5) to (3,37);
    \draw [->][color=violet, ultra thick](10,39.5) to (10,36.5);
    \draw [->][color=violet, ultra thick](10,31.5) to (3,29);
    \draw [->][color=violet, ultra thick](10,31.5) to (10,28.5);
    \draw [->][color=violet, ultra thick](10,23.5) to (8,20.5);
    \draw [->][color=violet, ultra thick](8,15.5) to (3,13);

\end{tikzpicture}
\end{example}

\begin{example}[Théorème 8 : $n = 3$]
    $ld_3 = 4^2 = 16$
    \begin{itemize}
        \item Mots de la forme $XXX000$ :
            \subitem $123000$
        \item Mots de la forme $XX0X00$ :
            \subitem $120300$
            \hspace{2cm} $130200$
            \hspace{2cm} $230100$
        \item Mots de la forme $XX00X0$ :
            \subitem $120030$
            \hspace{2cm} $130020$
            \hspace{2cm} $230010$
        \item Mots de la forme $X0XX00$ :
            \subitem $102300$
            \hspace{2cm} $201300$
            \hspace{2cm} $301200$
        \item Mots de la forme $X0X0X0$ :
            \subitem $102030$
            \hspace{2cm} $103020$
            \hspace{2cm} $201030$
            \subitem $203010$
            \hspace{2cm} $301020$
            \hspace{2cm} $302010$
    \end{itemize}
    
\end{example}

\newpage
\section{Exemples pour la Partie 2}
\begin{example}[Proposition 9 : $n = 6, \mathcal{PF'}_n \to \mathcal{D}_n$]
    ~\
    \begin{itemize}
        \item $f = (1, 1, 2, 4, 5, 5)$
            \subitem $l_1 = 2$
            \hspace{2cm} $l_2 = 1$
            \hspace{2cm} $l_3 = 0$
            \subitem $l_4 = 1$
            \hspace{2cm} $l_5 = 2$
            \hspace{2cm} $l_6 = 0$
        \item $w = (110100101100)$
    \end{itemize}
    \begin{tikzpicture}[scale = 0.18]
    \node at (0,0)   {$(1,1,1,1)$};
    \node at (0,6)   {$(1,1,1,2)$};                
    \node at (-6,12) {$(1,1,2,2)$};
    \node at (6,12)  {$(1,1,1,3)$};
    \node at (-10,18) {$(1,2,2,2)$};
    \node at (0,18)  {$(1,1,2,3)$};
    \node at (10,18)  {$(1,1,1,4)$};
    \node at (-10,24) {$(1,2,2,3)$};
    \node at (0,24)  {$(1,1,3,3)$};
    \node at (10,24)  {$(1,1,2,4)$};
    \node at (-10,30) {$(1,2,3,3)$};
    \node at (0,30)  {$(1,2,2,4)$};
    \node at (10,30)  {$(1,1,3,4)$};
    \node at (0,36)  {$(1,2,3,4)$};


    \draw [->][color=magenta, ultra thick] (-1,35) to (-9,32);
    \draw [->][color=magenta, ultra thick](-10,29) to (-10,25);
    \draw [->][color=magenta, ultra thick](-10,29) to (-1,26); 
    \draw [->][color=magenta, ultra thick](-10,23) to (-10,19);
    \draw [->][color=magenta, ultra thick](-10,23) to (-1,20);
    \draw [->][color=magenta, ultra thick](-10,17) to (-6,13);
    \draw [->][color=magenta, ultra thick](-6,11) to (-1,8);


    \draw [->][color=green, ultra thick](0,35) to (0,31);
    \draw [->][color=green, ultra thick](0,29) to (-7,26);
    \draw [->][color=green, ultra thick](0,29) to (7,26);
    \draw [->][color=green, ultra thick](0,23) to (0,19);
    \draw [->][color=green, ultra thick](0,17) to (-4,14);
    \draw [->][color=green, ultra thick](0,17) to (4,14);
    \draw [->][color=green, ultra thick](0,5) to (0,1);

    \draw [->][color=violet, ultra thick](1,35) to (10,32);
    \draw [->][color=violet, ultra thick](10,29) to (1,26);
    \draw [->][color=violet, ultra thick](10,29) to (10,25);
    \draw [->][color=violet, ultra thick](10,23) to (1,20);
    \draw [->][color=violet, ultra thick](10,23) to (10,19);
    \draw [->][color=violet, ultra thick](10,17) to (8,13);
    \draw [->][color=violet, ultra thick](6,11) to (1,8);

\end{tikzpicture}
\end{example}

\begin{example}[Proposition 9 : $n = 6, \mathcal{D}_n \to \mathcal{PF'}_n$]
    ~\
    \begin{itemize}
        \item $w = 101011010010$
    \end{itemize}
        \begin{center}
        \begin{tikzpicture}[scale=0.7]
            \node (a) at (0, 0) {};
            \node (b) at (0, 7) {};
            \node (c) at (7, 0) {};
            \node (d) at (6.5, 6.5) {};
            \node (e) at (7, 6) [color = magenta]
                {$x = y$}; 
            \draw [dashed, very thick, ->] (a) to (b);
            \draw [dashed, very thick, ->] (a) to (c);
            \draw [dashed, very thick, ->]
                [color = magenta] (a) to (d);

            \node (1)  at (0,0)   {};
            \node (2)  at (0,1)   {};
            \node (3)  at (1,1)   {};
            \node (4)  at (1,2)   {};
            \node (5)  at (2,2)   {};
            \node (6)  at (2,3)   {};
            \node (7)  at (2,4)   {};
            \node (8)  at (3,4)   {};
            \node (9)  at (3,5)   {};
            \node (10) at (4,5)   {};
            \node (11) at (5,5)   {};
            \node (12) at (5,6)   {};
            \node (13) at (6,6)   {};
            \draw [->, ultra thick, color = cyan]
                (1)  to (2);
            \draw [->, ultra thick, color = cyan] 
                (2)  to (3);
            \draw [->, ultra thick, color = cyan]
                (3)  to (4);
            \draw [->, ultra thick, color = cyan]
                (4)  to (5);
            \draw [->, ultra thick, color = cyan]
                (5)  to (6);
            \draw [->, ultra thick, color = cyan]
                (6)  to (7);
            \draw [->, ultra thick, color = cyan]
                (7)  to (8);
            \draw [->, ultra thick, color = cyan]
                (8)  to (9);
            \draw [->, ultra thick, color = cyan]
                (9)  to (10);
            \draw [->, ultra thick, color = cyan]
                (10) to (11);
            \draw [->, ultra thick, color = cyan]
                (11) to (12);
            \draw [->, ultra thick, color = cyan]
                (12) to (13);

            \node at (-0.2, -0.2) {$0$};
            \node at (-0.3, 1)    {$1$};
            \node at (1, -0.3)    {$1$};
            \node at (-0.3, 2)    {$2$};
            \node at (2, -0.3)    {$2$};
            \node at (-0.3, 3)    {$3$};
            \node at (3, -0.3)    {$3$};
            \node at (-0.3, 4)    {$4$};
            \node at (4, -0.3)    {$4$};
            \node at (-0.3, 5)    {$5$};
            \node at (5, -0.3)    {$5$};
            \node at (-0.3, 6)    {$6$};
            \node at (6, -0.3)    {$6$};

        \end{tikzpicture}
    \end{center}
    \begin{itemize}
        \item Distances : 
            \subitem $s_1 = 0$
                \hspace{2cm} $a_1 = 1$
            \subitem $s_2 = 1$
                \hspace{2cm} $a_2 = 2$
            \subitem $s_3 = 2$
                \hspace{2cm} $a_3 = 3$
            \subitem $s_4 = 2$
                \hspace{2cm} $a_4 = 3$
            \subitem $s_5 = 3$
                \hspace{2cm} $a_5 = 4$
            \subitem $s_6 = 5$
                \hspace{2cm} $a_6 = 6$
        \item $f = (1, 2, 3, 3, 4, 6)$
    \end{itemize}
    
\end{example}

\begin{example}[Définition 10 : $n = 7$]
    $10110011001100 \gtrdot_d 10110101001100$
    \begin{itemize}
        \item $w_1 = 10110$
        \item $w_2 = 1001100$
    \end{itemize}
    \input{fig/fig6.tex}
\end{example}

\begin{example}[Définition 13 : $n = 6$]
    $(1, 1, 2, 3, 4, 5) \gtrdot (1, 1, 2, 3, 3, 5)$    
\end{example}

\begin{example}[Proposition 15 : $n = 6, \mathcal{PF}_n \to \mathcal{LD}_n$]
    ~\
    \begin{itemize}
        \item $f = (5, 2, 1, 4, 5, 1)$
            \subitem $im_1 = \{3, 6\}$
            \hspace{16mm} $im_2 = \{2\}$
            \hspace{24mm} $im_3 = \emptyset$
            \subitem $im_4 = \{4\}$
            \hspace{2cm} $im_5 = \{1, 5\}$
            \hspace{2cm} $im_6 = \emptyset$
        \item $w = 360200401500$
    \end{itemize}
    \input{fig/fig8.tex}
\end{example}

\begin{example}[Proposition 15 : $n = 6, \mathcal{LD}_n \to \mathcal{PF}_n$]
    ~\
    \begin{itemize}
        \item $w = 402560010030$
    \end{itemize}
    \input{fig/fig9.tex}
    \begin{itemize}
        \item Distances :
            \subitem $s_1 = 0$
            \hspace{2cm} $s_2 = 1$
            \hspace{2cm} $s_3 = 1$
            \subitem $s_4 = 1$
            \hspace{2cm} $s_5 = 3$
            \hspace{2cm} $s_6 = 5$
        \item Labels :
            \subitem $dist_0 = \{4\}$
            \hspace{2cm} $dist_1 = \{2, 5, 6\}$
            \hspace{2cm} $dist_2 = \emptyset$
            \subitem $dist_3 = \{1\}$
            \hspace{2cm} $dist_4 = \emptyset$
            \hspace{32mm} $dist_5 = \{3\}$
        \item $f = (4, 2, 6, 1, 2, 2)$
    \end{itemize}
\end{example}

\begin{example}[Définition 16 : $n = 5$]
    $104503600200 \gtrdot_{ld} 10345060200$
    ~\\
    \begin{itemize*}
        \item $l = 10$
        \item $r = 0200$
        \item $x = 45$
        \item $x' = 345$
        \item $y = 3$
        \item $z = 36$
        \item $z' = 6$
    \end{itemize*}
    \begin{tikzpicture}[scale=1]
    \node at (3,0) {\large $\mathbf{\sigma (P)}$};
    \node [label = above : {$1$}] (1)
        at (4,5) {};
    \node [label = right : {$2$}] (2)
        at (5,3) {};
    \node [label = below : {$3$}] (3)
        at (4,1) {};
    \node [label = below : {$4$}] (4)
        at (2,1) {};
    \node [label = left : {$5$}]  (5)
        at (1,3) {};
    \node [label = above : {$6$}] (6)
        at (2,5) {};
    \draw [dashed][very thick]
    (1) -- (2) -- (3) -- (4)
        -- (5) -- (6) -- (1);
    \fill [color = orange] (4,5) -- (4,1)
        -- (2,5) -- cycle;
    \draw [color = violet][line width = 4pt] 
        (5,3) -- (2,1);
    \fill [color=green] (1,3) circle (0.2);
    
  \end{tikzpicture}
\end{example}

\begin{example}[Définition 17 : $n = 5$]
    Dans l'ordre, les montées de $104503600200$ sont :\\
    \begin{itemize*}
        \item $1$
        \item $45$
        \item $36$
        \item $\emptyset$
        \item $2$
        \item $\emptyset$
    \end{itemize*}
\end{example}

\newpage
\section{Exemples pour la Partie 3}
\begin{example}[Définition 10 : $a > b$ : $a = 7$, $b = 3$]
    ~
    \begin{itemize}
        \item Limites pour toute séquence de $\mathcal{PF}_{7,3}$
            une fois triée : $[1,\ 1 \frac{3}{7},\ 1 \frac{6}{7},\ 
            2 \frac{2}{7},\ 2 \frac{5}{7},\ 3 \frac{1}{7},\ 
            3 \frac{4}{7}]$
        \item $f_1 = (2, 1, 1, 3, 2, 3, 1) \in
            \mathcal{PF}_{7,3}$
        \item $f_2 = (2, 1, 2, 3, 2, 3, 1) \notin
            \mathcal{PF}_{7,3}$, bien que $f_2 \in
            \mathcal{PF}_7$
    \end{itemize}
\end{example}

\begin{example}[Définition 10 : $a < b$ : $a = 5$, $b = 7$]
    ~
    \begin{itemize}
        \item Limites pour toute séquence de $\mathcal{PF}_{5,7}$
            une fois triée : $[1,\ 2 \frac{2}{5},\ 3 \frac{4}{5},\ 
            5 \frac{1}{5},\ 6 \frac{3}{5}]$
        \item $f_3 = (6, 3, 5, 1, 2) \in
            \mathcal{PF}_{5,7}$, bien que $f_3 \notin
            \mathcal{PF}_5$
        \item $f_4 = (6, 3, 5, 1, 3) \notin
            \mathcal{PF}_{5,7}$\\
    \end{itemize}
\end{example}

\begin{example}[Théorème 6 : $a = 3, b = 5$]
    ~\\
    \begin{itemize*}\\
        \item $pf_{a,b} = 25$
        \item Limites : $[1,\ 2 \frac{2}{3},\ 
            4 \frac{1}{3}]$\\\\
        \subitem $(1, 1, 1)$
        \subitem $(1, 1, 2)$
        \subitem $(1, 1, 3)$
        \subitem $(1, 1, 4)$
        \subitem $(1, 2, 1)$
        \subitem $(1, 2, 2)$
        \subitem $(1, 2, 3)$
        \subitem $(1, 2, 4)$
        \subitem $(1, 3, 1)$
        \subitem $(1, 3, 2)$
        \subitem $(1, 4, 1)$
        \subitem $(1, 4, 2)$
        \subitem $(2, 1, 1)$
        \subitem $(2, 1, 2)$
        \subitem $(2, 1, 3)$
        \subitem $(2, 1, 4)$
        \subitem $(2, 2, 1)$
        \subitem $(2, 3, 1)$
        \subitem $(2, 4, 1)$
        \subitem $(3, 1, 1)$
        \subitem $(3, 1, 2)$
        \subitem $(3, 2, 1)$
        \subitem $(4, 1, 1)$
        \subitem $(4, 1, 2)$
        \subitem $(4, 2, 1)$\\
    \end{itemize*}
\end{example}

\begin{example}[Définition 11 : $a < b : a = 3, b = 5$]
    ~
    \begin{itemize}
        \item $w_1 = 10100010 \text{ n'est \emph{pas} un 3, 5 - mot de
        Dyck, car } |101000|_1 = 2 < \frac{3}{5}|101000|_0 = 2 \frac{2}{5}.$
        \item $w_2 = 10100100 \text{ \emph{est} un 3, 5 - mot de Dyck : }$
    \end{itemize}
    \begin{center}
    \begin{tikzpicture}[scale=0.7]
        \node (a) at (0, 0) {};
        \node (b) at (0, 4) {};
        \node (c) at (6, 0) {};
        \node (d) at (5.5, 3.3) {};
        \node (e) at (6, 3) [color = magenta]
            {$y = \frac{3}{5}x$}; 
        \draw [dashed, very thick, ->] (a) to (b);
        \draw [dashed, very thick, ->] (a) to (c);
        \draw [dashed, very thick, ->]
            [color = magenta] (a) to (d);

        \node (1)  at (0,0)   {};
        \node (2)  at (0,1)   {};
        \node (3)  at (1,1)   {};
        \node (4)  at (1,2)   {};
        \node (5)  at (2,2)   {};
        \node (6)  at (3,2)   {};
        \node (7)  at (3,3)   {};
        \node (8)  at (4,3)   {};
        \node (9)  at (5,3)   {};
        \draw [->, ultra thick, color = cyan]
            (1)  to (2);
        \draw [->, ultra thick, color = cyan] 
            (2)  to (3);
        \draw [->, ultra thick, color = cyan]
            (3)  to (4);
        \draw [->, ultra thick, color = cyan]
            (4)  to (5);
        \draw [->, ultra thick, color = cyan]
            (5)  to (6);
        \draw [->, ultra thick, color = cyan]
            (6)  to (7);
        \draw [->, ultra thick, color = cyan]
            (7)  to (8);
        \draw [->, ultra thick, color = cyan]
            (8)  to (9);

        \node at (-0.2, -0.2) {$0$};
        \node at (-0.3, 1)    {$1$};
        \node at (1, -0.3)    {$1$};
        \node at (-0.3, 2)    {$2$};
        \node at (2, -0.3)    {$2$};
        \node at (-0.3, 3)    {$3$};
        \node at (3, -0.3)    {$3$};
        \node at (4, -0.3)    {$4$};
        \node at (5, -0.3)    {$5$};

    \end{tikzpicture}
\end{center}
\end{example}

\begin{example}[Définition 11 : $a > b : a = 7, b = 3$]
    ~
    \begin{itemize}
        \item $w_1 = 1110011110 \text{ n'est \emph{pas} un 7, 3 - mot de
        Dyck, car } |11100|_1 = 3 < \frac{7}{3}|11100|_0 = 4 \frac{1}{3}.$
        \item $w_2 = 1110111010 \text{ \emph{est} un 7, 3 - mot de Dyck : }$
    \end{itemize}
    \input{fig/fig12.tex}
\end{example}

\begin{example}[Théorème 8 : $a = 7, b = 2$]
    $r_n = 4$.
    \begin{center}
        \begin{tikzpicture}[scale = 0.7]
    \node (a) at (0, 0) {};
    \node (b) at (0, 8) {};
    \node (c) at (3, 0) {};
    \node (d) at (2.2, 7.7) {};
    \node (e) at (3, 7) [color = magenta]
        {$y = \frac{7}{2}x$}; 
    \draw [dashed, very thick, ->] (a) to (b);
    \draw [dashed, very thick, ->] (a) to (c);
    \draw [dashed, very thick, ->]
        [color = magenta] (a) to (d);

    \node (1)  at (0,0)   {};
    \node (2)  at (0,1)   {};
    \node (3)  at (0,2)   {};
    \node (4)  at (0,3)   {};
    \node (5)  at (0,4)   {};
    \node (6)  at (1,4)   {};
    \node (7)  at (1,5)   {};
    \node (8)  at (1,6)   {};
    \node (9)  at (1,7)   {};
    \node (10) at (2,7)   {};
    \draw [->, ultra thick, color = cyan]
        (1)  to (2);
    \draw [->, ultra thick, color = cyan] 
        (2)  to (3);
    \draw [->, ultra thick, color = cyan]
        (3)  to (4);
    \draw [->, ultra thick, color = cyan]
        (4)  to (5);
    \draw [->, ultra thick, color = cyan]
        (5)  to (6);
    \draw [->, ultra thick, color = cyan]
        (6)  to (7);
    \draw [->, ultra thick, color = cyan]
        (7)  to (8);
    \draw [->, ultra thick, color = cyan]
        (8)  to (9);
    \draw [->, ultra thick, color = cyan]
        (9)  to (10);

    \node at (-0.2, -0.2) {$0$};
    \node at (-0.3, 1)    {$1$};
    \node at (1, -0.3)    {$1$};
    \node at (-0.3, 2)    {$2$};
    \node at (2, -0.3)    {$2$};
    \node at (-0.3, 3)    {$3$};
    \node at (-0.3, 4)    {$4$};
    \node at (-0.3, 5)    {$5$};
    \node at (-0.3, 6)    {$6$};
    \node at (-0.3, 7)    {$7$};
    \node at (1, -1)      {$111101110$};

\end{tikzpicture}
\begin{tikzpicture}[scale = 0.7]
    \node (a) at (0, 0) {};
    \node (b) at (0, 8) {};
    \node (c) at (3, 0) {};
    \node (d) at (2.2, 7.7) {};
    \node (e) at (3, 7) [color = magenta]
        {$y = \frac{7}{2}x$}; 
    \draw [dashed, very thick, ->] (a) to (b);
    \draw [dashed, very thick, ->] (a) to (c);
    \draw [dashed, very thick, ->]
        [color = magenta] (a) to (d);

    \node (1)  at (0,0)   {};
    \node (2)  at (0,1)   {};
    \node (3)  at (0,2)   {};
    \node (4)  at (0,3)   {};
    \node (5)  at (0,4)   {};
    \node (6)  at (0,5)   {};
    \node (7)  at (1,5)   {};
    \node (8)  at (1,6)   {};
    \node (9)  at (1,7)   {};
    \node (10) at (2,7)   {};
    \draw [->, ultra thick, color = cyan]
        (1)  to (2);
    \draw [->, ultra thick, color = cyan] 
        (2)  to (3);
    \draw [->, ultra thick, color = cyan]
        (3)  to (4);
    \draw [->, ultra thick, color = cyan]
        (4)  to (5);
    \draw [->, ultra thick, color = cyan]
        (5)  to (6);
    \draw [->, ultra thick, color = cyan]
        (6)  to (7);
    \draw [->, ultra thick, color = cyan]
        (7)  to (8);
    \draw [->, ultra thick, color = cyan]
        (8)  to (9);
    \draw [->, ultra thick, color = cyan]
        (9)  to (10);

    \node at (-0.2, -0.2) {$0$};
    \node at (-0.3, 1)    {$1$};
    \node at (1, -0.3)    {$1$};
    \node at (-0.3, 2)    {$2$};
    \node at (2, -0.3)    {$2$};
    \node at (-0.3, 3)    {$3$};
    \node at (-0.3, 4)    {$4$};
    \node at (-0.3, 5)    {$5$};
    \node at (-0.3, 6)    {$6$};
    \node at (-0.3, 7)    {$7$};
    \node at (1, -1)      {$111110110$};

\end{tikzpicture}
\begin{tikzpicture}[scale = 0.7]
    \node (a) at (0, 0) {};
    \node (b) at (0, 8) {};
    \node (c) at (3, 0) {};
    \node (d) at (2.2, 7.7) {};
    \node (e) at (3, 7) [color = magenta]
        {$y = \frac{7}{2}x$}; 
    \draw [dashed, very thick, ->] (a) to (b);
    \draw [dashed, very thick, ->] (a) to (c);
    \draw [dashed, very thick, ->]
        [color = magenta] (a) to (d);

    \node (1)  at (0,0)   {};
    \node (2)  at (0,1)   {};
    \node (3)  at (0,2)   {};
    \node (4)  at (0,3)   {};
    \node (5)  at (0,4)   {};
    \node (6)  at (0,5)   {};
    \node (7)  at (0,6)   {};
    \node (8)  at (1,6)   {};
    \node (9)  at (1,7)   {};
    \node (10) at (2,7)   {};
    \draw [->, ultra thick, color = cyan]
        (1)  to (2);
    \draw [->, ultra thick, color = cyan] 
        (2)  to (3);
    \draw [->, ultra thick, color = cyan]
        (3)  to (4);
    \draw [->, ultra thick, color = cyan]
        (4)  to (5);
    \draw [->, ultra thick, color = cyan]
        (5)  to (6);
    \draw [->, ultra thick, color = cyan]
        (6)  to (7);
    \draw [->, ultra thick, color = cyan]
        (7)  to (8);
    \draw [->, ultra thick, color = cyan]
        (8)  to (9);
    \draw [->, ultra thick, color = cyan]
        (9)  to (10);

    \node at (-0.2, -0.2) {$0$};
    \node at (-0.3, 1)    {$1$};
    \node at (1, -0.3)    {$1$};
    \node at (-0.3, 2)    {$2$};
    \node at (2, -0.3)    {$2$};
    \node at (-0.3, 3)    {$3$};
    \node at (-0.3, 4)    {$4$};
    \node at (-0.3, 5)    {$5$};
    \node at (-0.3, 6)    {$6$};
    \node at (-0.3, 7)    {$7$};
    \node at (1, -1)      {$111111010$};

\end{tikzpicture}
\begin{tikzpicture}[scale = 0.7]
    \node (a) at (0, 0) {};
    \node (b) at (0, 8) {};
    \node (c) at (3, 0) {};
    \node (d) at (2.2, 7.7) {};
    \node (e) at (3, 7) [color = magenta]
        {$y = \frac{7}{2}x$}; 
    \draw [dashed, very thick, ->] (a) to (b);
    \draw [dashed, very thick, ->] (a) to (c);
    \draw [dashed, very thick, ->]
        [color = magenta] (a) to (d);

    \node (1)  at (0,0)   {};
    \node (2)  at (0,1)   {};
    \node (3)  at (0,2)   {};
    \node (4)  at (0,3)   {};
    \node (5)  at (0,4)   {};
    \node (6)  at (0,5)   {};
    \node (7)  at (0,6)   {};
    \node (8)  at (0,7)   {};
    \node (9)  at (1,7)   {};
    \node (10) at (2,7)   {};
    \draw [->, ultra thick, color = cyan]
        (1)  to (2);
    \draw [->, ultra thick, color = cyan] 
        (2)  to (3);
    \draw [->, ultra thick, color = cyan]
        (3)  to (4);
    \draw [->, ultra thick, color = cyan]
        (4)  to (5);
    \draw [->, ultra thick, color = cyan]
        (5)  to (6);
    \draw [->, ultra thick, color = cyan]
        (6)  to (7);
    \draw [->, ultra thick, color = cyan]
        (7)  to (8);
    \draw [->, ultra thick, color = cyan]
        (8)  to (9);
    \draw [->, ultra thick, color = cyan]
        (9)  to (10);

    \node at (-0.2, -0.2) {$0$};
    \node at (-0.3, 1)    {$1$};
    \node at (1, -0.3)    {$1$};
    \node at (-0.3, 2)    {$2$};
    \node at (2, -0.3)    {$2$};
    \node at (-0.3, 3)    {$3$};
    \node at (-0.3, 4)    {$4$};
    \node at (-0.3, 5)    {$5$};
    \node at (-0.3, 6)    {$6$};
    \node at (-0.3, 7)    {$7$};
    \node at (1, -1)      {$111111100$};

\end{tikzpicture}
    \end{center}
\end{example}

\begin{example}[Définition 12 : $a < b : a = 3, b = 5$]
    $w = 20130000$ :\\
    \begin{center}
\begin{tikzpicture}[scale = 0.22]
    \node at (0,0) {$(1,1,1,1,1)$};

    \node at (-18,5) {$(1,1,1,1,2)$};
    \node at (-9,5)  {$(1,1,1,2,1)$};
    \node at (0,5)   {$(1,1,2,1,1)$};
    \node at (9,5)   {$(1,2,1,1,1)$};
    \node at (18,5)  {$(2,1,1,1,1)$};

    \node at (-23,10) {$(1,1,1,2,2)$};
    \node at (-18,13) {$(1,1,2,1,2)$};
    \node at (-13,10) {$(1,1,2,2,1)$};
    \node at (-8,13)  {$(1,2,1,1,2)$};
    \node at (-3,10)  {$(1,2,1,2,1)$};
    \node at (3,13)   {$(1,2,2,1,1)$};
    \node at (8,10)   {$(2,1,1,1,2)$};
    \node at (13,13)  {$(2,1,1,2,1)$};
    \node at (18,10)  {$(2,1,2,1,1)$};
    \node at (23,13)  {$(2,2,1,1,1)$};

    \draw [->][color=magenta, ultra thick]
        (-23,9) to (-24,8);
    \draw [->][color=magenta, ultra thick]
        (-18,12) to (-19,11);
    \draw [->][color=magenta, ultra thick]
        (-8,12) to (-9,11);
    \draw [->][color=magenta, ultra thick]
        (8,9) to (7,8);
    \draw[color=magenta, ultra thick]
        (-18.5,7.2) -- (-17.5,6.2);
    \draw[color=magenta, ultra thick]
        (-18.5,6.2) -- (-17.5,7.2);
    \draw [->][out=-90,in=150, ultra thick] 
        [color=magenta](-18,4) to (-2,1);

    \draw [->][color=brown!7!orange, ultra thick]
        (-23,9) to (-22,8);
    \draw [->][color=brown!7!orange, ultra thick]
        (-13,9) to (-14,8);
    \draw [->][color=brown!7!orange, ultra thick]
        (-3,9) to (-4,8);
    \draw [->][color=brown!7!orange, ultra thick]
        (13,12) to (12,11);
    \draw[color=brown!7!orange, ultra thick]
        (-9.5,7.2) -- (-8.5,6.2);
    \draw[color=brown!7!orange, ultra thick]
        (-9.5,6.2) -- (-8.5,7.2);
    \draw [->][out=-90,in=90, ultra thick]
        [color=brown!7!orange](-9,4) to (-1,1);

    \draw [->][color=yellow, ultra thick]
        (-18,12) to (-17,11); 
    \draw [->][color=yellow, ultra thick]
        (-13,9) to (-12,8); 
    \draw [->][color=yellow, ultra thick]
        (3,12) to (2,11); 
    \draw [->][color=yellow, ultra thick]
        (18,9) to (17,8); 
    \draw[color=yellow, ultra thick]
        (-0.5,7.2) -- (0.5,6.2);
    \draw[color=yellow, ultra thick]
        (-0.5,6.2) -- (0.5,7.2);
    \draw [->][out=-90,in=90, ultra thick] 
        [color=yellow](0,4) to (0,1);

    \draw [->][color = green,  ultra thick]
        (-8,12) to (-7,11);
    \draw [->][color=green, ultra thick]
        (-3,9) to (-2,8);
    \draw [->][color=green, ultra thick]
        (3,12) to (4,11);
    \draw [->][color=green, ultra thick]
        (23,12) to (22,11);
    \draw[color=green, ultra thick]
        (8.5,7.2) -- (9.5,6.2);
    \draw[color=green, ultra thick]
        (8.5,6.2) -- (9.5,7.2);
    \draw [->][out=-90,in=90, ultra thick]
        [color=green](9,4) to (1,1);

    \draw [->][color=violet, ultra thick]
        (8,9) to (9,8);
    \draw [->][color=violet, ultra thick]
        (13,12) to (14,11);
    \draw [->][color=violet, ultra thick]
        (18,9) to (19,8);
    \draw [->][color=violet, ultra thick]
        (23,12) to (24,11);
    \draw[color=violet, ultra thick]
        (17.5,7.2) -- (18.5,6.2);
    \draw[color=violet, ultra thick]
        (17.5,6.2) -- (18.5,7.2);
    \draw [->][out=-90,in=30, ultra thick] 
        [color=violet](18,4) to (2,1);

\end{tikzpicture}
\end{center}
   \end{example}

\begin{example}[Définition 12 : $a > b : a = 7, b = 3$]
    $w_2 = 2456017030$ :\\
 \begin{tikzpicture}[scale=1]
    \node at (3,0) {\large $\mathbf{K(P)}$};
    \node [label = above : {$1$}] (1)
        at (4,5) {};
    \node [label = right : {$2$}] (2)
        at (5,3) {};
    \node [label = below : {$3$}] (3)
        at (4,1) {};
    \node [label = below : {$4$}] (4)
        at (2,1) {};
    \node [label = left : {$5$}]  (5)
        at (1,3) {};
    \node [label = above : {$6$}] (6)
        at (2,5) {};
    \draw [dashed][very thick]
    (1) -- (2) -- (3) -- (4)
        -- (5) -- (6) -- (1);
    \draw [color = orange][line width = 4pt] 
        (4,5) -- (1,3);            
    \draw [color = violet][line width = 4pt] 
        (4,1) -- (2,1);
    \fill [color=green] (5,3) circle (0.2); 
    \fill [color=green] (2,5) circle (0.2);   
  \end{tikzpicture}
\end{example}

\begin{example}[Théorème 9 : $a = 4, b = 3$]
    $lr_{a,b} = 3^3 = 27$
    \begin{itemize}
        \item Mots de la forme $XXXX000$ :
            \subitem $1234000$
        \item Mots de la forme $XXX0X00$ :
            \subitem $1230400$
            \hspace{2cm} $1240300$
            \hspace{2cm} $1340200$
            \subitem $2340100$
        \item Mots de la forme $XX0XX00$ :
            \subitem $1203400$
            \hspace{2cm} $1302400$
            \hspace{2cm} $1402300$
            \subitem $2301400$
            \hspace{2cm} $2401300$
            \hspace{2cm} $3401200$
        \item Mots de la forme $XXX00X0$ :
            \subitem $1230040$
            \hspace{2cm} $1240030$
            \hspace{2cm} $1340020$
            \subitem $2340010$
        \item Mots de la forme $XX0X0X0$ :
            \subitem $1203040$
            \hspace{2cm} $1204030$
            \hspace{2cm} $1302040$
            \subitem $1304020$
            \hspace{2cm} $1402030$
            \hspace{2cm} $1403020$
            \subitem $2301040$
            \hspace{2cm} $2304010$
            \hspace{2cm} $2401030$
            \subitem $2403010$
            \hspace{2cm} $3401020$
            \hspace{2cm} $3402010$
    \end{itemize}
    
\end{example}

\end{document}