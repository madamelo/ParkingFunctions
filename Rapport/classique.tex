\subsection{Le cas primitif}

Les exemples 9 à 13 donnés en annexe B illustrent les propositions,
définitions et théorèmes de cette section.

\begin{prop}
    Puisque $\mathcal{PF'}_n$ et $\mathcal{D}_n$ ont le même cardinal,
    nous pouvons créer une \emph{bijection} entre les fonctions de parking
    classiques primitives de longueur $n$ et les mots de Dyck de
    longueur $2n$.
\end{prop}

\begin{proof}
    ~\
\begin{itemize}
    \item $\mathcal{PF'}_n \to \mathcal{D}_n$ :
    Soit $f = (a_1, \ldots, a_n) \in \mathcal{PF'}_n$
    une fonction de parking classique primitive.
    Pour tout $i \in \{1, \ldots, n\}$, notons $l_i$ le nombre
    d'occurences de $i$ dans $f$.\\
    Le mot de Dyck correspondant sera alors
    $\underbrace{1 \cdots 1}_{l_1}0
     \underbrace{1 \cdots 1}_{l_2}0 \cdots
     \underbrace{1 \cdots 1}_{l_n}0$.
    
    \item $\mathcal{D}_n \to \mathcal{PF'}_n$ :
    Soit $w \in \mathcal{D}_n$ un mot de Dyck. Considérons sa
    représentation sous la forme d'un chemin de Dyck.
    Notons $s_i$ l'abscisse du $i^{e}$ pas Nord.
    On pose alors $a_i = s_i + 1$.\\
    La fonction de parking primitive correspondante sera ainsi
    $(a_1, \ldots, a_n)$.
\end{itemize}
\end{proof}

Nous proposons maintenant des relations de couverture pour ces deux
ensembles, telles que les posets ainsi créés soient isomorphes, et que 
l'un puisse être obtenu en appliquant la bijection ci-dessus à l'autre.

\begin{definition}[$\gtrdot_d$]
    Soient $w$ et $w'$ deux mots de Dyck de longueur $2n$.
    On dit que $w$ couvre $w'$, noté $w \gtrdot_d w'$, s'il existe deux
    mots $w_1$ et $w_2$ tels que :
    \begin{itemize}
        \item $w = w_101w_2$
        \item $w' = w_110w_2$
    \end{itemize}  
\end{definition}

\begin{rem}
    Si $w_1 \gtrdot_d w_2$, alors le chemin de Dyck correspondant à $w_2$
    est \emph{au dessus} de celui correspondant à $w_1$, et La
    \emph{différence} entre les deux chemins est un carré de côté 1.
\end{rem}

\begin{definition}[Chemins de Dyck Imbriqués]
    Deux chemins de Dyck $w_1$ et $w_2$ sont dits \emph{imbriqués}
    si $w_1$ est égal à $w_2$ ou au dessus de $w_2$. 
\end{definition}

On déduit donc la proposition suivante de la remarque précédente.

\begin{prop}
    IS'il existe une séquence $w_1 \gtrdot_d w_2 \gtrdot_d
    w_3 \gtrdot_d \cdots \gtrdot_d w_k$ avec $k \geqslant 0$,
    alors $w_1$ et $w_k$ sont \emph{imbriqués}.
\end{prop}

Cette relation de couverture engendre notre \emph{poset} pour
$\mathcal{D}_n$.
Ainsi, le poset contient l'\emph{intervalle} $[w_1;w_2]$ si et seulement
si $w_1$ et $w_2$ sont imbriqués.

On définit maintenant la relation bijective sur les fonctions de parking.
Celle-ci sera la même pour les 4 types de fonctions de parking (classiques,
classiques primitives, rationnelles, et rationnelles primitives).

\begin{definition}[$\gtrdot$]
    Soient $f$ et $g$ deux fonctions de parking.
    On dit que $f$ couvre $g$, noté $f \gtrdot g$, s'il existe $i$ tel que :
    \begin{itemize}
        \item $f = (a_1, \ldots, a_{i-1}, a_i,\ \ \ \ 
            a_{i+1}, \ldots, a_n)$
        \item $g = (a_1, \ldots, a_{i-1}, a_i - 1, a_{i+1},
        \ldots, a_n)$
    \end{itemize}
\end{definition}

Cette relation de couverture engendre notre poset pour $\mathcal{PF'}_n$.

\begin{theorem}[Théorème principal]
    Le nombre d'intervalles dans ces posets est égal au $n+1^{e}$ terme
    de la suite de l'OEIS 
    \href{https://oeis.org/A005700}{A005700}
    \footnote{https://oeis.org/A005700}.\\
    Alec Mihailovs a démontré que le $n^{e}$ terme de cette séquence est
    égal à $$\frac {6 (2n)! (2n+2)!}{n!(n+1)!(n+2)!(n+3)!}$$.
\end{theorem}

Les premiers termes de cette suite sont $1, 1, 3, 14, 84,
594, 4719, 40898, 379236, 3711916, ...$\\

\begin{proof}
    Puisque le nombre d'intervalles dans le poset de $\mathcal{D}_n$
    peut être vu comme le nombre de paires $(w_1, w_k)$ telles que
    $w_1 \gtrdot_d w_2 \gtrdot_d \cdots \gtrdot_d w_k$, alors nous pouvons
    décrire le nombre d'intervalles comme étant le nombre de 
    \emph{paires de chemins de Dyck imbriqués}.
    Ce nombre est bien égal à la suite A005700 (Bruce Westbury, 2013).
\end{proof}

On souhaite maintenant étendre cette construction au cas non-primitif.
Bien que celle sur les fonctions de parking soit la même, il
reste à expliciter la bijection, et à définir la relation de couverture
sur les mots de Dyck \emph{décorés}.

\subsection{Le cas général}

Les exemples 14 à 18 donnés en annexe B illustrent les propositions,
définitions et théorèmes de cette section.

\begin{prop}
    Puisque $\mathcal{PF}_n$ et $\mathcal{LD}_n$ ont le même cardinal,
    nous pouvons créer une \emph{bijection} entre les fonctions de parking
    classiques de longueur $n$ et les mots de Dyck décorés de
    longueur $2n$.
\end{prop}

\begin{proof}
    ~\
    \begin{itemize}
        \item $\mathcal{PF}_n \to \mathcal{LD}_n$ :
        Soit $f = (a_1, \ldots, a_n) \in \mathcal{PF}_n$ une fonction de
        parking. Pour tout $i \in \{1, \ldots, n\}$, posons $im_i$ :
        $\{j\ |\ a_j = i\}$. \\
        Notons alors $im_{i,1}, \ldots, im_{i,k_i}$ les éléments de $im_i$
        triés par ordre croissant.\\
        Le mot de Dyck décoré correspondant sera 
        $\underbrace{im_{1,1} \cdots im_{1,k_1}}_{im_1}0
         \underbrace{im_{2,1} \cdots im_{2,k_2}}_{im_2}0
         \cdots
         \underbrace{im_{n,1} \cdots im_{n,k_n}}_{im_n}0$.

        \item $\mathcal{LD}_n \to \mathcal{PF}_n$ :
        Soit $w$ un mot de Dyck décoré. Considérons sa représentation sous
        la forme d'un chemin de Dyck. Notons $s_i$ l'abscisse du $i^{e}$ pas
        Nord.\\
        On note alors $label(i)$ le label du $i^{e}$ pas nord, et
        $dist_i = \{label(j) | s_j = i\}$ l'ensemble  des labels des pas
        Nord à distance $i$ de l'axe des ordonnées.\\
        Ainsi, si $j \in dist_i$, on pose $a_j = i + 1$.\\
        La fonction de parking correspondante sera donc $(a_1, \ldots, a_n)$.
    \end{itemize}
\end{proof}

La relation suivante est l'extension de $\gtrdot_d$ au cas décoré.

\begin{definition}[$\gtrdot_{ld}$]
    Soient $w$ et $w'$ deux mots de Dyck décorés. On dit que $w$ couvre
    $w'$, noté $w \gtrdot_{ld} w'$, s'il existe $l$, $r$, $x$, $x'$, $y$,
    $z$, et $z'$ tels que :
    \begin{itemize}
        \item $l$ est le mot vide, ou finit par un $0$
        \item $r$ est le mot vide, ou commence par un $0$
        \item $x = x_1x_2 \cdots$ avec $x_i > 0$ pour tout $i$
        \item $z = z_1z_2 \cdots$ avec $z_i > 0$ pour tout $i$
        \item $x' = x$ où $y$ est correctement inséré en ordre croissant
        \item $y$ apparait dans $z$, et $z' = z$ où $y$ à été supprimé
        \item $w = lx0zr$
        \item $w' = lx'0z'r$
    \end{itemize}  
\end{definition}