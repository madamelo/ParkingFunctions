% TODO : ajouter du texte introductif

\subsection{Fonctions de Parking}

En annexe A, les exemples 1 à 4 illustrent les définitions et théorèmes
de cette section.

\begin{definition}[Fonction de Parking]
    Une \emph{fonction de parking} est une séquence d'entiers strictement
    positifs $(a_1, a_2, \ldots, a_n)$ dont le tri croissant
    $(b_1, b_2, \ldots, b_n)$ respecte la condition suivante :
    $b_i \leqslant i$ pour tout $i \leqslant n$.
\end{definition}

En d'autres termes, $\#\{i\ |\ a_i \leqslant k\} \geqslant k\ 
\forall k \leqslant n$.

On note $\mathcal{PF}_n$ l'ensemble des fonctions de parking de longueur $n$.

\begin{theorem}[Konheim et Weiss, 1966]
    Soit $pf_n$ le cardinal de $\mathcal{PF}_n$.
    Nous avons $$pf_n = (n + 1)^{n-1}.$$
\end{theorem}

\begin{definition}[Fonction de Parking Primitive]
    Une fonction de parking $(a_1, a_2, \ldots, a_n)$ est dite
    \emph{primitive} si elle est déjà triée en ordre croissant.    
\end{definition}

On note $\mathcal{PF'}_n$ l'ensemble des fonctions de parking primitives
de longueur $n$.

\begin{theorem}[Stanley, 1999]
    Soit $pf'_n$ le cardinal de $\mathcal{PF'}_n$.
    Nous avons $$pf'_n = \frac{1}{n + 1} \binom{2n}{n}.$$
\end{theorem}

Ce nombre est le $n^{e}$ nombre de Catalan $Cat(n)$.

\subsection{Chemins de Dyck}

En annexe A, les exemples 5 à 8 illustrent les définitions et théorèmes
de cette section.

\begin{notation}
    On note par $|w|_s$ le nombre d'occurences du symbole $s$ dans
    le mot $w$ .
\end{notation}

\begin{definition}[Mot de Dyck]
    Un \emph{mot de Dyck} est un mot $w \in \{0,1\}^*$ tel que :
    \begin{itemize}
        \item pour tout \emph{préfixe} $w'$ de $w$,
            $|w'|_1 \geqslant |w'|_0$.
        \item $|w|_0 = |w|_1$.
    \end{itemize}
\end{definition}

Un mot de Dyck de longueur $2n$ peut être représenté par un \emph{chemin}
allant du point $(0,0)$ au point $(n,n)$, et restant au dessus de l'axe
$y = x$, appelé \emph{chemin de Dyck} :
\begin{itemize}
    \item Chaque $1$ correspond à un \emph{pas Nord}
    $\uparrow$. 
    \item Chaque $0$ correspond à un \emph{pas Est}
    $\rightarrow$.
\end{itemize}

On note $\mathcal{D}_n$ l'ensemble des mots de Dyck de longeur $2n$.

\begin{theorem}[André, 1887]
    Soit $d_n$ le cardinal de $\mathcal{D}_n$.
    Nous avons $$d_n = \frac{1}{n + 1} \binom {2n}{n}.$$
\end{theorem}

\begin{definition}[Mot de Dyck Décoré]
    Un \emph{mot de Dyck décoré} est un mot $w \in 
    \{0, \ldots, n\}^*$ tel que :
    \begin{itemize}
        \item pour tout préfixe $w'$ de $w$,
            $|w'|_{\neq 0} \geqslant |w'|_0$.
        \item $|w|_0 = |w|_{\neq 0}$.
        \item pour tout $i \in \{1, \ldots, n\}$, $w$ contient
            exactement une occurence de $i$.
        \item si $w_i \neq 0$ et $w_{i+1} \neq 0$,
            alors $w_i < w_{i+1}$. Autrement dit, les labels de pas Nord
            consécutifs doivent être croissants.
    \end{itemize}
\end{definition}

Un mot de Dyck décoré de longueur $2n$ peut être représenté par un 
\emph{chemin} allant du point $(0,0)$ au point $(n,n)$, où chaque pas
North est associé à un label :
\begin{itemize}
    \item Chaque $i \neq 0$ correspond à un \emph{pas Nord} $\uparrow$
    de label $i$.
    \item Chaque $0$ corresponds à un \emph{pas Est} $\rightarrow$.
\end{itemize}

Ces chemins sont appelés \emph{chemins de Dyck décorés}.\\
On note $\mathcal{LD}_n$ l'ensemble des mots de Dyck décorés de longueur
$2n$.

\begin{theorem}
    Soit $ld_n$ le cardinal de $\mathcal{LD}_n$.
    Nous avons $$ld_n = (n + 1)^{n - 1}.$$
\end{theorem}