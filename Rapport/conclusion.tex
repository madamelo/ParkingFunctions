Nous avons ainsi répondu aux deux problématiques abordées :\\

Premièrement, nous avons défini des relations de couvertures créant des
posets pour les quatre types de fonctions de parking, ainsi que pour les
quatre types de chemins de Dyck que nous avons mis en bijections avec ces
dernières.
Dans chaque cas, le poset de l'un peut être obtenu en appliquant la
bijection présentée au poset de l'autre.
De plus, dans le cas rationnel, nous traitons aussi bien la configuration
$a > b$ que $a < b$.
Dans plusieurs travaux sur les fonctions de parking rationnelles
et les structures en bijection avec celles-ci, seul le cas $a < b$ avait
été traité.

La particularité de notre approche réside autant dans cette généralisation
que dans l'utilisation des chemins de Dyck.
En effet, la plupart des travaux précédents se basaient sur les
\emph{partitions non-croisées} (\cite{ref4, ref5, ref8, ref9}).
Ici, nous avons choisi de ne pas utiliser cette structure, car les posets
obtenus en appliquant la bijection Partitions Non Croisées $\to$ Fonctions
de Parking ne laissaient pas paraître de relation de couverture évidente
pour les fonctions de parking.

De plus, dans le cas rationnel, la définition de partitions non-croisées
rationnelles est complexe.
A notre connaissance, la seule définition donnée pour tous les entiers $a$
et $b$ premiers entre eux (et non seulement lorsque $a < b$) est celle donnée
par Bodnar (\cite{ref8}), qui nécessite de nombreuses étapes et définitions
intermédiaires.\\

Ensuite, nous avons étendu la notion d'arbres de parking au cas rationnel.\\

Une question qui émerge naturellement de ce travail est le besoin d'une
preuve pour la Conjecture Principale.
De plus, s'il existe une preuve entièrement combinatoire, nous pourrions
ainsi obtenir une formule et une interprétation combinatoire à la suite
entière A196304, sans nécessiter de passer par les séries génératrices.

Quant au cas rationnel, de futurs travaux pourront inclure la recherche de
formules généralisées pour le nombre d'intervalles des posets, dans le cas
primitif comme dans le cas général.

Enfin, en suivant le cheminement de \cite{ref9} sur les arbres de parking,
il pourra être intéressant d'étudier les relations de couverture sur les
arbres de parking rationnels.