\begin{example}[Définition 10 : $a > b$ : $a = 7$, $b = 3$]
    ~
    \begin{itemize}
        \item Limites pour toute séquence de $\mathcal{PF}_{7,3}$
            une fois triée : $[1,\ 1 \frac{3}{7},\ 1 \frac{6}{7},\ 
            2 \frac{2}{7},\ 2 \frac{5}{7},\ 3 \frac{1}{7},\ 
            3 \frac{4}{7}]$
        \item $f_1 = (2, 1, 1, 3, 2, 3, 1) \in
            \mathcal{PF}_{7,3}$
        \item $f_2 = (2, 1, 2, 3, 2, 3, 1) \notin
            \mathcal{PF}_{7,3}$, bien que $f_2 \in
            \mathcal{PF}_7$
    \end{itemize}
\end{example}

\begin{example}[Définition 10 : $a < b$ : $a = 5$, $b = 7$]
    ~
    \begin{itemize}
        \item Limites pour toute séquence de $\mathcal{PF}_{5,7}$
            une fois triée : $[1,\ 2 \frac{2}{5},\ 3 \frac{4}{5},\ 
            5 \frac{1}{5},\ 6 \frac{3}{5}]$
        \item $f_3 = (6, 3, 5, 1, 2) \in
            \mathcal{PF}_{5,7}$, bien que $f_3 \notin
            \mathcal{PF}_5$
        \item $f_4 = (6, 3, 5, 1, 3) \notin
            \mathcal{PF}_{5,7}$\\
    \end{itemize}
\end{example}