\documentclass{beamer}

\usepackage[francais]{babel}
\usepackage[utf8]{inputenc}
\usepackage[T1]{fontenc}

\usepackage{amsmath}
\usepackage{amssymb}
\usepackage{amsthm}
\usepackage{amsbsy}

\usepackage{tikz}
\usetikzlibrary{shapes.multipart}
\usetikzlibrary{arrows}
\usetikzlibrary{patterns}

\linespread{1.3}
\setbeamertemplate{footline}[frame number]
\usetheme{Boadilla}

\AtBeginSection[] {
  \begin{frame}
    \frametitle{Plan}
    \tableofcontents[currentsection]
  \end{frame}
}

\title{Fonctions de Parking}
\author{Tessa Lelièvre-Osswald}
\institute{Encadrant : Matthieu Josuat-Vergès \\ IRIF - Pôle Combinatoire}
\date{\today} 

\begin{document}

\frame{\titlepage}

\begin{frame}
    \frametitle{Introduction : Combinatoire}
    \begin{itemize}
        \item \textbf{Combinatoire} : domaine des mathématiques et de 
        l'informatique théorique étudiant les ensembles finis
        \emph{structurés} par leur énumération et leur comptage.
        \item \textbf{Branches principales} :
        \begin{itemize}
            \item combinatoire \emph{énumérative} : dénombrement.
            \item combinatoire \emph{bijective} : déduire une égalité
            entre les cardinaux de deux classes combinatoires  en bijection.
        \end{itemize}

        \item \textbf{Classe combinatoire} : ensemble $\mathcal{A}$ muni
        d'une application $\mathcal{A} \to \mathbb{N}$, appelée \emph{taille}. 
    \end{itemize}
\end{frame}

\begin{frame}
    \frametitle{Introduction : Exemples}
    \begin{itemize}
        \item \textbf{Classes combinatoires} :
        \begin{itemize}
            \item Mots de longueur $n$ sur l'alphabet $\{0,1\}$ :
                $|\{0,1\}^n| = 2^n$
            \item Permutations de $\{1, \ldots, n\}$ :
                $|\mathfrak{S}_n| = n!$
            \item k-cycles de $\mathfrak{S}_n$ :
                $\displaystyle|^k\mathfrak{S}_n| = \frac{n!}{(n-k)!k}$
        \end{itemize}
        \item \textbf{Bijection} : $\mathfrak{S}_n \longleftrightarrow
            \ ^{n+1}\mathfrak{S}_{n+1}$:
            \begin{itemize}
                \item $\mathfrak{S}_n \to\ ^{n+1}\mathfrak{S}_{n+1}$ :
                Soit $\sigma = a_1 \ldots a_n$ notre permutation.
                    $\sigma' = (n+1 \ a_1 \ldots a_n) \in \ 
                    ^{n+1}\mathfrak{S}_{n+1}.$
                \item $^{n+1}\mathfrak{S}_{n+1} \to\ \mathfrak{S}_n$ :
                    Soit $\sigma' = (a_1 \ldots a_{n+1})$ notre permutation
                    circulaire. Notons $i$ l'indice tel que $a_i = n + 1$.
                    $\sigma = a_{i+1} \ldots a_{n+1} a_1 \ldots a_{i-1} \in 
                    \mathfrak{S}_n.$
                \item $\displaystyle \frac{(n+1)!}{(n+1 - (n+1))!(n+1)} =
                    \frac{(n+1)!}{0!(n+1)} = \frac{n!(n+1)}{n+1} = n!$
            \end{itemize}
    \end{itemize}
\end{frame}

\begin{frame}
    \frametitle{Introduction : Chemins de Dyck}
    \begin{columns}
        \begin{column}{0.6\textwidth}
            \begin{itemize}
                \item \textbf{Mot de Dyck} : $\mathcal{D}_n = \{w \in
                    \{0,1\}^{2n}$ respectant les deux conditions $\}$ :
                \begin{itemize}
                    \item $|w|_0 = |w|_1 = n$
                    \item Pour tout préfixe $w'$ de $w$, $|w'|_0 \leqslant
                        |w'|_1$
                \end{itemize}
                \item \textbf{Chemin de Dyck} :
                \begin{itemize}
                    \item Chaque 1 devient un pas Nord ($\uparrow$)
                    \item Chaque 0 devient un pas Est ($\rightarrow$)
                \end{itemize}
                \item $\displaystyle d_n = |\mathcal{D}_n| = Cat(n) =
                    \frac{1}{n+1}\binom{2n}{n}$
            \end{itemize}
        \end{column}
        \begin{column}{0.4\textwidth}
            \begin{itemize}
                \item Exemple : $w = 1011010100$
            \end{itemize}
            \input{fig/fig1.tex}
        \end{column}
    \end{columns}
\end{frame}

\begin{frame}
    \frametitle{Introduction : Chemins de Dyck}
    \begin{columns}
        \begin{column}{0.6\textwidth}
            \begin{itemize}
                \item \textbf{Mot de Dyck étiquetté} : $\mathcal{LD}_n = 
                    \{w \in \{0,\ldots, n\}^{2n}$ respectant les 
                    trois conditions $\}$ :
                \begin{itemize}
                    \item $|w|_0 = |w|_{\neq 0} = n$
                    \item Pour tout préfixe $w'$ de $w$, $|w'|_0 \leqslant
                        |w'|_{\neq 0}$
                    \item Pour tout $i \in \{1, \ldots, n\}, |w|_i = 1$
                \end{itemize}
                \item \textbf{Chemin de Dyck étiquetté} :
                \begin{itemize}
                    \item Chaque $i \neq 0$ devient un pas Nord ($\uparrow$)
                        étiquetté par $i$
                    \item Chaque 0 devient un pas Est ($\rightarrow$)
                \end{itemize}
                \item $\displaystyle ld_n = |\mathcal{LD}_n| = (n+1)^{n-1}$
            \end{itemize}
        \end{column}
        \begin{column}{0.4\textwidth} 
                \begin{itemize}
                    \item Exemple : $w = 4015002030$
                \end{itemize}
                \begin{center}
    \begin{tikzpicture}[scale=0.6]
        \node (a) at (0, 0) {};
        \node (b) at (0, 6) {};
        \node (c) at (6, 0) {};
        \node (d) at (5.5, 5.5) {};
        \node (e) at (4.5, 6) [color = magenta]
            {$x = y$}; 
        \draw [dashed, very thick, ->] (a) to (b);
        \draw [dashed, very thick, ->] (a) to (c);
        \draw [dashed, very thick, ->]
            [color = magenta] (a) to (d);

        \node (1)  at (0,0)   {};
        \node (2)  at (0,1)   {};
        \node (3)  at (1,1)   {};
        \node (4)  at (1,2)   {};
        \node (5)  at (1,3)   {};
        \node (6)  at (2,3)   {};
        \node (7)  at (3,3)   {};
        \node (8)  at (3,4)   {};
        \node (9)  at (4,4)   {};
        \node (10) at (4,5)   {};
        \node (11) at (5,5)   {};
        \draw [->, ultra thick, color = cyan]
            (1)  to (2);
        \draw [->, ultra thick, color = cyan] 
            (2)  to (3);
        \draw [->, ultra thick, color = cyan]
            (3)  to (4);
        \draw [->, ultra thick, color = cyan]
            (4)  to (5);
        \draw [->, ultra thick, color = cyan]
            (5)  to (6);
        \draw [->, ultra thick, color = cyan]
            (6)  to (7);
        \draw [->, ultra thick, color = cyan]
            (7)  to (8);
        \draw [->, ultra thick, color = cyan]
            (8)  to (9);
        \draw [->, ultra thick, color = cyan]
            (9)  to (10);
        \draw [->, ultra thick, color = cyan]
            (10) to (11);

        \node at (-0.2, -0.2) {$0$};
        \node at (-0.3, 1)    {$1$};
        \node at (1, -0.3)    {$1$};
        \node at (-0.3, 2)    {$2$};
        \node at (2, -0.3)    {$2$};
        \node at (-0.3, 3)    {$3$};
        \node at (3, -0.3)    {$3$};
        \node at (-0.3, 4)    {$4$};
        \node at (4, -0.3)    {$4$};
        \node at (-0.3, 5)    {$5$};
        \node at (5, -0.3)    {$5$};

        \node [color = cyan] at (-1, 0.5) {\textbf{4}};
        \node [color = cyan] at (-1, 1.5) {\textbf{1}};
        \node [color = cyan] at (-1, 2.5) {\textbf{5}};
        \node [color = cyan] at (-1, 3.5) {\textbf{2}};
        \node [color = cyan] at (-1, 4.5) {\textbf{3}};

    \end{tikzpicture}
\end{center}
        \end{column}
    \end{columns}
\end{frame}

\begin{frame}
    \frametitle{Introduction : Fonctions de Parking}
    \begin{itemize}
        \item \textbf{Fonction de Parking primitive} : $\mathcal{PF'}_n =
            \{(a_1, \ldots, a_n)$ | $1 \leqslant a_i \leqslant i$ pour tout
            $i$ entre $1$ et $n$, et $a_i \leqslant \ldots \leqslant a_n\}$
        \begin{itemize}
            \item Exemple : $(1, 1, 3, 3, 4) \in \mathcal{PF'}_5$
            \item Contre-exemple : $(1, 1, 3, 2, 4) \not \in \mathcal{PF'}_5$
        \end{itemize}
        \item $\displaystyle pf'_n = |\mathcal{PF'}_n = Cat(n) =
            \frac{1}{n+1}\binom{2n}{n}$
        \item \textbf{Fonction de Parking} : $\mathcal{PF}_n = \{(a_1,
            \ldots, a_n)$ dont le tri croissant $(b_1, \ldots, b_n) \in
            \mathcal{PF'}_n\}$
        \begin{itemize}
            \item Exemple : $(1, 1, 3, 2, 4) \in \mathcal{PF}_5$
            \item Contre-exemple : $(2, 1, 4, 5, 4) \not \in \mathcal{PF}_5$
        \end{itemize}
        \item $\displaystyle pf_n = |\mathcal{PF}_n = (n+1)^{n-1}$
    \end{itemize}
\end{frame}

\begin{frame}
    \frametitle{Introduction : Posets}
    \begin{itemize}
        \item \textbf{Poset} : Ensemble $\mathcal{E}$ partiellement ordonné
        : ensemble muni d'une \emph{relation d'ordre} $\preccurlyeq$
        permettant de comparer certains couples d'éléments de l'ensemble,
        muni de propriétés :
        \begin{itemize}
            \item Réflexivité : $e \in \mathcal{E} \to e \preccurlyeq e$
            \item Anti-symétrie : $e_1 \preccurlyeq e_2 \wedge e_2
                \preccurlyeq e_1 \to e_1 = e_2$
            \item Transitivité : $e_1 \preccurlyeq e_2 \wedge e_2
                \preccurlyeq e_3 \to e_1 \preccurlyeq e_3$
        \end{itemize}
        \item \textbf{Exemple} :
            \begin{itemize}
                \item $\mathcal{E} = \mathbb{N} \times \mathbb{N}$
                \item $\preccurlyeq$ : $(a,b) \preccurlyeq (c, d)$ ssi
                    $a \leqslant c$ et $b \leqslant d$
                \item $(3,8)$ et $(2,9)$ sont incomparables
            \end{itemize}
    \end{itemize}
\end{frame}

\begin{frame}
    \frametitle{Plan}
    \tableofcontents
\end{frame}

\section{Des posets pour le cas classique}

\subsection{Posets classiques primitifs}

\begin{frame}
    \frametitle{Des relations de couverture pour $\mathcal{D}_n$ et
        $\mathcal{PF'}_n$}
    \begin{itemize}
        \item $\mathcal{D}_n$ : $w \gtrdot_d w'$, s'il existe deux mots
            $w_1$ et $w_2$ tels que :
        \begin{itemize}
            \item $w = w_101w_2$
            \item $w' = w_110w_2$
        \end{itemize} 
        \item  Si $w_1 \gtrdot_d w_2$, alors le chemin de Dyck correspondant
        à $w_2$ est \emph{au dessus} de celui correspondant à $w_1$, et la
        \emph{différence} entre les deux chemins est un carré de côté 1.
        \\~\\
        \item $\mathcal{PF'}_n$ : $f \gtrdot g$ s'il existe $i$ tel que :
        \begin{itemize}
            \item $f = (a_1, \ldots, a_{i-1}, a_i,\ \ \ \ 
                a_{i+1}, \ldots, a_n)$
            \item $g = (a_1, \ldots, a_{i-1}, a_i - 1, a_{i+1}, \ldots, a_n)$
        \end{itemize}
    \end{itemize}
\end{frame}

\begin{frame}
    \frametitle{\textbf{Bijection} entre les deux ensembles}
    \begin{itemize}
        \item $\mathcal{PF'}_n \to \mathcal{D}_n$ :
        \begin{itemize}
            \item $f = (a_1, \ldots, a_n) \in \mathcal{PF'}_n$.
            \item $l_i$ = nombre d'occurences de $i$ dans $f$.
            \item Mot de Dyck correspondant : $\underbrace{1 \cdots 1}_{l_1}
                0\underbrace{1 \cdots 1}_{l_2}0 \cdots\underbrace{1 \cdots
                1}_{l_n}0$.
        \end{itemize}
        \item $\mathcal{D}_n \to \mathcal{PF'}_n$ :
        \begin{itemize}
            \item $w \in \mathcal{D}_n$.
            \item Considérons son chemin de Dyck.
            \item $s_i$ = abscisse du $i^{e}$ pas Nord. $a_i = s_i + 1$.
            \item Fonction de parking primitive correspondante :
                $(a_1, \ldots, a_n)$.
        \end{itemize}
    \end{itemize}
\end{frame}

\begin{frame}
    \frametitle{Posets \textbf{bijectifs} obtenus pour $\mathcal{D}_4$ et
        $\mathcal{PF'}_4$}
    \begin{columns}
        \begin{column}{0.4\textwidth}
            \begin{center}
                \begin{tikzpicture}[scale = 0.13]
    \draw [ultra thick, color = cyan] (0,0) -- (0,1)
        -- (0,2) -- (0,3) -- (0,4) -- (1,4) -- (2,4)
        -- (3,4) -- (4,4);

    \draw [ultra thick, color = cyan] (0,8) -- (0,9)
        -- (0,10) -- (0,11) -- (1,11) -- (1,12) -- (2,12)
        -- (3,12) -- (4,12);
        
    \draw [ultra thick, color = cyan] (-6,16) -- (-6,17)
        -- (-6,18) -- (-5,18) -- (-5,19) -- (-5,20)
        -- (-4,20) -- (-3,20) -- (-2,20);

    \draw [ultra thick, color = cyan] (6,16) -- (6,17)
        -- (6,18) -- (6,19) -- (7,19) -- (8,19) -- (8,20)
        -- (9,20) -- (10,20);

    \draw [ultra thick, color = cyan] (-8,24) -- (-8,25)
        -- (-7,25) -- (-7,26) -- (-7,27) -- (-7,28)
        -- (-6,28) -- (-5,28) -- (-4,28);

    \draw [ultra thick, color = cyan] (0,24) -- (0,25)
        -- (0,26) -- (1,26) -- (1,27) -- (2,27) -- (2,28)
        -- (3,28) -- (4,28);

    \draw [ultra thick, color = cyan] (8,24) -- (8,25)
        -- (8,26) -- (8,27) -- (9,27) -- (10,27) -- (11,27)
        -- (11,28) -- (12,28);

    \draw [ultra thick, color = cyan] (-8,32) -- (-8,33)
        -- (-7,33) -- (-7,34) -- (-7,35) -- (-6,35)
        -- (-6,36) -- (-5,36) -- (-4,36);

    \draw [ultra thick, color = cyan] (0,32) -- (0,33)
        -- (0,34) -- (1,34) -- (2,34) -- (2,35) -- (2,36)
        -- (3,36) -- (4,36);

    \draw [ultra thick, color = cyan] (8,32) -- (8,33)
        -- (8,34) -- (9,34) -- (9,35) -- (10,35) -- (11,35)
        -- (11,36) -- (12,36);

    \draw [ultra thick, color = cyan] (-8,40) -- (-8,41)
        -- (-7,41) -- (-7,42) -- (-6,42) -- (-6,43)
        -- (-6,44) -- (-5,44) -- (-4,44);

    \draw [ultra thick, color = cyan] (0,40) -- (0,41)
        -- (1,41) -- (1,42) -- (1,43) -- (2,43) -- (3,43)
        -- (3,44) -- (4,44);

    \draw [ultra thick, color = cyan] (8,40) -- (8,41)
        -- (8,42) -- (9,42) -- (10,42) -- (10,43) -- (11,43)
        -- (11,44) -- (12,44);

    \draw [ultra thick, color = cyan] (0,48) -- (0,49)
        -- (1,49) -- (1,50) -- (2,50) -- (2,51) -- (3,51)
        -- (3,52) -- (4,52);


    \draw [->][color=magenta, ultra thick](1,47.5) to (-5.5,45);
    \draw [->][color=magenta, ultra thick](-6,39.5) to (-6,36.5);
    \draw [->][color=magenta, ultra thick](-6,39.5) to (1,36.5); 
    \draw [->][color=magenta, ultra thick](-6,31.5) to (-6,28.5);
    \draw [->][color=magenta, ultra thick](-6,31.5) to (1,28.5);
    \draw [->][color=magenta, ultra thick](-6,23.5) to (-4,20.5);
    \draw [->][color=magenta, ultra thick](-4,15.5) to (1,13);


    \draw [->][color=green, ultra thick](2,47.5) to (2,44.5);
    \draw [->][color=green, ultra thick](2,39.5) to (-3.5,36.5);
    \draw [->][color=green, ultra thick](2,39.5) to (7.5,36.5);
    \draw [->][color=green, ultra thick](2,31.5) to (2,28.5);
    \draw [->][color=green, ultra thick](2,23.5) to (-1.5,20.5);
    \draw [->][color=green, ultra thick](2,23.5) to (5.5,20.5);
    \draw [->][color=green, ultra thick](2,7.5) to (2,5);

    \draw [->][color=violet, ultra thick](3,47.5) to (9.5,45);
    \draw [->][color=violet, ultra thick](10,39.5) to (3,37);
    \draw [->][color=violet, ultra thick](10,39.5) to (10,36.5);
    \draw [->][color=violet, ultra thick](10,31.5) to (3,29);
    \draw [->][color=violet, ultra thick](10,31.5) to (10,28.5);
    \draw [->][color=violet, ultra thick](10,23.5) to (8,20.5);
    \draw [->][color=violet, ultra thick](8,15.5) to (3,13);

\end{tikzpicture}
            \end{center}
        \end{column}
        \begin{column}{0.6\textwidth}
            \begin{center}
                \begin{tikzpicture}[scale = 0.18]
    \node at (0,0)   {$(1,1,1,1)$};
    \node at (0,6)   {$(1,1,1,2)$};                
    \node at (-6,12) {$(1,1,2,2)$};
    \node at (6,12)  {$(1,1,1,3)$};
    \node at (-10,18) {$(1,2,2,2)$};
    \node at (0,18)  {$(1,1,2,3)$};
    \node at (10,18)  {$(1,1,1,4)$};
    \node at (-10,24) {$(1,2,2,3)$};
    \node at (0,24)  {$(1,1,3,3)$};
    \node at (10,24)  {$(1,1,2,4)$};
    \node at (-10,30) {$(1,2,3,3)$};
    \node at (0,30)  {$(1,2,2,4)$};
    \node at (10,30)  {$(1,1,3,4)$};
    \node at (0,36)  {$(1,2,3,4)$};


    \draw [->][color=magenta, ultra thick] (-1,35) to (-9,32);
    \draw [->][color=magenta, ultra thick](-10,29) to (-10,25);
    \draw [->][color=magenta, ultra thick](-10,29) to (-1,26); 
    \draw [->][color=magenta, ultra thick](-10,23) to (-10,19);
    \draw [->][color=magenta, ultra thick](-10,23) to (-1,20);
    \draw [->][color=magenta, ultra thick](-10,17) to (-6,13);
    \draw [->][color=magenta, ultra thick](-6,11) to (-1,8);


    \draw [->][color=green, ultra thick](0,35) to (0,31);
    \draw [->][color=green, ultra thick](0,29) to (-7,26);
    \draw [->][color=green, ultra thick](0,29) to (7,26);
    \draw [->][color=green, ultra thick](0,23) to (0,19);
    \draw [->][color=green, ultra thick](0,17) to (-4,14);
    \draw [->][color=green, ultra thick](0,17) to (4,14);
    \draw [->][color=green, ultra thick](0,5) to (0,1);

    \draw [->][color=violet, ultra thick](1,35) to (10,32);
    \draw [->][color=violet, ultra thick](10,29) to (1,26);
    \draw [->][color=violet, ultra thick](10,29) to (10,25);
    \draw [->][color=violet, ultra thick](10,23) to (1,20);
    \draw [->][color=violet, ultra thick](10,23) to (10,19);
    \draw [->][color=violet, ultra thick](10,17) to (8,13);
    \draw [->][color=violet, ultra thick](6,11) to (1,8);

\end{tikzpicture}
            \end{center}
        \end{column}
    \end{columns}
\end{frame}

\subsection{Posets classiques non-primitifs}

\section{Des posets pour le cas rationnel}

\subsection{Le cas rationnel}

\subsection{Posets rationnels primitifs}

\subsection{Posets rationnels non-primitifs}

\section{Conclusion}

\end{document}