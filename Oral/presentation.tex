\documentclass{beamer}

\usepackage[francais]{babel}
\usepackage[utf8]{inputenc}
\usepackage[T1]{fontenc}

\usepackage{amsmath}
\usepackage{amssymb}
\usepackage{amsthm}
\usepackage{amsbsy}

\usepackage{natbib}

\usepackage{tikz}
\usetikzlibrary{shapes.multipart}
\usetikzlibrary{arrows}
\usetikzlibrary{patterns}

\linespread{1.3}
\setbeamertemplate{footline}[frame number]
\usetheme{Boadilla}

\AtBeginSection[] {
  \begin{frame}
    \frametitle{Plan}
    \tableofcontents[currentsection]
  \end{frame}
}

\newtheorem*{thm}{Théorème}
\newtheorem*{conj}{Conjecture}

\title{Fonctions de Parking}
\author{Tessa Lelièvre-Osswald}
\institute{Encadrant : Matthieu Josuat-Vergès \\ IRIF - Pôle Combinatoire}
\date{\today} 

\begin{document}

\frame{\titlepage} %1

\begin{frame} %2
    \frametitle{Introduction : Combinatoire}
    \begin{itemize}
        \item \textbf{Combinatoire} : domaine des mathématiques et de 
        l'informatique théorique étudiant les ensembles finis
        \emph{structurés} par leur énumération et leur comptage.
        \item \textbf{Branches principales} :
        \begin{itemize}
            \item combinatoire \emph{énumérative} : dénombrement.
            \item combinatoire \emph{bijective} : déduire une égalité
            entre les cardinaux de deux classes combinatoires  en bijection.
        \end{itemize}

        \item \textbf{Classe combinatoire} : ensemble $\mathcal{A}$ muni
        d'une application $\mathcal{A} \to \mathbb{N}$, appelée \emph{taille}. 
    \end{itemize}
\end{frame}

\begin{frame} %3
    \frametitle{Introduction : Exemples}
    \begin{itemize}
        \item \textbf{Classes combinatoires} :
        \begin{itemize}
            \item Mots de longueur $n$ sur l'alphabet $\{0,1\}$ :
                $|\{0,1\}^n| = 2^n$
            \item Permutations de $\{1, \ldots, n\}$ :
                $|\mathfrak{S}_n| = n!$
            \item k-cycles de $\mathfrak{S}_n$ :
                $\displaystyle|^k\mathfrak{S}_n| = \frac{n!}{(n-k)!k}$
        \end{itemize}
        \item \textbf{Bijection} : $\mathfrak{S}_n \longleftrightarrow
            \ ^{n+1}\mathfrak{S}_{n+1}$:
            \begin{itemize}
                \item $\mathfrak{S}_n \to\ ^{n+1}\mathfrak{S}_{n+1}$ :
                Soit $\sigma = a_1 \ldots a_n$ notre permutation.
                    $\sigma' = (n+1 \ a_1 \ldots a_n) \in \ 
                    ^{n+1}\mathfrak{S}_{n+1}.$
                \item $^{n+1}\mathfrak{S}_{n+1} \to\ \mathfrak{S}_n$ :
                    Soit $\sigma' = (a_1 \ldots a_{n+1})$ notre permutation
                    circulaire. Notons $i$ l'indice tel que $a_i = n + 1$.
                    $\sigma = a_{i+1} \ldots a_{n+1} a_1 \ldots a_{i-1} \in 
                    \mathfrak{S}_n.$
                \item $\displaystyle \frac{(n+1)!}{(n+1 - (n+1))!(n+1)} =
                    \frac{(n+1)!}{0!(n+1)} = \frac{n!(n+1)}{n+1} = n!$
            \end{itemize}
    \end{itemize}
\end{frame}

\begin{frame} %4
    \frametitle{Introduction : Chemins de Dyck}
    \begin{columns}
        \begin{column}{0.6\textwidth}
            \begin{itemize}
                \item \textbf{Mot de Dyck} : $\mathcal{D}_n =$ ensemble
                des $w \in \{0,1\}^{2n}$ respectant les deux conditions :
                \begin{itemize}
                    \item $|w|_0 = |w|_1 = n$
                    \item Pour tout préfixe $w'$ de $w$, $|w'|_0 \leqslant
                        |w'|_1$
                \end{itemize}
                \item \textbf{Chemin de Dyck} :
                \begin{itemize}
                    \item Chaque 1 devient un pas Nord ($\uparrow$)
                    \item Chaque 0 devient un pas Est ($\rightarrow$)
                \end{itemize}
            \end{itemize}

            \begin{thm}[Taille de $\mathcal{D}_n$]
                $\displaystyle d_n = |\mathcal{D}_n| = Cat(n) =
                    \frac{1}{n+1}\binom{2n}{n}$
            \end{thm}

        \end{column}
        \begin{column}{0.4\textwidth}
            \begin{itemize}
                \item Exemple : $w = 1011010100$
            \end{itemize}
            \input{fig/fig1.tex}
        \end{column}
    \end{columns}
\end{frame}

\begin{frame} %5
    \fontsize{9}{11}
    \frametitle{Introduction : Chemins de Dyck étiquettés}
    \begin{columns}
        \begin{column}{0.65\textwidth}
            \begin{itemize}
                \item \textbf{Mot de Dyck étiquetté} : $\mathcal{LD}_n =$
                    ensemble des $w \in \{0,\ldots, n\}^{2n}$ respectant les 
                    quatre conditions :
                \begin{itemize}
                    \item $|w|_0 = |w|_{\neq 0} = n$
                    \item Pour tout préfixe $w'$ de $w$, $|w'|_0 \leqslant
                        |w'|_{\neq 0}$
                    \item Pour tout $i \in \{1, \ldots, n\}, |w|_i = 1$
                    \item Si $w_i, w_{i+1} > 0$, alors $w_i < w_{i+1}$
                \end{itemize}
                \item \textbf{Chemin de Dyck étiquetté} :
                \begin{itemize}
                    \item Chaque $i \neq 0$ devient un pas Nord ($\uparrow$)
                        étiquetté par $i$
                    \item Chaque 0 devient un pas Est ($\rightarrow$)
                \end{itemize}
            \end{itemize}

            \begin{thm}[Taille de $\mathcal{LD}_n$]
                $\displaystyle ld_n = |\mathcal{LD}_n| =
                (n+1)^{n-1}$
            \end{thm}

        \end{column}
        \begin{column}{0.35\textwidth} 
                \begin{itemize}
                    \item Exemple : $w = 4015002030$
                \end{itemize}
                \begin{center}
    \begin{tikzpicture}[scale=0.6]
        \node (a) at (0, 0) {};
        \node (b) at (0, 6) {};
        \node (c) at (6, 0) {};
        \node (d) at (5.5, 5.5) {};
        \node (e) at (4.5, 6) [color = magenta]
            {$x = y$}; 
        \draw [dashed, very thick, ->] (a) to (b);
        \draw [dashed, very thick, ->] (a) to (c);
        \draw [dashed, very thick, ->]
            [color = magenta] (a) to (d);

        \node (1)  at (0,0)   {};
        \node (2)  at (0,1)   {};
        \node (3)  at (1,1)   {};
        \node (4)  at (1,2)   {};
        \node (5)  at (1,3)   {};
        \node (6)  at (2,3)   {};
        \node (7)  at (3,3)   {};
        \node (8)  at (3,4)   {};
        \node (9)  at (4,4)   {};
        \node (10) at (4,5)   {};
        \node (11) at (5,5)   {};
        \draw [->, ultra thick, color = cyan]
            (1)  to (2);
        \draw [->, ultra thick, color = cyan] 
            (2)  to (3);
        \draw [->, ultra thick, color = cyan]
            (3)  to (4);
        \draw [->, ultra thick, color = cyan]
            (4)  to (5);
        \draw [->, ultra thick, color = cyan]
            (5)  to (6);
        \draw [->, ultra thick, color = cyan]
            (6)  to (7);
        \draw [->, ultra thick, color = cyan]
            (7)  to (8);
        \draw [->, ultra thick, color = cyan]
            (8)  to (9);
        \draw [->, ultra thick, color = cyan]
            (9)  to (10);
        \draw [->, ultra thick, color = cyan]
            (10) to (11);

        \node at (-0.2, -0.2) {$0$};
        \node at (-0.3, 1)    {$1$};
        \node at (1, -0.3)    {$1$};
        \node at (-0.3, 2)    {$2$};
        \node at (2, -0.3)    {$2$};
        \node at (-0.3, 3)    {$3$};
        \node at (3, -0.3)    {$3$};
        \node at (-0.3, 4)    {$4$};
        \node at (4, -0.3)    {$4$};
        \node at (-0.3, 5)    {$5$};
        \node at (5, -0.3)    {$5$};

        \node [color = cyan] at (-1, 0.5) {\textbf{4}};
        \node [color = cyan] at (-1, 1.5) {\textbf{1}};
        \node [color = cyan] at (-1, 2.5) {\textbf{5}};
        \node [color = cyan] at (-1, 3.5) {\textbf{2}};
        \node [color = cyan] at (-1, 4.5) {\textbf{3}};

    \end{tikzpicture}
\end{center}
        \end{column}
    \end{columns}
\end{frame}

\begin{frame} %6
    \frametitle{Introduction : Fonctions de Parking}
    \only<1>{
    \begin{itemize}
        \item \textbf{Fonction de Parking primitive} : $\mathcal{PF'}_n =
            \{(a_1, \ldots, a_n)$ | $1 \leqslant a_i \leqslant i$ pour tout
            $i$ entre $1$ et $n$, et $a_i \leqslant \ldots \leqslant a_n\}$
        \begin{itemize}
            \item Exemple : $(1, 1, 3, 3, 4) \in \mathcal{PF'}_5$
            \item Contre-exemple : $(1, 1, 3, \color{blue}{2} \color{black},
                4) \not \in \mathcal{PF'}_5$
        \end{itemize}
    \end{itemize}

    \begin{thm}[Taille de $\mathcal{PF'}_n$]
        $\displaystyle pf'_n = |\mathcal{PF'}_n| = Cat(n)
        = \frac{1}{n+1}\binom{2n}{n}$
    \end{thm}

    }\only<2>{
    \begin{itemize}
        \item \textbf{Fonction de Parking} : $\mathcal{PF}_n = \{(a_1,
            \ldots, a_n)$ dont le tri croissant $(b_1, \ldots, b_n) \in
            \mathcal{PF'}_n\}$
        \begin{itemize}
            \item Exemple : $(1, 1, 3, 2, 4) \in \mathcal{PF}_5$
            \item Contre-exemple : $(2, 1, 4, 5, 4) \not \in \mathcal{PF}_5$,
                car $(1, 2, \color{blue}{4} \color{black}, 4, 5) \not \in
                \mathcal{PF'}_5$
        \end{itemize}
    \end{itemize}

    \begin{thm}[Taille de $\mathcal{PF}_n$]
        $\displaystyle pf_n = |\mathcal{PF}_n| = (n+1)^{n-1}$
    \end{thm}

    }

\end{frame}

\begin{frame} %7
    \frametitle{Introduction : Posets}
    \only<1>{
    \begin{itemize}
        \item \textbf{Poset} : Ensemble $\mathcal{E}$ partiellement ordonné
        : ensemble muni d'une \emph{relation d'ordre} $\preccurlyeq$
        permettant de comparer certains couples d'éléments de l'ensemble,
        muni de propriétés :
        \begin{itemize}
            \item Réflexivité : Si $e \in \mathcal{E}$, alors $e 
                \preccurlyeq e$
            \item Anti-symétrie : Si $e_1 \preccurlyeq e_2$ et $e_2
                \preccurlyeq e_1$, alors $e_1 = e_2$
            \item Transitivité : Si $e_1 \preccurlyeq e_2$ et $e_2
                \preccurlyeq e_3$, alors $e_1 \preccurlyeq e_3$
        \end{itemize}
    \end{itemize}

    }\only<2>{
    \begin{itemize}
        \item \textbf{Exemple} :
            \begin{itemize}
                \item $\mathcal{E} = \mathbb{N} \times \mathbb{N}$
                \item $\preccurlyeq$ : $(a,b) \preccurlyeq (c, d)$ ssi
                    $a \leqslant c$ et $b \leqslant d$.
                \item $(3,8)$ et $(2,9)$ sont incomparables.
            \end{itemize}
        \item \textbf{Ici} : On définira nos ordres via une \emph{relation
            de couverture}, dont on construira la clôture réfléxive et
            transitive.
        \item \textbf{Exemple} : $\mathcal{E} = \mathbb{N}$
            \begin{itemize} 
                \item $e \prec e + 1$
                \item Soit $\preccurlyeq$ la clôture réfléxive et
                    transitive de $\prec$.
                \item On a alors $\preccurlyeq\ =\ \leqslant$.
            \end{itemize}
    \end{itemize}
    }
\end{frame}

\begin{frame} %8
    \frametitle{Plan}
    \tableofcontents
\end{frame}

\section{Des posets pour le cas classique} %9

\subsection{Posets classiques primitifs}

\begin{frame} %10
    \frametitle{1.1) Des relations de couverture pour $\mathcal{D}_n$ et
        $\mathcal{PF'}_n$}
    \begin{itemize}
        \item $\mathcal{D}_n$ : $w \gtrdot_d w'$, s'il existe deux mots
            $w_1$ et $w_2$ tels que :
        \begin{itemize}
            \item $w = w_1\color{red}01\color{black}w_2$
            \item $w' = w_1\color{red}10\color{black}w_2$
        \end{itemize} 
        \item  Si $w_1 \gtrdot_d w_2$, alors le chemin de Dyck correspondant
        à $w_2$ est \emph{au dessus} de celui correspondant à $w_1$, et la
        \emph{différence} entre les deux chemins est un carré de côté 1.
        \\~\\
        \item $\mathcal{PF'}_n$ : $f \gtrdot g$ s'il existe $i$ tel que :
        \begin{itemize}
            \item $f = (a_1, \ldots, a_{i-1}, a_i,\ \ \ \ 
                a_{i+1}, \ldots, a_n)$
            \item $g = (a_1, \ldots, a_{i-1}, a_i \color{red} - 1\color{black}
                , a_{i+1}, \ldots, a_n)$
        \end{itemize}
    \end{itemize}
\end{frame}

\begin{frame} %11
    \frametitle{1.1) \textbf{Bijection} entre les deux ensembles}
    \begin{itemize}
        \item $\mathcal{PF'}_n \to \mathcal{D}_n$ :
        \begin{itemize}
            \item $f = (a_1, \ldots, a_n) \in \mathcal{PF'}_n$.
            \item $l_i$ = nombre d'occurences de $i$ dans $f$.
            \item Mot de Dyck correspondant : $\underbrace{1 \cdots 1}_{l_1}
                0\underbrace{1 \cdots 1}_{l_2}0 \cdots\underbrace{1 \cdots
                1}_{l_n}0$.
        \end{itemize}
        \item $\mathcal{D}_n \to \mathcal{PF'}_n$ :
        \begin{itemize}
            \item $w \in \mathcal{D}_n$.
            \item Considérons son chemin de Dyck.
            \item $s_i$ = abscisse du $i^{e}$ pas Nord. $a_i = s_i + 1$.
            \item Fonction de parking primitive correspondante :
                $(a_1, \ldots, a_n)$.
        \end{itemize}
    \end{itemize}
\end{frame}

\begin{frame} %12
    \frametitle{1.1) Posets \textbf{bijectifs} obtenus pour $\mathcal{D}_4$
        et $\mathcal{PF'}_4$}
    \begin{columns}
        \begin{column}{0.4\textwidth}
            \begin{center}
                \begin{tikzpicture}[scale = 0.13]
    \draw [ultra thick, color = cyan] (0,0) -- (0,1)
        -- (0,2) -- (0,3) -- (0,4) -- (1,4) -- (2,4)
        -- (3,4) -- (4,4);

    \draw [ultra thick, color = cyan] (0,8) -- (0,9)
        -- (0,10) -- (0,11) -- (1,11) -- (1,12) -- (2,12)
        -- (3,12) -- (4,12);
        
    \draw [ultra thick, color = cyan] (-6,16) -- (-6,17)
        -- (-6,18) -- (-5,18) -- (-5,19) -- (-5,20)
        -- (-4,20) -- (-3,20) -- (-2,20);

    \draw [ultra thick, color = cyan] (6,16) -- (6,17)
        -- (6,18) -- (6,19) -- (7,19) -- (8,19) -- (8,20)
        -- (9,20) -- (10,20);

    \draw [ultra thick, color = cyan] (-8,24) -- (-8,25)
        -- (-7,25) -- (-7,26) -- (-7,27) -- (-7,28)
        -- (-6,28) -- (-5,28) -- (-4,28);

    \draw [ultra thick, color = cyan] (0,24) -- (0,25)
        -- (0,26) -- (1,26) -- (1,27) -- (2,27) -- (2,28)
        -- (3,28) -- (4,28);

    \draw [ultra thick, color = cyan] (8,24) -- (8,25)
        -- (8,26) -- (8,27) -- (9,27) -- (10,27) -- (11,27)
        -- (11,28) -- (12,28);

    \draw [ultra thick, color = cyan] (-8,32) -- (-8,33)
        -- (-7,33) -- (-7,34) -- (-7,35) -- (-6,35)
        -- (-6,36) -- (-5,36) -- (-4,36);

    \draw [ultra thick, color = cyan] (0,32) -- (0,33)
        -- (0,34) -- (1,34) -- (2,34) -- (2,35) -- (2,36)
        -- (3,36) -- (4,36);

    \draw [ultra thick, color = cyan] (8,32) -- (8,33)
        -- (8,34) -- (9,34) -- (9,35) -- (10,35) -- (11,35)
        -- (11,36) -- (12,36);

    \draw [ultra thick, color = cyan] (-8,40) -- (-8,41)
        -- (-7,41) -- (-7,42) -- (-6,42) -- (-6,43)
        -- (-6,44) -- (-5,44) -- (-4,44);

    \draw [ultra thick, color = cyan] (0,40) -- (0,41)
        -- (1,41) -- (1,42) -- (1,43) -- (2,43) -- (3,43)
        -- (3,44) -- (4,44);

    \draw [ultra thick, color = cyan] (8,40) -- (8,41)
        -- (8,42) -- (9,42) -- (10,42) -- (10,43) -- (11,43)
        -- (11,44) -- (12,44);

    \draw [ultra thick, color = cyan] (0,48) -- (0,49)
        -- (1,49) -- (1,50) -- (2,50) -- (2,51) -- (3,51)
        -- (3,52) -- (4,52);


    \draw [->][color=magenta, ultra thick](1,47.5) to (-5.5,45);
    \draw [->][color=magenta, ultra thick](-6,39.5) to (-6,36.5);
    \draw [->][color=magenta, ultra thick](-6,39.5) to (1,36.5); 
    \draw [->][color=magenta, ultra thick](-6,31.5) to (-6,28.5);
    \draw [->][color=magenta, ultra thick](-6,31.5) to (1,28.5);
    \draw [->][color=magenta, ultra thick](-6,23.5) to (-4,20.5);
    \draw [->][color=magenta, ultra thick](-4,15.5) to (1,13);


    \draw [->][color=green, ultra thick](2,47.5) to (2,44.5);
    \draw [->][color=green, ultra thick](2,39.5) to (-3.5,36.5);
    \draw [->][color=green, ultra thick](2,39.5) to (7.5,36.5);
    \draw [->][color=green, ultra thick](2,31.5) to (2,28.5);
    \draw [->][color=green, ultra thick](2,23.5) to (-1.5,20.5);
    \draw [->][color=green, ultra thick](2,23.5) to (5.5,20.5);
    \draw [->][color=green, ultra thick](2,7.5) to (2,5);

    \draw [->][color=violet, ultra thick](3,47.5) to (9.5,45);
    \draw [->][color=violet, ultra thick](10,39.5) to (3,37);
    \draw [->][color=violet, ultra thick](10,39.5) to (10,36.5);
    \draw [->][color=violet, ultra thick](10,31.5) to (3,29);
    \draw [->][color=violet, ultra thick](10,31.5) to (10,28.5);
    \draw [->][color=violet, ultra thick](10,23.5) to (8,20.5);
    \draw [->][color=violet, ultra thick](8,15.5) to (3,13);

\end{tikzpicture}
            \end{center}
        \end{column}
        \begin{column}{0.6\textwidth}
            \begin{center}
                \begin{tikzpicture}[scale = 0.18]
    \node at (0,0)   {$(1,1,1,1)$};
    \node at (0,6)   {$(1,1,1,2)$};                
    \node at (-6,12) {$(1,1,2,2)$};
    \node at (6,12)  {$(1,1,1,3)$};
    \node at (-10,18) {$(1,2,2,2)$};
    \node at (0,18)  {$(1,1,2,3)$};
    \node at (10,18)  {$(1,1,1,4)$};
    \node at (-10,24) {$(1,2,2,3)$};
    \node at (0,24)  {$(1,1,3,3)$};
    \node at (10,24)  {$(1,1,2,4)$};
    \node at (-10,30) {$(1,2,3,3)$};
    \node at (0,30)  {$(1,2,2,4)$};
    \node at (10,30)  {$(1,1,3,4)$};
    \node at (0,36)  {$(1,2,3,4)$};


    \draw [->][color=magenta, ultra thick] (-1,35) to (-9,32);
    \draw [->][color=magenta, ultra thick](-10,29) to (-10,25);
    \draw [->][color=magenta, ultra thick](-10,29) to (-1,26); 
    \draw [->][color=magenta, ultra thick](-10,23) to (-10,19);
    \draw [->][color=magenta, ultra thick](-10,23) to (-1,20);
    \draw [->][color=magenta, ultra thick](-10,17) to (-6,13);
    \draw [->][color=magenta, ultra thick](-6,11) to (-1,8);


    \draw [->][color=green, ultra thick](0,35) to (0,31);
    \draw [->][color=green, ultra thick](0,29) to (-7,26);
    \draw [->][color=green, ultra thick](0,29) to (7,26);
    \draw [->][color=green, ultra thick](0,23) to (0,19);
    \draw [->][color=green, ultra thick](0,17) to (-4,14);
    \draw [->][color=green, ultra thick](0,17) to (4,14);
    \draw [->][color=green, ultra thick](0,5) to (0,1);

    \draw [->][color=violet, ultra thick](1,35) to (10,32);
    \draw [->][color=violet, ultra thick](10,29) to (1,26);
    \draw [->][color=violet, ultra thick](10,29) to (10,25);
    \draw [->][color=violet, ultra thick](10,23) to (1,20);
    \draw [->][color=violet, ultra thick](10,23) to (10,19);
    \draw [->][color=violet, ultra thick](10,17) to (8,13);
    \draw [->][color=violet, ultra thick](6,11) to (1,8);

\end{tikzpicture}
            \end{center}
        \end{column}
    \end{columns}
\end{frame}

\begin{frame} %13
    \frametitle{1.1) Théorème Principal}
    \only<1>{
    \begin{thm}[Théorème principal]
    Le nombre d'intervalles dans ces posets est égal à
    $$\frac {6 (2n)! (2n+2)!}{n!(n+1)!(n+2)!(n+3)!}$$.
    \end{thm}

    \begin{proof}
        Puisque le nombre d'intervalles dans le poset de $\mathcal{D}_n$
        peut être vu comme le nombre de paires $(w_1, w_k)$ telles que
        $w_1 \gtrdot_d w_2 \gtrdot_d \cdots \gtrdot_d w_k$, alors nous pouvons
        décrire le nombre d'intervalles comme étant le nombre de 
        \emph{paires de chemins de Dyck imbriqués}.\\
        On introduit alors une notion de chemin de Dyck \emph{étoilé},
        c'est-à-dire un chemin de Dyck dont certains pas sont annotés d'une
        étoile (*).
    \end{proof}
    }\only<2>{
    \begin{proof}
        Ces chemins sont une représentation des paires de chemins de
        Dyck imbriqués : en omettant les étoiles, on obtient le chemin du
        dessous.
        Ensuite, pour déduire le chemin du dessus :
        \begin{itemize}
            \item Un pas non étoilé est conservé
            \item Un pas Nord étoilé est remplacé par un pas Est
            \item Un pas Est étoilé est remplacé par un pas Nord.
        \end{itemize}
        On créé maintenant une bijection entre les chemins de Dyck étoilés
        de longueur $2n$ correspondant à des paires de chemins de Dyck de
        longueur $2n$, et les chemins allant du point (0,0) au point (0,0)
        composées de $2n$ pas Nord, Est, Sud, Ouest qui restent dans le
        premier octant.
        Pour cela, on effectue la transformation suivante :
    \end{proof}
    }\only<3>{
    \begin{proof}
        \begin{itemize}
            \item Nord non-étoilé $\longleftrightarrow$ Nord et
                Nord étoilé $\longleftrightarrow$ Ouest
            \item Est non-étoilé $\longleftrightarrow$ Sud et
                Est étoilé $\longleftrightarrow$ Est
        \end{itemize}
        Ainsi, puisque $\mathcal{D}_n \longleftrightarrow \{$Chemins de Dyck
        étoilés de longueur $2n\} \longleftrightarrow \{$ Chemins NESO de
        longueur $2n\}$, on sait que le nombre de paires de chemins de Dyck
        imbriqués de longueur $2n$ est égal au nombre de chemins NESO de
        longueur $2n$.
        Or, par définition, ce nombre est égal au $n+1^{e}$ terme de la suite
        de l'OEIS A005700 (Ce qui rejoint le commentaire de 
        Bruce Westbury).
        Alec Mihailovs a démontré que ce nombre est bien égal à 
        $\displaystyle\frac {6 (2n)! (2n+2)!}{n!(n+1)!(n+2)!(n+3)!}$.
    \end{proof}

    Les premiers termes de cette suite sont 1, 1, 3, 14, 84,
    594, 4719, 40898 ...
    }
\end{frame}

\subsection{Posets classiques non-primitifs}

\begin{frame} %14
    \frametitle{1.2) Des relations de couverture pour $\mathcal{LD}_n$ et
        $\mathcal{PF}_n$}
    \begin{itemize}
        \item $\mathcal{LD}_n$ : $w \gtrdot_{ld} w'$ s'il existe des mots
            $l$, $r$, $x$, $x'$, $z$, $z'$, et une lettre $y$ tels que :
        \begin{itemize}
            \item $l$ est le mot vide, ou finit par un $0$
            \item $r$ est le mot vide, ou commence par un $0$
            \item $x = x_1x_2 \cdots$ avec $x_i > 0$ pour tout $i$
            \item $z = z_1z_2 \cdots$ avec $z_i > 0$ pour tout $i$
            \item $x' = x$ où $y$ est correctement inséré en ordre croissant
            \item $y$ apparait dans $z$, et $z' = z$ où $y$ à été supprimé
            \item $w = l\color{red}x\color{black}0\color{red}z\color{black}
                r$ et $w' = l\color{red}x'\color{black}0\color{red}z'
                \color{black}r$
        \end{itemize}
        \item  Si $w_1 \gtrdot_{ld} w_2$, alors le chemin de Dyck correspondant
        à $w_2$ est \emph{au dessus} de celui correspondant à $w_1$, et la
        \emph{différence} entre les deux chemins est un carré de côté 1.
        \item $\mathcal{PF}_n$ : on garde la même relation que pour
            $\mathcal{PF'}_n$.
    \end{itemize}
\end{frame}

\begin{frame} %15
    \frametitle{1.2) \textbf{Bijection} entre les deux ensembles}
    \begin{itemize}
        \item $\mathcal{PF}_n \to \mathcal{LD}_n$ :
        \begin{itemize}
            \item $f = (a_1, \ldots, a_n) \in \mathcal{PF}_n$.
            \item $im_i$ : $\{j\ |\ a_j = i\}$.
            \item $im_{i,1}, \ldots, im_{i,k_i}$ = éléments de $im_i$
                triés par ordre croissant.
            \item Mot de Dyck décoré correspondant : $\underbrace{im_{1,1}
                \cdots im_{1,k_1}}_{im_1}0\underbrace{im_{2,1} \cdots
                im_{2,k_2}}_{im_2}0 \cdots \underbrace{im_{n,1} \cdots 
                im_{n,k_n}}_{im_n}0$.
        \end{itemize}
        \item $\mathcal{LD}_n \to \mathcal{PF}_n$ :
        \begin{itemize}
            \item $w \in \mathcal{LD}_n$. Considérons son chemin de Dyck.
            \item $s_i$ = abscisse du $i^{e}$ pas Nord.
            \item $label(i)$ = étiquette du $i^{e}$ pas Nord.
            \item $dist_i = \{label(j) | s_j = i\}$.
                Si $j \in dist_i$ alors $a_j = i + 1$.
            \item Fonction de parking correspondante : $(a_1, \ldots, a_n)$.
        \end{itemize}
    \end{itemize}
\end{frame}

\begin{frame} %16
    \frametitle{1.2) Posets \textbf{bijectifs} obtenus pour $\mathcal{LD}_3$
        et $\mathcal{PF}_3$}
    \begin{center}
            \begin{center}
        \begin{tikzpicture}[scale=0.7]
            \node (a) at (0, 0) {};
            \node (b) at (0, 7) {};
            \node (c) at (7, 0) {};
            \node (d) at (6.5, 6.5) {};
            \node (e) at (7, 6) [color = magenta]
                {$x = y$}; 
            \draw [dashed, very thick, ->] (a) to (b);
            \draw [dashed, very thick, ->] (a) to (c);
            \draw [dashed, very thick, ->]
                [color = magenta] (a) to (d);

            \node (1)  at (0,0)   {};
            \node (2)  at (0,1)   {};
            \node (3)  at (1,1)   {};
            \node (4)  at (1,2)   {};
            \node (5)  at (2,2)   {};
            \node (6)  at (2,3)   {};
            \node (7)  at (2,4)   {};
            \node (8)  at (3,4)   {};
            \node (9)  at (3,5)   {};
            \node (10) at (4,5)   {};
            \node (11) at (5,5)   {};
            \node (12) at (5,6)   {};
            \node (13) at (6,6)   {};
            \draw [->, ultra thick, color = cyan]
                (1)  to (2);
            \draw [->, ultra thick, color = cyan] 
                (2)  to (3);
            \draw [->, ultra thick, color = cyan]
                (3)  to (4);
            \draw [->, ultra thick, color = cyan]
                (4)  to (5);
            \draw [->, ultra thick, color = cyan]
                (5)  to (6);
            \draw [->, ultra thick, color = cyan]
                (6)  to (7);
            \draw [->, ultra thick, color = cyan]
                (7)  to (8);
            \draw [->, ultra thick, color = cyan]
                (8)  to (9);
            \draw [->, ultra thick, color = cyan]
                (9)  to (10);
            \draw [->, ultra thick, color = cyan]
                (10) to (11);
            \draw [->, ultra thick, color = cyan]
                (11) to (12);
            \draw [->, ultra thick, color = cyan]
                (12) to (13);

            \node at (-0.2, -0.2) {$0$};
            \node at (-0.3, 1)    {$1$};
            \node at (1, -0.3)    {$1$};
            \node at (-0.3, 2)    {$2$};
            \node at (2, -0.3)    {$2$};
            \node at (-0.3, 3)    {$3$};
            \node at (3, -0.3)    {$3$};
            \node at (-0.3, 4)    {$4$};
            \node at (4, -0.3)    {$4$};
            \node at (-0.3, 5)    {$5$};
            \node at (5, -0.3)    {$5$};
            \node at (-0.3, 6)    {$6$};
            \node at (6, -0.3)    {$6$};

        \end{tikzpicture}
    \end{center}
    \end{center}
\end{frame}

\begin{frame} %17
    \frametitle{1.2) Posets \textbf{bijectifs} obtenus pour $\mathcal{LD}_3$
        et $\mathcal{PF}_3$}
    \begin{center}
        \input{fig/fig6.tex}
    \end{center}
\end{frame}

\begin{frame} %18
    \frametitle{1.2) Conjecture Principale}
    \only<1>{
    \begin{conj}[Conjecture Principale]
        Le nombre d'intervalles des posets définis ici pour $\mathcal{LD}_n$
        et $\mathcal{PF}_n$ est donnée par :
        \begin{itemize}
            \item $a(1) = 1$
            \item $\displaystyle a(n+1) = \sum_{k = 1}^{n}{(-1)^{k-1}
                \binom{n}{k} \binom{n+2-k}{2}^k a (n+1-k)}$
        \end{itemize}
    \end{conj}
    \begin{itemize}
        \item Cette formule donne le $n+1^{e}$ terme de la suite de l'OEIS
        A196304.
        \item Les premiers termes de cette suite sont $1, 1, 5, 64, 1587,
        65421, 4071178, 357962760, 4237910716, ...$.
    \end{itemize}
    }\only<2>{
    
    Bien que, à notre connaissance, il n'existe pas de structure combinatoire associée à cette suite, les tests effectués via Sagemath
    sur $n = 1, 2, \cdots, 8$ suggèrent que le nombre d'intervalles de nos
    posets puissent en être une.
    }
\end{frame}

\section{Des posets pour le cas rationnel} %19

\subsection{Le cas rationnel}

\begin{frame} %20
    \frametitle{2.1) Chemins de Dyck rationnels}
    \begin{columns}
        \begin{column}{0.65\textwidth}
            \begin{itemize}
                \item \textbf{Mot de Dyck rationnel} : $\mathcal{R}_{a,b} =$
                    l'ensemble des $w \in \{0,1\}^{a+b}$ respectant les trois
                    conditions :
                \begin{itemize}
                    \item Pour tout préfixe $w'$ de $w$, $\frac{a}{b}|w'|_0
                     \leqslant |w'|_1$
                     \item $|w|_0 = b$
                     \item $|w|_1 = a$
                \end{itemize}
                \item \textbf{Chemin de Dyck rationnel} :
                \begin{itemize}
                    \item Chaque 1 devient un pas Nord ($\uparrow$)
                    \item Chaque 0 devient un pas Est ($\rightarrow$)
                \end{itemize}
            \end{itemize}

            \begin{thm} [Taille de $\mathcal{R}_{a,b}$]           
                $\displaystyle r_{a,b} = |\mathcal{R}_{a,b}|
                = Cat(a,b) = \frac{1}{a+b}\binom{a+b}{a}$
            \end{thm}

        \end{column}
        \begin{column}{0.35\textwidth}
            \begin{itemize}
                \item Exemple : $w = 1110111010 \in \mathcal{R}_{7,3}$
            \end{itemize}
            \input{fig/fig8.tex}
        \end{column}
    \end{columns}
\end{frame}

\begin{frame} %21
    \frametitle{2.1) Chemins de Dyck rationnels}
    \begin{itemize}
        \item $w = 10100100 \in \mathcal{R}_{3,5}$
    \end{itemize}
    \input{fig/fig7.tex}
\end{frame}

\begin{frame} %22
    \fontsize{9}{11}
    \frametitle{2.1) Chemins de Dyck rationnels étiquettés}
    \begin{columns}
        \begin{column}{0.7\textwidth}
            \begin{itemize}
                \item \textbf{Mot de Dyck rationnel étiquetté} : 
                    $\mathcal{LR}_{a,b} =$ l'ensemble des $w \in
                    \{0,\ldots, a\}^{a+b}$ respectant les cinq conditions :
                \begin{itemize}
                    \item Pour tout préfixe $w'$ de $w$, $\frac{a}{b}|w'|_0
                        \leqslant |w'|_{\neq 0}$
                    \item Pour tout $i \in \{1, \ldots, a\}, |w|_i = 1$
                    \item $|w|_0 = a$ et $|w|_{\neq 0} = b$
                    \item Les entiers non-nuls consécutifs sont en ordre
                        croissant.
                \end{itemize}
                \item \textbf{Chemin de Dyck rationnel étiquetté} :
                \begin{itemize}
                    \item Chaque $i \neq 0$ devient un pas Nord ($\uparrow$)
                        étiquetté par $i$
                    \item Chaque 0 devient un pas Est ($\rightarrow$)
                \end{itemize}
            \end{itemize}

            \begin{thm} [Taille de $\mathcal{LR}_{a,b}$]
                $\displaystyle lr_{a,b} = |\mathcal{LR}_{a,b}| = b^{a-1}$
            \end{thm}

        \end{column}
        \begin{column}{0.3\textwidth} 
                \begin{itemize}
                    \item Exemple : $w = 2456017030 \in \mathcal{LR}_{7,3}$
                \end{itemize}
                \input{fig/fig9.tex}
        \end{column}
    \end{columns}
\end{frame}

\begin{frame} %23
    \frametitle{2.1) Chemins de Dyck rationnels étiquettés}
    \begin{itemize}
        \item Exemple : $w = 20130000 \in \mathcal{LR}_{3,5}$
    \end{itemize}
    \begin{tikzpicture}[scale=1]
    \node at (3,0) {\large $\mathbf{\sigma (P)}$};
    \node [label = above : {$1$}] (1)
        at (4,5) {};
    \node [label = right : {$2$}] (2)
        at (5,3) {};
    \node [label = below : {$3$}] (3)
        at (4,1) {};
    \node [label = below : {$4$}] (4)
        at (2,1) {};
    \node [label = left : {$5$}]  (5)
        at (1,3) {};
    \node [label = above : {$6$}] (6)
        at (2,5) {};
    \draw [dashed][very thick]
    (1) -- (2) -- (3) -- (4)
        -- (5) -- (6) -- (1);
    \fill [color = orange] (4,5) -- (4,1)
        -- (2,5) -- cycle;
    \draw [color = violet][line width = 4pt] 
        (5,3) -- (2,1);
    \fill [color=green] (1,3) circle (0.2);
    
  \end{tikzpicture}
\end{frame}

\begin{frame} %24
    \frametitle{2.1) Fonctions de parking rationnelles}
    \only<1>{
    \begin{itemize}
        \item \textbf{Fonction de Parking rationnelle primitive} :
        $\mathcal{PF'}_{a,b} = \{(u_1, \ldots, u_a)$ | $1 \leqslant u_i
        \leqslant 1 + \frac{b}{a}(i-1)$ pour tout $i$ entre $1$ et $a$,
        et $u_i \leqslant \ldots \leqslant u_a\}$
        \begin{itemize}
            \item Exemple : $(1, 1, 1, 2, 2, 2, 3) \in \mathcal{PF'}_{7,3}$
            \item Contre-exemple : $(1, 1, \color{blue}{2} \color{black}, 2,
                 2, 2, 3) \not \in \mathcal{PF'}_{7,3}$
            \item Exemple : $(1, 2, 4) \in \mathcal{PF'}_{3,5}$
        \end{itemize}
    \end{itemize}

    \begin{thm} [Taille de $\mathcal{PF'}_{a,b}$]
        $\displaystyle pf'_{a,b} = |\mathcal{PF'}_{a,b}|
            = Cat(a,b) = \frac{1}{a+b}\binom{a+b}{a}$
    \end{thm}

    }\only<2>{
    \begin{itemize}
        \item \textbf{Fonction de Parking rationnelle} : $\mathcal{PF}_{a,b}
            = \{(u_1, \ldots, u_a)$ dont le tri croissant $(v_1, \ldots,
            v_a) \in \mathcal{PF'}_{a,b}\}$
        \begin{itemize}
            \item Exemple : $(3, 1, 2, 2, 1, 2, 1) \in \mathcal{PF}_{7,3}$
            \item Exemple : $(4, 1, 2) \in \mathcal{PF}_{3,5}$
        \end{itemize}
    \end{itemize}

    \begin{thm} [Taille de $\mathcal{PF}_{a,b}$]
        $\displaystyle pf_{a,b} = |\mathcal{PF}_{a,b}| = b^{a-1}$
    \end{thm}
    }
\end{frame}

\begin{frame} %25
    \frametitle{2) Relations et bijections}
    On utilise ici les mêmes relations et bijections que dans le cas
    classique.
\end{frame}

\subsection{Posets rationnels primitifs}

\begin{frame} %26
    \frametitle{2.2) Posets \textbf{bijectifs} pour $\mathcal{R}_{5,3}$ et
        $\mathcal{PF'}_{5,3}$}
        \begin{columns}
            \begin{column}{0.35\textwidth}
                \begin{center}
\begin{tikzpicture}[scale = 0.18]
    \draw [ultra thick, color = cyan] (0,0) -- (0,1)
        -- (0,2) -- (0,3) -- (0,4) -- (0,5) -- (1,5)
        -- (2,5) -- (3,5);

    \draw [ultra thick, color = cyan] (0,9) -- (0,10)
        -- (0,11) -- (0,12) -- (0,13) -- (1,13) -- (1,14)
        -- (2,14) -- (3,14);
        
    \draw [ultra thick, color = cyan] (-4,18) -- (-4,19)
        -- (-4,20) -- (-4,21) -- (-3,21) -- (-3,22)
        -- (-3,23) -- (-2,23) -- (-1,23);

    \draw [ultra thick, color = cyan] (4,18) -- (4,19)
        -- (4,20) -- (4,21) -- (4,22) -- (5,22) -- (6,22)
        -- (6,23) -- (7,23);

    \draw [ultra thick, color = cyan] (-4,27) -- (-4,28)
        -- (-4,29) -- (-3,29) -- (-3,30) -- (-3,31)
        -- (-3,32) -- (-2,32) -- (-1,32);

    \draw [ultra thick, color = cyan] (4,27) -- (4,28)
        -- (4,29) -- (4,30) -- (5,30) -- (5,31) -- (6,31)
        -- (6,32) -- (7,32);

    \draw [ultra thick, color = cyan] (0,36) -- (0,37)
        -- (0,38) -- (1,38) -- (1,39) -- (1,40) -- (2,40)
        -- (2,41) -- (3,41);

    \draw [->][out=-90,in=90, ultra thick] 
        [color=magenta](1.2,35.5) to (-2.5,32.5);
    \draw [->][color=magenta, ultra thick]
        (-2.5,26.5) to (-2.5,23.5);
    \draw [->][color=magenta, ultra thick]
        (-2.5,17.5) to (-1,14.5);        

    \draw [->][out=-90,in=90, ultra thick] 
        [color=green](1.5,8.5) to (1.5,5.5);

    \draw [->][out=-90,in=90, ultra thick]
        [color=violet](1.8,35.5) to (5.5,32.5);
    \draw [->][color=violet, ultra thick]
        (5.5,26.5) to (5.5,23.5);
    \draw [->][color=violet, ultra thick]
        (5.5,26.5) to (-1.7,23.7);
    \draw [->][color=violet, ultra thick]
        (5.5,17.5) to (4,14.5);

\end{tikzpicture}
\end{center}
            \end{column}
            \begin{column}{0.65\textwidth}
                \input{fig/fig12.tex}
            \end{column}
        \end{columns}
\end{frame}

\begin{frame} %27
    \frametitle{2.2) Posets \textbf{bijectifs} pour $\mathcal{R}_{3,7}$ et
        $\mathcal{PF'}_{3,7}$}
        \begin{columns}
            \begin{column}{0.5\textwidth}
                \begin{center}
    \begin{tikzpicture}[scale=0.7]
        \node (a) at (0, 0) {};
        \node (b) at (0, 4) {};
        \node (c) at (6, 0) {};
        \node (d) at (5.5, 3.3) {};
        \node (e) at (6, 3) [color = magenta]
            {$y = \frac{3}{5}x$}; 
        \draw [dashed, very thick, ->] (a) to (b);
        \draw [dashed, very thick, ->] (a) to (c);
        \draw [dashed, very thick, ->]
            [color = magenta] (a) to (d);

        \node (1)  at (0,0)   {};
        \node (2)  at (0,1)   {};
        \node (3)  at (1,1)   {};
        \node (4)  at (1,2)   {};
        \node (5)  at (2,2)   {};
        \node (6)  at (3,2)   {};
        \node (7)  at (3,3)   {};
        \node (8)  at (4,3)   {};
        \node (9)  at (5,3)   {};
        \draw [->, ultra thick, color = cyan]
            (1)  to (2);
        \draw [->, ultra thick, color = cyan] 
            (2)  to (3);
        \draw [->, ultra thick, color = cyan]
            (3)  to (4);
        \draw [->, ultra thick, color = cyan]
            (4)  to (5);
        \draw [->, ultra thick, color = cyan]
            (5)  to (6);
        \draw [->, ultra thick, color = cyan]
            (6)  to (7);
        \draw [->, ultra thick, color = cyan]
            (7)  to (8);
        \draw [->, ultra thick, color = cyan]
            (8)  to (9);

        \node at (-0.2, -0.2) {$0$};
        \node at (-0.3, 1)    {$1$};
        \node at (1, -0.3)    {$1$};
        \node at (-0.3, 2)    {$2$};
        \node at (2, -0.3)    {$2$};
        \node at (-0.3, 3)    {$3$};
        \node at (3, -0.3)    {$3$};
        \node at (4, -0.3)    {$4$};
        \node at (5, -0.3)    {$5$};

    \end{tikzpicture}
\end{center}
            \end{column}
            \begin{column}{0.5\textwidth}
                \begin{tikzpicture}[scale = 0.7]
    \node (a) at (0, 0) {};
    \node (b) at (0, 8) {};
    \node (c) at (3, 0) {};
    \node (d) at (2.2, 7.7) {};
    \node (e) at (3, 7) [color = magenta]
        {$y = \frac{7}{2}x$}; 
    \draw [dashed, very thick, ->] (a) to (b);
    \draw [dashed, very thick, ->] (a) to (c);
    \draw [dashed, very thick, ->]
        [color = magenta] (a) to (d);

    \node (1)  at (0,0)   {};
    \node (2)  at (0,1)   {};
    \node (3)  at (0,2)   {};
    \node (4)  at (0,3)   {};
    \node (5)  at (0,4)   {};
    \node (6)  at (1,4)   {};
    \node (7)  at (1,5)   {};
    \node (8)  at (1,6)   {};
    \node (9)  at (1,7)   {};
    \node (10) at (2,7)   {};
    \draw [->, ultra thick, color = cyan]
        (1)  to (2);
    \draw [->, ultra thick, color = cyan] 
        (2)  to (3);
    \draw [->, ultra thick, color = cyan]
        (3)  to (4);
    \draw [->, ultra thick, color = cyan]
        (4)  to (5);
    \draw [->, ultra thick, color = cyan]
        (5)  to (6);
    \draw [->, ultra thick, color = cyan]
        (6)  to (7);
    \draw [->, ultra thick, color = cyan]
        (7)  to (8);
    \draw [->, ultra thick, color = cyan]
        (8)  to (9);
    \draw [->, ultra thick, color = cyan]
        (9)  to (10);

    \node at (-0.2, -0.2) {$0$};
    \node at (-0.3, 1)    {$1$};
    \node at (1, -0.3)    {$1$};
    \node at (-0.3, 2)    {$2$};
    \node at (2, -0.3)    {$2$};
    \node at (-0.3, 3)    {$3$};
    \node at (-0.3, 4)    {$4$};
    \node at (-0.3, 5)    {$5$};
    \node at (-0.3, 6)    {$6$};
    \node at (-0.3, 7)    {$7$};
    \node at (1, -1)      {$111101110$};

\end{tikzpicture}
\begin{tikzpicture}[scale = 0.7]
    \node (a) at (0, 0) {};
    \node (b) at (0, 8) {};
    \node (c) at (3, 0) {};
    \node (d) at (2.2, 7.7) {};
    \node (e) at (3, 7) [color = magenta]
        {$y = \frac{7}{2}x$}; 
    \draw [dashed, very thick, ->] (a) to (b);
    \draw [dashed, very thick, ->] (a) to (c);
    \draw [dashed, very thick, ->]
        [color = magenta] (a) to (d);

    \node (1)  at (0,0)   {};
    \node (2)  at (0,1)   {};
    \node (3)  at (0,2)   {};
    \node (4)  at (0,3)   {};
    \node (5)  at (0,4)   {};
    \node (6)  at (0,5)   {};
    \node (7)  at (1,5)   {};
    \node (8)  at (1,6)   {};
    \node (9)  at (1,7)   {};
    \node (10) at (2,7)   {};
    \draw [->, ultra thick, color = cyan]
        (1)  to (2);
    \draw [->, ultra thick, color = cyan] 
        (2)  to (3);
    \draw [->, ultra thick, color = cyan]
        (3)  to (4);
    \draw [->, ultra thick, color = cyan]
        (4)  to (5);
    \draw [->, ultra thick, color = cyan]
        (5)  to (6);
    \draw [->, ultra thick, color = cyan]
        (6)  to (7);
    \draw [->, ultra thick, color = cyan]
        (7)  to (8);
    \draw [->, ultra thick, color = cyan]
        (8)  to (9);
    \draw [->, ultra thick, color = cyan]
        (9)  to (10);

    \node at (-0.2, -0.2) {$0$};
    \node at (-0.3, 1)    {$1$};
    \node at (1, -0.3)    {$1$};
    \node at (-0.3, 2)    {$2$};
    \node at (2, -0.3)    {$2$};
    \node at (-0.3, 3)    {$3$};
    \node at (-0.3, 4)    {$4$};
    \node at (-0.3, 5)    {$5$};
    \node at (-0.3, 6)    {$6$};
    \node at (-0.3, 7)    {$7$};
    \node at (1, -1)      {$111110110$};

\end{tikzpicture}
\begin{tikzpicture}[scale = 0.7]
    \node (a) at (0, 0) {};
    \node (b) at (0, 8) {};
    \node (c) at (3, 0) {};
    \node (d) at (2.2, 7.7) {};
    \node (e) at (3, 7) [color = magenta]
        {$y = \frac{7}{2}x$}; 
    \draw [dashed, very thick, ->] (a) to (b);
    \draw [dashed, very thick, ->] (a) to (c);
    \draw [dashed, very thick, ->]
        [color = magenta] (a) to (d);

    \node (1)  at (0,0)   {};
    \node (2)  at (0,1)   {};
    \node (3)  at (0,2)   {};
    \node (4)  at (0,3)   {};
    \node (5)  at (0,4)   {};
    \node (6)  at (0,5)   {};
    \node (7)  at (0,6)   {};
    \node (8)  at (1,6)   {};
    \node (9)  at (1,7)   {};
    \node (10) at (2,7)   {};
    \draw [->, ultra thick, color = cyan]
        (1)  to (2);
    \draw [->, ultra thick, color = cyan] 
        (2)  to (3);
    \draw [->, ultra thick, color = cyan]
        (3)  to (4);
    \draw [->, ultra thick, color = cyan]
        (4)  to (5);
    \draw [->, ultra thick, color = cyan]
        (5)  to (6);
    \draw [->, ultra thick, color = cyan]
        (6)  to (7);
    \draw [->, ultra thick, color = cyan]
        (7)  to (8);
    \draw [->, ultra thick, color = cyan]
        (8)  to (9);
    \draw [->, ultra thick, color = cyan]
        (9)  to (10);

    \node at (-0.2, -0.2) {$0$};
    \node at (-0.3, 1)    {$1$};
    \node at (1, -0.3)    {$1$};
    \node at (-0.3, 2)    {$2$};
    \node at (2, -0.3)    {$2$};
    \node at (-0.3, 3)    {$3$};
    \node at (-0.3, 4)    {$4$};
    \node at (-0.3, 5)    {$5$};
    \node at (-0.3, 6)    {$6$};
    \node at (-0.3, 7)    {$7$};
    \node at (1, -1)      {$111111010$};

\end{tikzpicture}
\begin{tikzpicture}[scale = 0.7]
    \node (a) at (0, 0) {};
    \node (b) at (0, 8) {};
    \node (c) at (3, 0) {};
    \node (d) at (2.2, 7.7) {};
    \node (e) at (3, 7) [color = magenta]
        {$y = \frac{7}{2}x$}; 
    \draw [dashed, very thick, ->] (a) to (b);
    \draw [dashed, very thick, ->] (a) to (c);
    \draw [dashed, very thick, ->]
        [color = magenta] (a) to (d);

    \node (1)  at (0,0)   {};
    \node (2)  at (0,1)   {};
    \node (3)  at (0,2)   {};
    \node (4)  at (0,3)   {};
    \node (5)  at (0,4)   {};
    \node (6)  at (0,5)   {};
    \node (7)  at (0,6)   {};
    \node (8)  at (0,7)   {};
    \node (9)  at (1,7)   {};
    \node (10) at (2,7)   {};
    \draw [->, ultra thick, color = cyan]
        (1)  to (2);
    \draw [->, ultra thick, color = cyan] 
        (2)  to (3);
    \draw [->, ultra thick, color = cyan]
        (3)  to (4);
    \draw [->, ultra thick, color = cyan]
        (4)  to (5);
    \draw [->, ultra thick, color = cyan]
        (5)  to (6);
    \draw [->, ultra thick, color = cyan]
        (6)  to (7);
    \draw [->, ultra thick, color = cyan]
        (7)  to (8);
    \draw [->, ultra thick, color = cyan]
        (8)  to (9);
    \draw [->, ultra thick, color = cyan]
        (9)  to (10);

    \node at (-0.2, -0.2) {$0$};
    \node at (-0.3, 1)    {$1$};
    \node at (1, -0.3)    {$1$};
    \node at (-0.3, 2)    {$2$};
    \node at (2, -0.3)    {$2$};
    \node at (-0.3, 3)    {$3$};
    \node at (-0.3, 4)    {$4$};
    \node at (-0.3, 5)    {$5$};
    \node at (-0.3, 6)    {$6$};
    \node at (-0.3, 7)    {$7$};
    \node at (1, -1)      {$111111100$};

\end{tikzpicture}
            \end{column}
        \end{columns}
\end{frame}

\subsection{Posets rationnels non-primitifs}

\begin{frame} %28
    \frametitle{2.3) Posets \textbf{bijectifs} pour $\mathcal{LR}_{5,2}$ et
        $\mathcal{PF}_{5,2}$}
        \begin{tikzpicture}[scale=1]
    \node at (3,0) {\large $\mathbf{K(P)}$};
    \node [label = above : {$1$}] (1)
        at (4,5) {};
    \node [label = right : {$2$}] (2)
        at (5,3) {};
    \node [label = below : {$3$}] (3)
        at (4,1) {};
    \node [label = below : {$4$}] (4)
        at (2,1) {};
    \node [label = left : {$5$}]  (5)
        at (1,3) {};
    \node [label = above : {$6$}] (6)
        at (2,5) {};
    \draw [dashed][very thick]
    (1) -- (2) -- (3) -- (4)
        -- (5) -- (6) -- (1);
    \draw [color = orange][line width = 4pt] 
        (4,5) -- (1,3);            
    \draw [color = violet][line width = 4pt] 
        (4,1) -- (2,1);
    \fill [color=green] (5,3) circle (0.2); 
    \fill [color=green] (2,5) circle (0.2);   
  \end{tikzpicture}
\end{frame}

\begin{frame} %29
    \frametitle{2.3) Posets \textbf{bijectifs} pour $\mathcal{LR}_{5,2}$ et
        $\mathcal{PF}_{5,2}$}
        \begin{center}
\begin{tikzpicture}[scale = 0.22]
    \node at (0,0) {$(1,1,1,1,1)$};

    \node at (-18,5) {$(1,1,1,1,2)$};
    \node at (-9,5)  {$(1,1,1,2,1)$};
    \node at (0,5)   {$(1,1,2,1,1)$};
    \node at (9,5)   {$(1,2,1,1,1)$};
    \node at (18,5)  {$(2,1,1,1,1)$};

    \node at (-23,10) {$(1,1,1,2,2)$};
    \node at (-18,13) {$(1,1,2,1,2)$};
    \node at (-13,10) {$(1,1,2,2,1)$};
    \node at (-8,13)  {$(1,2,1,1,2)$};
    \node at (-3,10)  {$(1,2,1,2,1)$};
    \node at (3,13)   {$(1,2,2,1,1)$};
    \node at (8,10)   {$(2,1,1,1,2)$};
    \node at (13,13)  {$(2,1,1,2,1)$};
    \node at (18,10)  {$(2,1,2,1,1)$};
    \node at (23,13)  {$(2,2,1,1,1)$};

    \draw [->][color=magenta, ultra thick]
        (-23,9) to (-24,8);
    \draw [->][color=magenta, ultra thick]
        (-18,12) to (-19,11);
    \draw [->][color=magenta, ultra thick]
        (-8,12) to (-9,11);
    \draw [->][color=magenta, ultra thick]
        (8,9) to (7,8);
    \draw[color=magenta, ultra thick]
        (-18.5,7.2) -- (-17.5,6.2);
    \draw[color=magenta, ultra thick]
        (-18.5,6.2) -- (-17.5,7.2);
    \draw [->][out=-90,in=150, ultra thick] 
        [color=magenta](-18,4) to (-2,1);

    \draw [->][color=brown!7!orange, ultra thick]
        (-23,9) to (-22,8);
    \draw [->][color=brown!7!orange, ultra thick]
        (-13,9) to (-14,8);
    \draw [->][color=brown!7!orange, ultra thick]
        (-3,9) to (-4,8);
    \draw [->][color=brown!7!orange, ultra thick]
        (13,12) to (12,11);
    \draw[color=brown!7!orange, ultra thick]
        (-9.5,7.2) -- (-8.5,6.2);
    \draw[color=brown!7!orange, ultra thick]
        (-9.5,6.2) -- (-8.5,7.2);
    \draw [->][out=-90,in=90, ultra thick]
        [color=brown!7!orange](-9,4) to (-1,1);

    \draw [->][color=yellow, ultra thick]
        (-18,12) to (-17,11); 
    \draw [->][color=yellow, ultra thick]
        (-13,9) to (-12,8); 
    \draw [->][color=yellow, ultra thick]
        (3,12) to (2,11); 
    \draw [->][color=yellow, ultra thick]
        (18,9) to (17,8); 
    \draw[color=yellow, ultra thick]
        (-0.5,7.2) -- (0.5,6.2);
    \draw[color=yellow, ultra thick]
        (-0.5,6.2) -- (0.5,7.2);
    \draw [->][out=-90,in=90, ultra thick] 
        [color=yellow](0,4) to (0,1);

    \draw [->][color = green,  ultra thick]
        (-8,12) to (-7,11);
    \draw [->][color=green, ultra thick]
        (-3,9) to (-2,8);
    \draw [->][color=green, ultra thick]
        (3,12) to (4,11);
    \draw [->][color=green, ultra thick]
        (23,12) to (22,11);
    \draw[color=green, ultra thick]
        (8.5,7.2) -- (9.5,6.2);
    \draw[color=green, ultra thick]
        (8.5,6.2) -- (9.5,7.2);
    \draw [->][out=-90,in=90, ultra thick]
        [color=green](9,4) to (1,1);

    \draw [->][color=violet, ultra thick]
        (8,9) to (9,8);
    \draw [->][color=violet, ultra thick]
        (13,12) to (14,11);
    \draw [->][color=violet, ultra thick]
        (18,9) to (19,8);
    \draw [->][color=violet, ultra thick]
        (23,12) to (24,11);
    \draw[color=violet, ultra thick]
        (17.5,7.2) -- (18.5,6.2);
    \draw[color=violet, ultra thick]
        (17.5,6.2) -- (18.5,7.2);
    \draw [->][out=-90,in=30, ultra thick] 
        [color=violet](18,4) to (2,1);

\end{tikzpicture}
\end{center}
\end{frame}

\begin{frame} %30
    \frametitle{2.3) Posets \textbf{bijectifs} pour $\mathcal{LR}_{2,7}$ et
        $\mathcal{PF}_{2,7}$}
        \begin{columns}
            \begin{column}{0.5\textwidth}
                \begin{tikzpicture}[scale = 0.75]
    \node (0)  at (0,0) [align = center]
    [rectangle, draw]
        {$123$\\$123$};
    \node (1)  at (-8,6)[align = center]
    [rectangle, draw]
        {$1|23$\\$123$};
    \node (2)  at (-6,6) [align = center]
    [rectangle, draw]
        {$13|2$\\$123$};
    \node (3)  at (-4,6) [align = center]
    [rectangle, draw]
        {$12|3$\\$123$};
    \node (4)  at (-2,6) [align = center]
    [rectangle, draw]
        {$13|2$\\$132$};
    \node (5)  at (0,6) [align = center]
    [rectangle, draw]
        {$12|3$\\$132$};
    \node (6)  at (2,6) [align = center]
    [rectangle, draw]
        {$1|23$\\$213$};
    \node (7)  at (4,6) [align = center]
    [rectangle, draw]
        {$13|2$\\$213$};
    \node (8)  at (6,6) [align = center]
    [rectangle, draw]
        {$12|3$\\$231$};
    \node (9)  at (8,6) [align = center]
    [rectangle, draw]
        {$1|23$\\$312$};
    \node (10) at (-5,12) [align = center]
    [rectangle, draw]
        {$1|2|3$\\$123$};
    \node (11) at (-3,12) [align = center]
    [rectangle, draw]
        {$1|2|3$\\$132$};
    \node (12) at (-1,12) [align = center]
    [rectangle, draw]
        {$1|2|3$\\$213$};
    \node (13) at (1,12) [align = center]
    [rectangle, draw]
        {$1|2|3$\\$231$};
    \node (14) at (3,12) [align = center]
    [rectangle, draw]
        {$1|2|3$\\$312$};
    \node (15) at (5,12) [align = center]
    [rectangle, draw]
        {$1|2|3$\\$321$};

    \draw [->][color=magenta, ultra thick]
        (-5,11) to (-8.2,7);
    \draw [->][color=magenta, ultra thick]
        (-5,11) to (-6.2, 7); 
    \draw [->][color=magenta, ultra thick]
        (-5,11) to (-4.2,7);
    \draw [->][out=-90,in=90, ultra thick] 
        [color=magenta](-8,5) to (-0.5,1);
    \draw [->][out=-90,in=90, ultra thick] 
        [color=magenta](-6,5) to (-0.5,1);

    \draw [->][color=yellow, ultra thick]
        (-3,11) to (-7.8,7);
    \draw [->][color=yellow, ultra thick]
        (-3,11) to (-2.2,7);
    \draw [->][color=yellow, ultra thick]
        (-3,11) to (-0.2, 7);
    \draw [->][out=-90,in=90, ultra thick] 
        [color=yellow](-2,5) to (-0.1,1);
    \draw [->][out=-90,in=90, ultra thick] 
        [color=yellow](0,5) to (-0.1,1);
    
    \draw [->][color=orange, ultra thick]
        (-1,11) to (1.8,7);
    \draw [->][color=orange, ultra thick]
        (-1,11) to (3.8,7);
    \draw [->][color=orange, ultra thick]
        (-1,11) to (-3.8,7);
    \draw [->][out=-90,in=90, ultra thick] 
        [color=orange](-4,5) to (-0.3,1);

    \draw [->][color=green, ultra thick]
        (1,11) to (2.2,7);
    \draw [->][color=green, ultra thick]
        (1,11) to (-1.8,7);
    \draw [->][color=green, ultra thick]
        (1,11) to (5.8,7);
    \draw [->][out=-90,in=90, ultra thick] 
        [color=green](2,5) to (0.1,1);

    \draw [->][color=cyan, ultra thick]
        (3,11) to (7.8,7);
    \draw [->][color=cyan, ultra thick]
        (3,11) to (4.2,7);
    \draw [->][color=cyan, ultra thick]
        (3,11) to (0.2,7);
    \draw [->][out=-90,in=90, ultra thick] 
        [color=cyan](4,5) to (0.3,1);

    \draw [->][color=violet, ultra thick]
        (5,11) to (8.2,7);
    \draw [->][color=violet, ultra thick]
        (5,11) to (-5.8,7);
    \draw [->][color=violet, ultra thick]
        (5,11) to (6.2,7);
    \draw [->][out=-90,in=90, ultra thick] 
        [color=violet](6,5) to (0.5,1);
    \draw [->][out=-90,in=90, ultra thick] 
        [color=violet](8,5) to (0.5,1);
    
\end{tikzpicture}
            \end{column}
            \begin{column}{0.5\textwidth}
                \begin{center}
\begin{tikzpicture}[scale = 0.2]
    \node at (0,0) {$(1,1)$};

    \node at (-6,5) {$(1,2)$};
    \node at (6,5)  {$(2,1)$};

    \node at (-6,10) {$(1,3)$};
    \node at (6,10)  {$(3,1)$};

    \node at (-6,15) {$(1,4)$};
    \node at (6,15)  {$(4,1)$};

    \draw [->][color=magenta, ultra thick]
        (-6,14) to (-6,11);
    \draw [->][color=magenta, ultra thick]
        (-6,9) to (-6,6);
    \draw [->][out=-90,in=90, ultra thick] 
        [color=magenta](-6,4) to (-0.5,1);

    \draw [->][color=green, ultra thick]
        (6,14) to (6,11);
    \draw [->][color=green, ultra thick]
        (6,9) to (6,6);
    \draw [->][out=-90,in=90, ultra thick] 
        [color=green](6,4) to (0.5,1);

\end{tikzpicture}
\end{center}
            \end{column}
        \end{columns}
\end{frame}

\section{Conclusion} %31

\begin{frame} %32
    \frametitle{3) Résumé}
    \begin{itemize}
        \item Relations de couvertures créant des
        posets pour les quatre types de fonctions de parking et
        les quatre types de chemins de Dyck mis en bijections
        avec ces dernières.
        \item Dans chaque cas, le poset de l'un peut être obtenu en appliquant la
        bijection présentée au poset de l'autre.
        \item Cas rationnel : marche pour $a > b$ et $a < b$.
        \item La plupart des travaux précédents se basaient sur les
            \emph{partitions non-croisées}, et ne donnaient pas de relation de
            couverture évidente pour les fonctions de parking.
                
        \item Une autre partie de nos travaux, non-incluse dans cette
            présentation, a été d'étendre la notion d'\emph{arbre de parking}
            au cas rationnel, et d'encoder les notions présentées.
    \end{itemize}
\end{frame}

\begin{frame} %33
    \frametitle{3) Travaux futurs}
    \begin{itemize}
        \item Preuve pour la Conjecture Principale.
        \item Formules généralisées pour le nombre d'intervalles des posets
            dans le cas rationnel.
        \item Relation de couverture pour les arbres de parking rationnels.
    \end{itemize}
\end{frame}

\begin{frame} %34
    \frametitle{3) Bibliographie}
    \small
    \bibliographystyle{unsrt}
    \bibliography{ref}
    \nocite{*}
\end{frame}

\end{document}