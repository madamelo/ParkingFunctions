\documentclass[12pt]{article}

\usepackage{amsmath}
\usepackage{amssymb}
\usepackage{amsthm}

\usepackage{graphicx}
\graphicspath{ {./} }

\usepackage[utf8]{inputenc}
\usepackage[T1]{fontenc}

\usepackage[inline]{enumitem}

\newtheorem{theorem}{Theorem}
\newtheorem{lemma}{Lemma}
\newtheorem*{prop}{Proposition}
\newtheorem{definition}{Definition}
\newtheorem*{example}{Example}
\newtheorem*{notation}{Notation}
\newtheorem*{cor}{Corollary}

\begin{document}

\title{Rational Parking Functions}
\author{Matthieu Josuat-Vergès \and Tessa Lelièvre-Osswald}
\date{August 27, 2020}

\maketitle

\begin{abstract}
    This is an abstract about Rational Parking Functions
\end{abstract}

\begin{displaymath}
\end{displaymath}

\section{Parking Functions}

\begin{definition}[Parking Function]
    A \emph{parking function} is a sequence $(a_1, a_2, \ldots, a_n)$
    such that its non-decreasing reordering $(b_1, b_2, \ldots, b_n)$
    has $b_i < i$ for all $i$.\\
    We denote by $\mathcal{PF}_n$ the set of parking functions of length $n$.
    $$\mathcal{PF} = \bigcup_{n > 0}{\mathcal{PF}_n}$$.
\end{definition}

\begin{example}
    \begin{align*}
    &f_1 = (7, 3, 1, 4, 2, 5, 2) \in \mathcal{PF}_7\\
    &f_2 = (7, 3, 1, 4, 2, 5, 4) \notin \mathcal{PF}_7\\
    \end{align*}
\end{example}

\begin{theorem}
    Let $pf_n$ be the cardinal of $\mathcal{PF}_n$.
    We have $pf_n = (n + 1)^{n-1}$.
\end{theorem}

\begin{example}[$n = 1, 2, 3$]
    \text{} \\
    \begin{itemize*}
            \item $n = 1$ \text{ } $:$ \text{ } $pf_1 = 1$\\
            \subitem $(1)$\\
            \item $n = 2$ \text{ } $:$ \text{ } $pf_2 = 3$\\
            \subitem $(1, 1)$
            \subitem $(1, 2)$
            \subitem $(2, 1)$\\
            \item $n = 3$ \text{ } $:$ \text{ } $pf_3 = 16$\\
            \subitem $(1, 1, 1)$
            \subitem $(1, 1, 2)$
            \subitem $(1, 1, 3)$
            \subitem $(1, 2, 1)$
            \subitem $(1, 2, 2)$
            \subitem $(1, 2, 3)$
            \subitem $(1, 3, 1)$
            \subitem $(1, 3, 2)$
            \subitem $(2, 1, 1)$
            \subitem $(2, 1, 2)$
            \subitem $(2, 1, 3)$
            \subitem $(2, 2, 1)$
            \subitem $(2, 3, 1)$
            \subitem $(3, 1, 1)$
            \subitem $(3, 1, 2)$
            \subitem $(3, 2, 1)$\\
    \end{itemize*}
\end{example}

\begin{definition}[Primitive]
    A parking function $(a_1, a_2, \ldots, a_n)$ is said \emph{primitive} if
    it is already in non-decreasing order. \\
    We denote by $\mathcal{PF'}_n$ the set of primitive parking functions of length $n$.
    $$\mathcal{PF'} = \bigcup_{n > 0}{\mathcal{PF'}_n}$$
    
\end{definition}

\begin{example}
    \begin{align*}
        &f_1 = (1, 2, 2, 3) \in \mathcal{PF'}_4\\
        &f_2 = (1, 2, 3, 2) \notin \mathcal{PF'}_4
         \text{, even though } f_2 \in \mathcal{PF}_4\\
    \end{align*}
\end{example}

\begin{theorem}
    Let $pf'_n$ be the cardinal of $\mathcal{PF'}_n$.
    We have $pf'_n = \frac{1}{n + 1} \binom{2n}{n}$,
    which is the $n^{th}$ Catalan number.
\end{theorem}

\begin{example}[$n = 1, 2, 3$]
    \text{}\\
    \begin{itemize*}
        \item $n = 1$ \text{ } $:$ \text{ } $pf'_1 = 1$\\
        \subitem $(1)$\\
        \item $n = 2$ \text{ } $:$ \text{ } $pf'_2 = 2$\\
        \subitem $(1, 1)$
        \subitem $(1, 2)$\\
        \item $n = 3$ \text{ } $:$ \text{ } $pf'_3 = 5$\\
        \subitem $(1, 1, 1)$
        \subitem $(1, 1, 2)$
        \subitem $(1, 1, 3)$
        \subitem $(1, 2, 2)$
        \subitem $(1, 2, 3)$\\
    \end{itemize*}
\end{example}

\section{Non-crossing Partitions}

\begin{definition}[Non-crossing Partition]
    A \emph{non-crossing partition} of a set $E$ is
    a set partition $P = \{E_1, E_2, \ldots, E_k\}$ such that
    if $a, c \in E_i$, $b, d \in E_j$, and $i \neq j$, then
    we do \emph{not} have $a < b < c < d$, nor $a > b > c > d$.\\
    We denote by $\mathcal{NC}_n$ the set of non-crossing partitions
    of $\{1, 2, \ldots, n\}$.
    $$\mathcal{NC} = \bigcup_{n > 0}{\mathcal{NC}_n}$$
\end{definition}

\begin{notation}
    $[n] = \{1, 2, \ldots, n\}$
\end{notation}

\begin{example}[$E = \lbrack 6 \rbrack $]
    \begin{align*}
        &P_1 = \{\{1, 2, 5\}, \{3, 4\}, \{6\}\} \in \mathcal{NC}_6\\
        &P_2 = \{\{1, 2, 4\}, \{3, 5\}, \{6\}\} \notin \mathcal{NC}_6
    \end{align*}
\end{example}

\begin{theorem}
    Let $nc_n$ be the cardinal of $\mathcal{NC}_n$.
    We have $nc_n = \frac{1}{n + 1} \binom{2n}{n}$,
    which is the $n^{th}$ Catalan number.
\end{theorem}

\begin{example}[$n = 1, 2, 3$]
    \text{}\\
    \begin{itemize*}
        \item $n = 1$ \text{ } $:$ \text{ } $nc_1 = 1$\\
        \subitem $\{\{1\}\}$\\
        \item $n = 2$ \text{ } $:$ \text{ } $nc_2 = 2$\\
        \subitem $\{\{1, 2\}\}$
        \subitem $\{\{1\}, \{2\}\}$\\
        \item $n = 3$ \text{ } $:$ \text{ } $nc_3 = 5$\\
        \subitem $\{\{1, 2, 3\}\}$
        \subitem $\{\{1\}, \{2, 3\}\}$
        \subitem $\{\{1, 3\}, \{2\}\}$
        \subitem $\{\{1, 2\}, \{3\}\}$
        \subitem $\{\{1\}, \{2\}, \{3\}\}$\\
    \end{itemize*}
\end{example}

\begin{prop}
    This means we can create a \emph{bijection} between
    $\mathcal{PF'}_n$ and $\mathcal{NC}_n$.
\end{prop}

\begin{itemize}
    \item $\mathcal{NC}_n \to \mathcal{PF'}_n$ :
    For each block $B$ in the non-crossing partition, take
    $i = min (B)$, and $k_i = size (B)$.\\
    $k_i = 0$ if $i$ is not the minimum of a block.\\
    The corresponding parking function is
    $(\underbrace{1, \ldots, 1}_{k_1}, \underbrace{2, \ldots,
    2}_{k_2}, \ldots, \underbrace{n, \ldots, n}_{k_n})$.\\
    \item $\mathcal{PF'}_n \to \mathcal{NC}_n$ :
    For each $i$ in $[n]$, if $i$ appears $n_i$ times in the
    parking function, $B_i$ will be of size $n_i$ with minimum
    element $i$.
    There is a unique set partition $\displaystyle P = \bigcup_{i}{B_i}$
    of $[n]$ respecting these conditions that is non-crossing.
\end{itemize}

\begin{example}[$E = \lbrack 6 \rbrack$]
    \begin{align*}
        &P = \{\{1, 2, 5\}, \{3, 4\}, \{6\}\}
        &f = (1, 1, 1, 3, 3, 6)\\
    \end{align*}
\end{example}

\begin{cor}
    A non-crossing partition can be represented by the minimums
    and sizes of its blocks.
\end{cor}

\begin{example}
    $\{\{1, 2, 5\}, \{3, 4\}, \{6\}\}$ can be represented by
    the following dictionnary :\\
    \begin{itemize*}
        \item 1 : 3\\
        \item 3 : 2\\
        \item 6 : 1\\
    \end{itemize*}
\end{example}

A non-crossing partition of $[n]$ can be represented graphically
on a regular $n$-vertices polygon, with vertices labeled from $1$
to $n$ clockwise. We then represent each block $B = \{b_1, \ldots, b_k\}$
by the convex hull of $\{b_1, \ldots, b_k\}$.\\

\begin{example}[$P = \{\{1, 2, 5\}, \{3, 4\}, \{6\}\}$]
    \text {}\\
    \begin{align*}
    \includegraphics[scale = 0.3]{fig1}
    \end{align*}
\end{example}

Thus non-crossing meaning the hulls are \emph{disjoint}.\\

\subsection{The non-crossing partitions poset}

\begin{definition}[$\succ$]
    We say that $P$ covers $Q$, written $P \succ Q$,
    if $\exists B_i, B_j \in P$ such that
    $Q = P - \{B_i, B_j\} \cup \{B_i \cup B_j\}$    
\end{definition}

\begin{example}
    $\{\{1, 6\}, \{2, 3\}, \{4, 5\}\} \succ
    \{\{1, 2, 3, 6\}, \{4, 5\}\}$\\
    \begin{itemize*}
        \item $B_i = \{1, 6\}$\\
        \item $B_j = \{2, 3\}$\\
    \end{itemize*}
\end{example}

\begin{prop}
    This covering relation defines the \emph{poset} of
    non-crossing partitions of $[n]$.
    We denote by $\mathcal{NCC}_n$ the set of
    \emph{maximal chains} in the poset of $\mathcal{NC}_n$.\\
    $$\mathcal{NCC} = \bigcup_{n > 0}{\mathcal{NCC}_n}$$
\end{prop}

% todo : figure of the poset of NC_4

\begin{theorem}
    Let $ncc_n$ be the cardinal of $\mathcal{NC}_n$.
    We have $ncc_n = n^{n - 2}$.
\end{theorem}

\begin{example}[Shape of the poset of $\mathcal{NC}_4$]
    \text {}\\
    \begin{align*}
    \includegraphics[scale = 0.8]{fig2}
    \end{align*}
    \begin{center}
        This figure was generated with Sagemath.
        There are $4^2 = 16$ different maximal chains,
        and $\frac {1}{5} \binom{8}{4} = \frac{70}{5} = 14$
        elements in this poset.
    \end{center}
\end{example}

\subsection{Kreweras complement}

\begin{definition}[Associated permutation]
    The \emph{permutation} $\sigma$ associated to a non-crossing
    partition has a cycle $(b_1, \ldots, b_k)$ for each block
    $B = \{b_1, \ldots, b_k\}$ of the partition.
\end{definition}

\begin{example}
    The permutation associated to $\{\{1, 2, 5\}, \{3, 4\}, \{6\}\}$
    is $(1\ 2\ 5)\ (3\ 4)\ (6) = 254316$.
\end{example}

\begin{definition}[Kreweras complement]
    The \emph{Kreweras complement} $K (P)$ of a non-crossing
    partition $P$ is defined as follows :\\
    \begin{itemize*}
        \item Let $\sigma$ be the permutation associated to $P$\\
        \item Let $\pi$ be the permutation $(n\ n-1\ n-2\
        \ldots\ 3\ 2\ 1) = n123 \ldots n-1$\\
        \item $K (P)$ is the \emph{non-crossing partition}
        associated to $\pi \sigma$.\\
    \end{itemize*}
\end{definition}

\begin{example}[$P = \{\{1, 2, 5\}, \{3, 4\}, \{6\}\}$]
    \text {} \\
    \begin{itemize*}
        \item $\sigma = (1\ 2\ 5)\ (3\ 4)\ (6) = 254316$\\
        \item $\pi = (6\ 5\ 4\ 3\ 2\ 1) = 612345$\\
        \item $\pi \sigma = 143265 = (1)\ (2\ 4)\ (3)\ (5\ 6)$\\
        \item $K(P) = \{\{1\},\{2, 4\}, \{3\}, \{5, 6\}\}$\\
    \end{itemize*}
\end{example}

\begin{prop}[Kreweras minimums]
        Let $P = \{B_1, \ldots, B_k\}$ be a non-crossing partition.
        Let $K (P) = \{B'_1, \ldots, B'_l\}$ be its Kreweras complement.
        Then $$\bigcup_{1 \leq i \leq l}{min (B'_i)} =
        B_1 \cup \bigcup_{1 < j \leq k}{B_i - {max (B_i)}}$$\\
\end{prop}

\begin{example}[$P = \{\{1, 2, 5\}, \{3, 4\}, \{6\}\}$]
    \text{}\\
    \begin{itemize}
        \item $K (P) = \{\{1\},\{2, 4\}, \{3\}, \{5, 6\}\}$
        \item $\bigcup{min (B'_i)} = \{1, 2, 3, 5\}$
        \item $B_1\ \cup\ \bigcup{B_i - {max (B_i)}}
        = \{1, 2, 5\} \cup \{3, 4\} - \{4\} \cup \{6\} - \{6\}
        = \{1, 2, 5\} \cup \{3\} \cup \emptyset = \{1, 2, 3, 5\}$\\
    \end{itemize}
\end{example}

\begin{notation}
    $B_{[i]} = $ block containing $i$.
\end{notation}

\begin{prop}[Kreweras block sizes]
    Let $P = \{B_1, \ldots, B_k\}$ be a non-crossing partition.
    Let $K (P) = \{B'_1, \ldots, B'_l\}$ be its Kreweras complement.
    Then the size of the block $B'_i$ is defined as follows :
    \begin{itemize}
        \item Let $m_i$ be the the $i^{th}$ minimum of $K (P)$
        \item Define a \emph{transition} $\phi (e)$ as 
            \subitem Let $j = e + 1$ (or $1$ if $e = n$)
            \subitem $\phi(e) = max (B_{[j]})$
        \item The size of $B'_i$ is $k_{min}$ such that
        $k_{min} = min \{k > 0\ |\ \phi^k (m_i) \in B_{[m_i]}\}$.\\
    \end{itemize}
\end{prop}

\begin{example}[$P = \{\{1, 2, 5\}, \{3, 4\}, \{6\}\}$]
    \text{}
    \begin{itemize}
        \item $mins = \{1, 2, 3, 5\}$
        \item $m_1 = 1$
            \subitem $B_{[1]} = B_1$
            \subitem $max (B_{[2]} = max (B_1) = 5$
            \subitem The size for $m_1$ is $1$.
        \item $m_2$
            \subitem $B_{[2]} = B_1$
            \subitem $max (B_{[3]}) = max (B_2) = 4$
            \subitem $max (B_{[5]}) = max (B_1) = 5$
            \subitem The size for $m_2$ is $2$.
        \item $m_3 = 3$
            \subitem $B_{[3]} = B_2$
            \subitem $max (B_{[4]}) = max (B_2) = 4$
            \subitem The size for $m_3$ is $1$.
        \item $m_4 = 5$
            \subitem $B_{[5]} = B_1$
            \subitem $max (B_{[6]}) = max (B_3) = 6$
            \subitem $max (B_{[1]}) = max (B_1) = 5$
            \subitem The size for $m_4$ is $2$.
    \end{itemize}
\end{example}

\section{Non-crossing 2-partitions}

\begin{definition}[Non-crossing 2-partition]
    A \emph{non-crossing 2-partition} of a set $E$ is a pair $(P, \sigma)$
    where :\\
    \begin{itemize*}
        \item $P$ is a non-crossing partition of $E$\\
        \item $\sigma$ is a permutation of the elements of $E$\\
    \end{itemize*}
    We denote by $\mathcal{NC}^2_n$ the set of non-crossing
    2-partitions of $[n]$.
    $$\mathcal{NC}^2 = \bigcup_{n > 0}{\mathcal{NC}^2_n}$$.
\end{definition}

% TODO : CLARIFY ALLOWED SIGMAS

\begin{example}[$\mathcal{NC}^2_6$]
    \begin{itemize*}
            \subitem $P = \{\{1, 6\}, \{2, 3, 5\}, \{4\}\}$
            \subitem $\sigma = 416235$ 
    \end{itemize*}    
\end{example}

\begin{theorem}
    Let $nc^2_n$ be the cardinal of $\mathcal{NC}^2_n$.
    We have $nc^2_n = (n + 1)^{n-1}$.
\end{theorem}

\begin{example}[$n = 1, 2, 3$]
    \text{}
    \begin{itemize}
            \item $n = 1$ \text{ } $:$ \text{ } $nc^2_1 = 1$
            \subitem $\{\{1\}\}$ \hspace{1cm} $1$
            \item $n = 2$ \text{ } $:$ \text{ } $nc^2_2 = 3$
            \subitem $\{\{1\}, \{2\}\}$ \hspace{1cm} $12$
            \subitem $\{\{1\}, \{2\}\}$ \hspace{1cm} $21$
            \subitem $\{\{1, 2\}\}$ \hspace{14mm} $12$
            \item $n = 3$ \text{ } $:$ \text{ } $nc^2_3 = 16$
            \subitem $\{\{1\}, \{2\}, \{3\}\}$ \hspace{1cm}
                $123$
            \subitem $\{\{1\}, \{2\}, \{3\}\}$ \hspace{1cm}
                $132$            
            \subitem $\{\{1\}, \{2\}, \{3\}\}$ \hspace{1cm}
                $213$
            \subitem $\{\{1\}, \{2\}, \{3\}\}$ \hspace{1cm}
                $231$
            \subitem $\{\{1\}, \{2\}, \{3\}\}$ \hspace{1cm}
                $312$
            \subitem $\{\{1\}, \{2\}, \{3\}\}$ \hspace{1cm}
                $321$            
            \subitem $\{\{1, 2\}, \{3\}\}$ \hspace{14mm}
                $123$            
            \subitem $\{\{1, 2\}, \{3\}\}$ \hspace{14mm}
                $132$
            \subitem $\{\{1, 2\}, \{3\}\}$ \hspace{14mm}
                $231$
            \subitem $\{\{1\}, \{2, 3\}\}$ \hspace{14mm}
                $123$            
            \subitem $\{\{1\}, \{2, 3\}\}$ \hspace{14mm}
                $213$
            \subitem $\{\{1\}, \{2, 3\}\}$ \hspace{14mm}
                $312$
            \subitem $\{\{1, 3\}, \{2\}\}$ \hspace{14mm}
                $123$
            \subitem $\{\{1, 3\}, \{2\}\}$ \hspace{14mm}
                $132$
            \subitem $\{\{1, 3\}, \{2\}\}$ \hspace{14mm}
                $231$            
            \subitem $\{\{1, 2, 3\}\}$ \hspace{18mm}
                $123$\\
    \end{itemize}
\end{example}

\end{document}