\section{Parking Functions}

\begin{definition}[Parking Function]
    A \emph{parking function} is a sequence of positive integers
    $(a_1, a_2, \ldots, a_n)$
    such that its non-decreasing reordering $(b_1, b_2, \ldots, b_n)$
    has $b_i \leqslant i$ for all $i$.\\
    We denote by $\mathcal{PF}_n$ the set of parking functions of length $n$.
    $$\mathcal{PF} = \bigcup_{n > 0}{\mathcal{PF}_n}$$.
\end{definition}

\begin{example}
    \begin{align*}
    &f_1 = (7, 3, 1, 4, 2, 5, 2) \in \mathcal{PF}_7\\
    &f_2 = (7, 3, 1, 4, 2, 5, 4) \notin \mathcal{PF}_7\\
    \end{align*}
\end{example}

\begin{theorem}
    Let $pf_n$ be the cardinal of $\mathcal{PF}_n$.
    We have $$pf_n = (n + 1)^{n-1}$$.
\end{theorem}

\begin{example}[$n = 1, 2, 3$]
    ~\\
    \begin{itemize*}
            \item $n = 1$ \  $:$ \  $pf_1 = 1$\\
            \subitem $(1)$\\
            \item $n = 2$ \  $:$ \  $pf_2 = 3$\\
            \subitem $(1, 1)$
            \subitem $(1, 2)$
            \subitem $(2, 1)$\\
            \item $n = 3$ \  $:$ \  $pf_3 = 16$\\
            \subitem $(1, 1, 1)$
            \subitem $(1, 1, 2)$
            \subitem $(1, 1, 3)$
            \subitem $(1, 2, 1)$
            \subitem $(1, 2, 2)$
            \subitem $(1, 2, 3)$
            \subitem $(1, 3, 1)$
            \subitem $(1, 3, 2)$
            \subitem $(2, 1, 1)$
            \subitem $(2, 1, 2)$
            \subitem $(2, 1, 3)$
            \subitem $(2, 2, 1)$
            \subitem $(2, 3, 1)$
            \subitem $(3, 1, 1)$
            \subitem $(3, 1, 2)$
            \subitem $(3, 2, 1)$\\
    \end{itemize*}
\end{example}

\subsection{Primitive parking functions}

\begin{definition}[Primitive]
    A parking function $(a_1, a_2, \ldots, a_n)$ is said \emph{primitive} if
    it is already in non-decreasing order. \\
    We denote by $\mathcal{PF'}_n$ the set of primitive parking functions of length $n$.
    $$\mathcal{PF'} = \bigcup_{n > 0}{\mathcal{PF'}_n}$$
    
\end{definition}

\begin{example}
    \begin{align*}
        &f_1 = (1, 2, 2, 3) \in \mathcal{PF'}_4\\
        &f_2 = (1, 2, 3, 2) \notin \mathcal{PF'}_4
         \text{, even though } f_2 \in \mathcal{PF}_4\\
    \end{align*}
\end{example}

\begin{theorem}
    Let $pf'_n$ be the cardinal of $\mathcal{PF'}_n$.
    We have $$pf'_n = \frac{1}{n + 1} \binom{2n}{n}$$
    which is the $n^{th}$ Catalan number $Cat(n)$.
\end{theorem}

\begin{example}[$n = 1, 2, 3$]
    ~\\
    \begin{itemize*}
        \item $n = 1$ \  $:$ \  $pf'_1 = 1$\\
        \subitem $(1)$\\
        \item $n = 2$ \  $:$ \  $pf'_2 = 2$\\
        \subitem $(1, 1)$
        \subitem $(1, 2)$\\
        \item $n = 3$ \  $:$ \  $pf'_3 = 5$\\
        \subitem $(1, 1, 1)$
        \subitem $(1, 1, 2)$
        \subitem $(1, 1, 3)$
        \subitem $(1, 2, 2)$
        \subitem $(1, 2, 3)$\\
    \end{itemize*}
\end{example}
