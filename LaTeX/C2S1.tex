\section{Rational Parking Functions}

\begin{definition}[a, b - Parking Function]
    An \emph{a, b - parking function} is a sequence 
    $(a_1, a_2, \ldots, a_n)$ such that :\\
    \begin{itemize*}
        \item $n = a$\\
        \item its non-decreasing reordering 
        $(b_1, b_2, \ldots, b_n)$
        has $b_i \leqslant \frac{b}{a}(i-1) + 1$
        for all $i$.
    \end{itemize*}
\end{definition}

We denote by $\mathcal{PF}_{a,b}$ the set of 
a, b - parking functions.

\begin{example}
    ~\\
    \begin{itemize}
        \item Ex. 1 : $a > b$
            \subitem $a = 7$
            \subitem $b = 3$
            \subitem Limits of the non-decreasing
            reordering of any $f \in \mathcal{PF}_{7,3}$ :
            \subitem $[1,\ 1 \frac{3}{7},\ 1 \frac{6}{7},\ 
            2 \frac{2}{7},\ 2 \frac{5}{7},\ 3 \frac{1}{7},\ 
            3 \frac{4}{7}]$
            \subitem $f_1 = (2, 1, 1, 3, 2, 3, 1) \in
            \mathcal{PF}_{7,3}$
            \subitem $f_2 = (2, 1, 2, 3, 2, 3, 1) \notin
            \mathcal{PF}_{7,3}$, even though $f_2 \in
            \mathcal{PF}_7$
        \item Ex. 2 : $a < b$
            \subitem $a = 5$
            \subitem $b = 7$
            \subitem Limits of the non-decreasing            
            reordering of any $f \in \mathcal{PF}_{5,7}$ :
            \subitem $[1,\ 2 \frac{2}{5},\ 3 \frac{4}{5},\ 
            5 \frac{1}{5},\ 6 \frac{3}{5}]$
            \subitem $f_3 = (6, 3, 5, 1, 2) \in
            \mathcal{PF}_{5,7}$, even though $f_3 \notin
            \mathcal{PF}_5$
            \subitem $f_4 = (6, 3, 5, 1, 3) \notin
            \mathcal{PF}_{5,7}$\\
    \end{itemize}
\end{example}

\begin{theorem}[Armstrong, Loehr and Warrington, 2014]
    Let $pf_{a,b}$ be the cardinal of $\mathcal{PF}_{a,b}$.
    We have $$pf_{a,b} = b^{a-1}$$
\end{theorem}

\begin{example}[$a = 3, b = 5$]
    ~\\
    \begin{itemize*}\\
        \item $pf_{a,b} = 25$
        \item Limits : $[1,\ 2 \frac{2}{3},\ 
            4 \frac{1}{3}]$\\\\
        \subitem $(1, 1, 1)$
        \subitem $(1, 1, 2)$
        \subitem $(1, 1, 3)$
        \subitem $(1, 1, 4)$
        \subitem $(1, 2, 1)$
        \subitem $(1, 2, 2)$
        \subitem $(1, 2, 3)$
        \subitem $(1, 2, 4)$
        \subitem $(1, 3, 1)$
        \subitem $(1, 3, 2)$
        \subitem $(1, 4, 1)$
        \subitem $(1, 4, 2)$
        \subitem $(2, 1, 1)$
        \subitem $(2, 1, 2)$
        \subitem $(2, 1, 3)$
        \subitem $(2, 1, 4)$
        \subitem $(2, 2, 1)$
        \subitem $(2, 3, 1)$
        \subitem $(2, 4, 1)$
        \subitem $(3, 1, 1)$
        \subitem $(3, 1, 2)$
        \subitem $(3, 2, 1)$
        \subitem $(4, 1, 1)$
        \subitem $(4, 1, 2)$
        \subitem $(4, 2, 1)$\\
    \end{itemize*}
\end{example}

\begin{rem}    
    $\mathcal{PF}_{n, n+1} = \mathcal{PF}_n$.
    In fact, we do have $b^{a-1} = (n+1)^{n-1}$.
\end{rem}

Similarly to the integer case, we can define a notion of
\emph{primitivity} for rational parking functions.

\subsection{Rational primitive parking functions}

\begin{definition}[Rational Primitive]
    A rational parking function $f$ is said
    \emph{primitive} if it is already in
    non-decreasing order.
\end{definition}

We denote by $\mathcal{PF'}_{a,b}$ the set of
primitive a, b - parking functions.

\begin{example}[$a = 4, b = 3$]
    Limits : $[1,\ 1 \frac{3}{4},\ 2 \frac{1}{2},\ 
    3 \frac{1}{4}]$
    \begin{align*}
        &f_1 = (1, 1, 2, 2) \in \mathcal{PF'}_{4,3}\\
        &f_2 = (1, 1, 2, 1) \notin \mathcal{PF'}_{4,3},
        \text{ even though } f_2 \in \mathcal{PF}_{4,3}.
    \end{align*}
\end{example}

The following theorem can be seen as an extension of the main
result of \cite{ref10}, as we will see later that rational
primitive parking functions are in bijection with rational
Dyck paths.

\begin{theorem}
    Let $pf'_{a,b}$ be the cardinal of
    $\mathcal{PF'}_{a,b}$.
    We have $$\displaystyle pf'_{a,b} = 
    \frac{1}{a + b} \binom{a + b}{b}$$
\end{theorem}

which is the \emph{rational Catalan number} $Cat(a,b)$.

\begin{example}[$a = 3, b = 5$]
    ~\\
    \begin{itemize*}
        \item $pf'_{a,b} = 7$
        \item Limits : $[1,\ 2 \frac{2}{3},\ 
            4 \frac{1}{3}]$\\\\
        \subitem $(1, 1, 1)$
        \subitem $(1, 1, 2)$
        \subitem $(1, 1, 3)$
        \subitem $(1, 1, 4)$
        \subitem $(1, 2, 2)$
        \subitem $(1, 2, 3)$
        \subitem $(1, 2, 4)$\\
    \end{itemize*}    
\end{example}

\begin{rem}
    $\mathcal{PF'}_{n,n+1} = \mathcal{PF'}_n$.
    In fact, we do have
    $$\frac{1}{n + n + 1} \binom{n + n + 1}{n + 1}
    = \frac{1}{2n + 1} \binom {2n + 1}{n + 1}
    = \frac{1}{2n + 1} \frac{(2n + 1)!}{n ! (n+1)!}$$
    $$= \frac{(2n)!}{n!(n+1)!}
    = \frac{1}{n+1} \frac{(2n)!}{n!n!}
    = \frac{1}{n+1} \binom{2n}{n}$$
\end{rem}

In the same fashion as for classical parking functions, one
can define \emph{Rational Non-crossing Partitions} as a
bijecting combinatorial structure. Defined by Michelle Bodnar
in \cite{ref8} for any coprime $a$ and $b$, the construction
presented depends on a heavy mechanism relying on rational
Dyck paths. As there is -- to the best of our knowledge --
no easier way to define rational non-crossing partitions,
the following section has for only purpose to give an idea
of what such a partition looks like, and recall some of the
main results from \cite{ref8}.
