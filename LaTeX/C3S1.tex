\section{Parking Trees}

\begin{definition}[Parking Trees]
    A \emph{parking tree} is defined from a parking
    function $f = (a_1, \ldots, a_n) \in \mathcal{PF}_n$
    as follows :
    \begin{itemize}
        \item For $1 \leqslant i \leqslant n+1$, we define
            $s_i$ as $\{j\ |\ a_j = i\}$
        \item $[s_1, \ldots, s_{n+1}]$ describes the 
        pre-order depth-first traversal of the tree.
        \item Each node labeled by a set of size $k$
            has $k$ children. 
    \end{itemize}
\end{definition}

\begin{rem}
    The leaves of the tree are those corresponding to an
    element $i$ such that $1 \leqslant i \leqslant n + 1$,
    and $i$ is \emph{not} in $f$.\\
    Furthermore, as we will have $a$ total edges by 
    definition, the presence of a node corresponding
    to $n + 1$ is necessary, even though it will always
    be empty.
\end{rem}

\begin{example}[$n = 12$]
    ~\\
    \begin{itemize*}
        \item $f = (5, 7, 1, 3, 1, 8, 2, 7, 1, 3, 11, 11)$\\
        \item Labels : $[\{3, 5, 9\},\ \{7\},\ \{4, 10\},\ 
            \emptyset,\ \{1\},\ \emptyset,\ \{2,8\},\ 
            \{6\},\ \emptyset,\ \emptyset,\ \{11, 12\},\ 
            \emptyset, \emptyset]$
    \end{itemize*}

    \begin{center}
        \begin{tikzpicture}[scale=0.8]
            \node (1)  at (0,16)   {$\{3, 5, 9\}$};
            \node (2)  at (-6,12) {$\{7\}$};
            \node (3)  at (-6,8)  {$\{4, 10\}$};
            \node (5)  at (-5,4)   {$\{1\}$};
            \node (7)  at (0,12)   {$\{2, 8\}$};
            \node (8)  at (-2,8)   {$\{6\}$};
            \node (11) at (6,12)  {$\{11, 12\}$};

            \node (a) at (-1.5,16)    {$1$};
            \node (b) at (-7.5,12)    {$2$};
            \node (c) at (-7.5,8)     {$3$};
            \node (d) at (-7.5,6.15)  {$4$};
            \node (e) at (-6.5,4)     {$5$};
            \node (f) at (-6,2.15)    {$6$};
            \node (g) at (-1.5,12)    {$7$};
            \node (h) at (-3.5,8)     {$8$};
            \node (i) at (-3,6.15)    {$9$};
            \node (j) at (1.75,10.2)  {$10$};
            \node (k) at (7.75,12)    {$11$};
            \node (l) at (4.25,10.2)  {$12$};
            \node (m) at (7.75,10.2)  {$13$};

            \draw[ultra thick][color=brown!70!orange]
                (0,16)  circle (1);
            \draw[ultra thick][color=brown!70!orange]
                (-6,12) circle (1);
            \draw[ultra thick][color=brown!70!orange]
                (-6,8)  circle (1);
            \draw[ultra thick][color=brown!70!orange]
                (-5,4)  circle (1);
            \draw[ultra thick][color=brown!70!orange]
                (0,12)  circle (1);
            \draw[ultra thick][color=brown!70!orange]
                (-2,8)  circle (1);
            \draw[ultra thick][color=brown!70!orange]
                (6,12)  circle (1);

            \draw [->][ultra thick][color=brown!70!orange]
                (-0.5,15.15) to (-6,13.2);
            \draw [->][ultra thick][color=brown!70!orange]
                (0,15) to (0,13.2);
            \draw [->][ultra thick][color=brown!70!orange]
                (0.5,15.15) to (6,13.2);
            \draw [->][ultra thick][color=brown!70!orange]
                (-6,11) to (-6,9.2);
            \draw [->][ultra thick][color=brown!70!orange]
                (-0.25,11) to (-2,9.2);
            \draw [->][ultra thick][color=brown!70!orange]
                (-5.75,7) to (-5,5.2);

            \draw [-*][ultra thick][color=green!60!gray]
                (-6.25,7) to (-6.75,6);
            \draw [-*][ultra thick][color=green!60!gray]
                (-5,3) to (-5, 2);
            \draw [-*][ultra thick][color=green!60!gray]
                (-2,7) to (-2,6);
            \draw [-*][ultra thick][color=green!60!gray]
                (0.25,11) to (0.75,10);
            \draw [-*][ultra thick][color=green!60!gray]
                (5.75,11) to (5.25,10);
            \draw [-*][ultra thick][color=green!60!gray]
                (6.25,11) to (6.75,10);
        \end{tikzpicture}
    \end{center}    
\end{example}

Conversely, by reading the labels of a parking tree
depth-first in pre-order, we get the list of positions of
each number in the corresponding parking function, thus
creating a \emph{bijection}.

\begin{example}[From parking tree to parking function]
    ~
    \begin{center}
        \begin{tikzpicture}[scale=0.9]
            \node (1)  at (0,16)   {$\{5\}$};
            \node (2)  at (0,12)   {$\{2\}$};
            \node (3)  at (0,8)   {$\{1,3,4,7\}$};
            \node (6)  at (1,4)    {$\{8\}$};
            \node (7)  at (1,0)   {$\{6\}$};

            \node (a) at (-1.5,16)    {$1$};
            \node (b) at (-1.5,12)    {$2$};
            \node (c) at (-1.5,8)     {$3$};
            \node (d) at (-1.5,5.5)   {$4$};
            \node (e) at (-0.5,5.5)   {$5$};
            \node (f) at (-0.5,4)     {$6$};
            \node (g) at (-0.5,0)     {$7$};
            \node (h) at (0,-1.5)     {$8$};
            \node (i) at (1.5,5.5)    {$9$};

            \draw[ultra thick][color=brown!70!orange]
                (0,16)  circle (1);
            \draw[ultra thick][color=brown!70!orange]
                (0,12) circle (1);
            \draw[ultra thick][color=brown!70!orange]
                (0,8)  circle (1);
            \draw[ultra thick][color=brown!70!orange]
                (1,4)  circle (1);
            \draw[ultra thick][color=brown!70!orange]
                (1,0)  circle (1);

            \draw [->][ultra thick][color=brown!70!orange]
                (0,15) to (0,13.2);
            \draw [->][ultra thick][color=brown!70!orange]
                (0,11) to (0,9.2);
            \draw [->][ultra thick][color=brown!70!orange]
                (0.2,7) to (1,5.2);
            \draw [->][ultra thick][color=brown!70!orange]
                (1,3) to (1,1.2);

            \draw [-*][ultra thick][color=green!60!gray]
                (-0.5,7.15) to (-1.5,6);
            \draw [-*][ultra thick][color=green!60!gray]
                (-0.25,7.05) to (-0.5, 6);
            \draw [-*][ultra thick][color=green!60!gray]
                (0.5,7.15) to (1.5,6);
            \draw [-*][ultra thick][color=green!60!gray]
                (1,-1) to (1,-2);
        \end{tikzpicture}
    \end{center}  
\begin{itemize}
    \item The labels are $[\{5\},\ \{2\},\ \{1,3,4,7\},\ 
    \emptyset,\ \emptyset,\ \{8\},\ \{6\},\ \emptyset,\ 
    \emptyset]$.
    \item Thus the corresponding parking function is
        $(3,2,3,3,1,7,3,6) \in \mathcal{PF}_8$.
\end{itemize}
\end{example}