\section{Parking Trees}

\begin{definition}[Parking Trees]
    A \emph{parking tree} is defined from a parking
    function $f = (a_1, \ldots, a_n) \in \mathcal{PF}_n$
    as follows :
    \begin{itemize}
        \item For $1 \leqslant i \leqslant n+1$, we define
            $s_i$ as $\{j\ |\ a_j = i\}$
        \item $[s_1, \ldots, s_{n+1}]$ describes the 
        pre-order depth-first traversal of the tree.
        \item Each node labeled by a set of size $k$
            has $k$ children. 
    \end{itemize}
\end{definition}

\begin{rem}
    The leaves of the tree are those corresponding to an
    element $i$ such that $1 \leqslant i \leqslant n + 1$,
    and $i$ is \emph{not} in $f$.\\
    Furthermore, as we will have $a$ total edges by 
    definition, the presence of a node corresponding
    to $n + 1$ is necessary, even though it will always
    be empty.
\end{rem}

\begin{example}[$n = 12$]
    ~\\
    \begin{itemize*}
        \item $f = (5, 7, 1, 3, 1, 8, 2, 7, 1, 3, 11, 11)$\\
        \item Labels : $[\{3, 5, 9\},\ \{7\},\ \{4, 10\},\ 
            \emptyset,\ \{1\},\ \emptyset,\ \{2,8\},\ 
            \{6\},\ \emptyset,\ \emptyset,\ \{11, 12\},\ 
            \emptyset, \emptyset]$
    \end{itemize*}
    \input{fig/fig64.tex}    
\end{example}

Conversely, by reading the labels of a parking tree
depth-first in pre-order, we get the list of positions of
each number in the corresponding parking function, thus
creating a \emph{bijection}.

\begin{example}[From parking tree to parking function]
    \input{fig/fig65.tex}
    \begin{itemize}
        \item The labels are $[\{5\},\ \{2\},\ \{1,3,4,7\},\ 
        \emptyset,\ \emptyset,\ \{8\},\ \{6\},\ \emptyset,\ 
        \emptyset]$.
        \item Thus the corresponding parking function is
            $(3,2,3,3,1,7,3,6) \in \mathcal{PF}_8$.
    \end{itemize}
\end{example}