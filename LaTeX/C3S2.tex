\section{Rational Parking Trees}

\begin{definition}[Rational Parking Trees]
    A \emph{rational parking tree} is defined from a 
    rational parking function $f = (a_1, \ldots, a_a) 
    \in \mathcal{PF}_{a,b}$ as follows :
    \begin{itemize}
        \item For $1 \leqslant i \leqslant n+1$, we define
            the limit $l_i$ as the \emph{integer portion}
            of $\displaystyle \frac{b}{a}(i-1) + 1$.
            \subitem Let $l_0 = 0$.
        \item From these limits, we deduce the intervals
            $itv_i =\ ]l_{i-1}, l_i]$ for $1 \leqslant i
            \leqslant a+1$.
        \item For $1 \leqslant i \leqslant b + 1$, define
        $s_i$ as $\{j\ |\ a_j = i\}$.
        \item $[s_1, \ldots, s_{b+1}]$ describes the
        pre-order depth-first traversal of the tree.
        \item Each node labeled by a set of size $k$
            has $k$ \emph{groups} of children, which are
            defined by the intervals. 
    \end{itemize}
\end{definition}

\begin{example}[$a<b$]
    ~
    \begin{itemize}
        \item $a = 7$
        \item $b = 9$
        \item Limits : $[1,\ 2 \frac{2}{7},\ 
            3 \frac{4}{7},\ 4 \frac{6}{7},\  
            6 \frac{1}{7},\ 7 \frac{3}{7},\ 
            8 \frac{5}{7},\ 10]$
        \item Integral limits : $[0,1,2,3,4,6,7,8,10]$
        \item Intervals :
            \subitem $]0, 1]$ \hspace{5mm} $]1, 2]$
            \hspace{5mm} $]2, 3]$ \hspace{5mm} $]3, 4]$
            \hspace{5mm} $]4, 6]$ \hspace{5mm} $]6, 7]$
            \hspace{5mm} $]7, 8]$ \hspace{5mm} $]8, 10]$
        \item Children groups :
            \subitem $[1]$ \hspace{5mm} $[2]$ \hspace{5mm}
            $[3]$ \hspace{5mm} $[4]$ \hspace{5mm}
            $[5,6]$ \hspace{5mm} $[7]$ \hspace{5mm} $[8]$
        \item $f = (6,2,6,1,4,7,2)$
        \item Labels : $\{\{4\},\ \{2,7\},\ \emptyset,\ 
            \{5\},\ \emptyset,\ \{1,3\},\ \{6\},\ 
            \emptyset,\ \emptyset,\ \emptyset\}$\\
    \end{itemize}
    
    \begin{center}
        \begin{tikzpicture}[scale=0.8]
            \node (1)  at (0,17)   {$\{4\}$};
            \node (2)  at (0,13)   {$\{2,7\}$};
            \node (4)  at (3,9)   {$\{5\}$};
            \node (6)  at (6,5)    {$\{1,3\}$};
            \node (7)  at (3,1)    {$\{6\}$};

            \node (a) at (-1.5,17)    {$1$};
            \node (b) at (-1.5,13)    {$2$};
            \node (c) at (-2,11)      {$3$};
            \node (d) at (1.5,9)      {$4$};
            \node (e) at (1,7)        {$5$};
            \node (f) at (4.5,5)      {$6$};
            \node (g) at (1.5,1)      {$7$};
            \node (h) at (2.5,-1)     {$8$};
            \node (i) at (5.5,3)      {$9$};
            \node (j) at (8.5,3)      {$10$};

            \node[right] (ac) at (1.5,17) {$1$ children
                group : $[2]$};
            \node[right] (bc) at (1.5,13) {$2$ children
                groups : $[3]$ and $[4]$};
            \node[right] (dc) at (4.5,9) {$1$ children
                group : $[5,6]$};
            \node[right] (fc) at (7.5,5) {$2$ children
                groups : $[7]$ and $[9,10]$};
            \node[right] (gc) at (4.5,1) {$1$ children
                group : $[8]$};

            \draw[ultra thick][color=brown!70!orange]
                (0,17)  circle (1);
            \draw[ultra thick][color=brown!70!orange]
                (0,13) circle (1);
            \draw[ultra thick][color=brown!70!orange]
                (3,9)  circle (1);
            \draw[ultra thick][color=brown!70!orange]
                (6,5)  circle (1);
            \draw[ultra thick][color=brown!70!orange]
                (3,1)  circle (1);

            \draw [->][ultra thick][color=brown!70!orange]
                (0,16) to (0,14.2);
            \draw [->][ultra thick][color=brown!70!orange]
                (0.6,12.25) to (2.7,10.2);
            \draw [->][ultra thick][color=brown!70!orange]
                (3.6,8.25) to (5.7,6.2);
            \draw [->][ultra thick][color=brown!70!orange]
                (5.4,4.25) to (3.3,2.2);

            \draw [-*][ultra thick][color=green!60!gray]
                (-0.6,12.2) to (-1.6,11);
            \draw [-*][ultra thick][color=green!60!gray]
                (2.4,8.2) to (1.4,7);
            \draw [-*][ultra thick][color=green!60!gray]
                (3,0) to (3,-1);
            \draw [-*][ultra thick][color=green!60!gray]
                (6,4) to (6,3);
            \draw [-*][ultra thick][color=green!60!gray]
                (6.6,4.2) to (8,3);
        \end{tikzpicture}
    \end{center}
\end{example}

\begin{example}[$a>b$]
    ~
    \begin{itemize}
        \item $a = 9$
        \item $b = 7$
        \item Limits : $[1,\ 1 \frac{7}{9},\ 
            2 \frac{5}{9},\ 3 \frac{3}{9},\  
            4 \frac{1}{9},\ 4 \frac{8}{9},\ 
            5 \frac{6}{9},\ 6 \frac{4}{9},\ 
            7 \frac{2}{9},\ 8]$
        \item Integral limits : $[0,1,1,2,3,4,4,5,6,7,8]$
        \item Intervals :
            \subitem $]0, 1]$ \hspace{5mm} $]1, 1]$
            \hspace{5mm} $]1, 2]$ \hspace{5mm} $]2, 3]$
            \hspace{5mm} $]3, 4]$ 
            \subitem $]4, 4]$ \hspace{5mm} $[4, 5]$
            \hspace{5mm} $]5, 6]$ \hspace{5mm} $]6, 7]$
            \hspace{5mm} $]7, 8]$
        \item Children groups :
            \subitem $[1]$ \hspace{5mm} $\emptyset$ 
            \hspace{5mm} $[2]$ \hspace{5mm} $[3]$
            \hspace{5mm} $[4]$ \hspace{5mm} $\emptyset$
            \hspace{5mm} $[5]$ \hspace{5mm} $[6]$
            \hspace{5mm} $[7]$ \hspace{5mm} $[8]$
        \item $f = (4,2,2,1,4,5,7,4,1)$
        \item Labels : $\{\{4,9\},\ \{2,3\},\ \emptyset,\ 
            \{1,5,8\}, \{6\},\ \emptyset,\ \{7\},\ 
            \emptyset\}$\\
    \end{itemize}
    \begin{center}
    \begin{tikzpicture}[scale=0.8]
        \node (1)  at (0,17)   {$\{4,9\}$};
        \node (2)  at (0,13)   {$\{2,3\}$};
        \node (4)  at (3,9)   {$\{1,5,8\}$};
        \node (5)  at (0,5)    {$\{6\}$};
        \node (7)  at (6,5)    {$\{7\}$};

        \node (a) at (-1.5,17)    {$1$};
        \node (b) at (-1.5,13)    {$2$};
        \node (c) at (-2,11)      {$3$};
        \node (d) at (1.5,9)      {$4$};
        \node (e) at (1.5,5)      {$5$};
        \node (f) at (-1,3)       {$6$};
        \node (g) at (4.5,5)      {$7$};
        \node (h) at (5,3)        {$8$};

        \node[right] (ac) at (1.5,17) {$2$ children
            groups : $\emptyset$ and $[2]$};
        \node[right] (bc) at (1.5,13) {$2$ children
            groups : $[3]$ and $[4]$};
        \node[right] (dc) at (4.5,9) {$3$ children
            groups : $\emptyset$, $[5]$ and $[7]$};
        \node[left] (fc) at (-1.5,5) {$1$ children
            group : $[6]$};
        \node[right] (gc) at (7.5,5) {$1$ children
            group : $[8]$};

        \draw[ultra thick][color=brown!70!orange]
            (0,17)  circle (1);
        \draw[ultra thick][color=brown!70!orange]
            (0,13) circle (1);
        \draw[ultra thick][color=brown!70!orange]
            (3,9)  circle (1);
        \draw[ultra thick][color=brown!70!orange]
            (0,5)  circle (1);
        \draw[ultra thick][color=brown!70!orange]
            (6,5)  circle (1);

        \draw [->][ultra thick][color=brown!70!orange]
            (0,16) to (0,14.2);
        \draw [->][ultra thick][color=brown!70!orange]
            (0.6,12.25) to (2.7,10.2);
        \draw [->][ultra thick][color=brown!70!orange]
            (2.4,8.25) to (0.3,6.2);
        \draw [->][ultra thick][color=brown!70!orange]
            (3.6,8.25) to (5.7,6.2);

        \draw [-*][ultra thick][color=green!60!gray]
            (-0.6,12.2) to (-1.6,11);
        \draw [-*][ultra thick][color=green!60!gray]
            (0,4) to (0,3);
        \draw [-*][ultra thick][color=green!60!gray]
            (6,4) to (6,3);
    \end{tikzpicture}
\end{center}
\end{example}

In both cases, the converse direction of the
\emph{bijection} is obtained with the same method as for
classical parking trees.