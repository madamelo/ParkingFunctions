\section{Rational Parking Trees}

\begin{definition}[Rational Parking Trees]
    A \emph{rational parking tree} is defined from a 
    rational parking function $f = (a_1, \ldots, a_a) 
    \in \mathcal{PF}_{a,b}$ as follows :
    \begin{itemize}
        \item For $1 \leqslant i \leqslant n+1$, we define
            the limit $l_i$ as the \emph{integer portion}
            of $\displaystyle \frac{b}{a}(i-1) + 1$.
            \subitem Let $l_0 = 0$.
        \item From these limits, we deduce the intervals
            $itv_i =\ ]l_{i-1}, l_i]$ for $1 \leqslant i
            \leqslant a+1$.
        \item For $1 \leqslant i \leqslant b + 1$, define
        $s_i$ as $\{j\ |\ a_j = i\}$.
        \item $[s_1, \ldots, s_{b+1}]$ describes the
        pre-order depth-first traversal of the tree.
        \item Each node labeled by a set of size $k$
            has $k$ \emph{groups} of children, which are
            defined by the intervals. 
    \end{itemize}
\end{definition}

\begin{example}[$a<b$]
    ~
    \begin{itemize}
        \item $a = 7$
        \item $b = 9$
        \item Limits : $[1,\ 2 \frac{2}{7},\ 
            3 \frac{4}{7},\ 4 \frac{6}{7},\  
            6 \frac{1}{7},\ 7 \frac{3}{7},\ 
            8 \frac{5}{7},\ 10]$
        \item Integral limits : $[0,1,2,3,4,6,7,8,10]$
        \item Intervals :
            \subitem $]0, 1]$ \hspace{5mm} $]1, 2]$
            \hspace{5mm} $]2, 3]$ \hspace{5mm} $]3, 4]$
            \hspace{5mm} $]4, 6]$ \hspace{5mm} $]6, 7]$
            \hspace{5mm} $]7, 8]$ \hspace{5mm} $]8, 10]$
        \item Children groups :
            \subitem $[1]$ \hspace{5mm} $[2]$ \hspace{5mm}
            $[3]$ \hspace{5mm} $[4]$ \hspace{5mm}
            $[5,6]$ \hspace{5mm} $[7]$ \hspace{5mm} $[8]$
        \item $f = (6,2,6,1,4,7,2)$
        \item Labels : $\{\{4\},\ \{2,7\},\ \emptyset,\ 
            \{5\},\ \emptyset,\ \{1,3\},\ \{6\},\ 
            \emptyset,\ \emptyset,\ \emptyset\}$\\
    \end{itemize}
    \input{fig/fig66.tex}
\end{example}

\begin{example}[$a>b$]
    ~
    \begin{itemize}
        \item $a = 9$
        \item $b = 7$
        \item Limits : $[1,\ 1 \frac{7}{9},\ 
            2 \frac{5}{9},\ 3 \frac{3}{9},\  
            4 \frac{1}{9},\ 4 \frac{8}{9},\ 
            5 \frac{6}{9},\ 6 \frac{4}{9},\ 
            7 \frac{2}{9},\ 8]$
        \item Integral limits : $[0,1,1,2,3,4,4,5,6,7,8]$
        \item Intervals :
            \subitem $]0, 1]$ \hspace{5mm} $]1, 1]$
            \hspace{5mm} $]1, 2]$ \hspace{5mm} $]2, 3]$
            \hspace{5mm} $]3, 4]$ 
            \subitem $]4, 4]$ \hspace{5mm} $[4, 5]$
            \hspace{5mm} $]5, 6]$ \hspace{5mm} $]6, 7]$
            \hspace{5mm} $]7, 8]$
        \item Children groups :
            \subitem $[1]$ \hspace{5mm} $\emptyset$ 
            \hspace{5mm} $[2]$ \hspace{5mm} $[3]$
            \hspace{5mm} $[4]$ \hspace{5mm} $\emptyset$
            \hspace{5mm} $[5]$ \hspace{5mm} $[6]$
            \hspace{5mm} $[7]$ \hspace{5mm} $[8]$
        \item $f = (4,2,2,1,4,5,7,4,1)$
        \item Labels : $\{\{4,9\},\ \{2,3\},\ \emptyset,\ 
            \{1,5,8\}, \{6\},\ \emptyset,\ \{7\},\ 
            \emptyset\}$\\
    \end{itemize}
    \input{fig/fig67.tex}
\end{example}

In both cases, the converse direction of the
\emph{bijection} is obtained with the same method as for
classical parking trees.